\documentclass[a4paper]{article}
\usepackage[utf8]{inputenc}
\usepackage{amsmath}
\usepackage{amssymb}
\usepackage{caption}
\usepackage{mathtools}
\usepackage{amsfonts}
\usepackage{lastpage}
\usepackage{tikz}
\usepackage{float}
\usepackage{textcomp}
\usetikzlibrary{patterns}
\usepackage{pdfpages}
\usepackage{gauss}
\usepackage{fancyvrb}
\usepackage[table]{colortbl}
\usepackage{fancyhdr}
\usepackage{graphicx}
\usepackage[margin=2.5 cm]{geometry}

\definecolor{listinggray}{gray}{0.9}
\usepackage{listings}
\lstset{
	language=,
	literate=
		{æ}{{\ae}}1
		{ø}{{\o}}1
		{å}{{\aa}}1
		{Æ}{{\AE}}1
		{Ø}{{\O}}1
		{Å}{{\AA}}1,
	backgroundcolor=\color{listinggray},
	tabsize=3,
	rulecolor=,
	basicstyle=\scriptsize,
	upquote=true,
	aboveskip={0.2\baselineskip},
	columns=fixed,
	showstringspaces=false,
	extendedchars=true,
	breaklines=true,
	prebreak =\raisebox{0ex}[0ex][0ex]{\ensuremath{\hookleftarrow}},
	frame=single,
	showtabs=false,
	showspaces=false,
	showlines=true,
	showstringspaces=false,
	identifierstyle=\ttfamily,
	keywordstyle=\color[rgb]{0,0,1},
	commentstyle=\color[rgb]{0.133,0.545,0.133},
	stringstyle=\color[rgb]{0.627,0.126,0.941},
  moredelim=**[is][\color{blue}]{@}{@},
}

\lstdefinestyle{base}{
  emptylines=1,
  breaklines=true,
  basicstyle=\ttfamily\color{black},
}

\pagestyle{fancy}
\def\checkmark{\tikz\fill[scale=0.4](0,.35) -- (.25,0) -- (1,.7) -- (.25,.15) -- cycle;}
\newcommand*\circled[1]{\tikz[baseline=(char.base)]{
            \node[shape=circle,draw,inner sep=2pt] (char) {#1};}}
\newcommand*\squared[1]{%
  \tikz[baseline=(R.base)]\node[draw,rectangle,inner sep=0.5pt](R) {#1};\!}
\newcommand{\comment}[1]{%
  \text{\phantom{(#1)}} \tag{#1}}
\def\el{[\![}
\def\er{]\!]}
\def\dpip{|\!|}
\def\MeanN{\frac{1}{N}\sum^N_{n=1}}
\cfoot{Page \thepage\ of \pageref{LastPage}}
\DeclareGraphicsExtensions{.pdf,.png,.jpg}
\author{Nikolaj Dybdahl Rathcke (rfq695)}
\title{Proactive Computer Security \\ Assignment 5}
\lhead{PCS}
\rhead{Assignment 5}

\begin{document}
\maketitle

\section{Fuzzer}
The fuzzer is found \texttt{fuzzer.py}. The fuzzer I have implemented is a very simple one and the approach for its construction is very simple as well. I started by making a random byte string of a fixed length and then slowly increasing (by $1$) the length of it if it did not cause any crash. The string is always stripped for new line bytes, \texttt{NULL} bytes and spaces as I suspected these might mess up the fuzzing. It would send the random byte string through all the available commands and when it started causing crashes, I used the process of elimination to realize my crash came from the \texttt{JOINING} command. Obviously, the crash could be caused by other commands as well, but I knew this one worked. Thus, all the fuzzer does is creating a random byte string of length $1000$ and then incrementing the length by one until it crashes (random string for  each iteration), that is, until a \texttt{core} file is produced. I have commented the fuzzer, so it should make sense as it's also very simple. \\
Note that the connected is closed and started for each iteration in order to join a channel again. If the connection is closed for some reason, we reestablish it as well.

\section{Findings}
Usually, the fuzzer has to run for under $50$ iterations before crashing. The line that succeeded to crash it and make a \texttt{core} file is written to a file \texttt{ded.txt} (you can toggle the \texttt{autowin} variable if you want to see the fuzzer in action as right now it just uses one that worked located in \texttt{ded2.txt}). \\
If we fire up \texttt{gdb} on the core produced by the string in \texttt{ded2.txt} (it's called \texttt{core2} because the fuzzer runs until there is a file named \texttt{core} in the path):
\begin{verbatim}
    $ gdb yairc core2
\end{verbatim}
We can see that it has a segmentation fault at \texttt{0x15ad13fe} (if we use the line in \texttt{ded2.txt}). If we take a look at the input we gave and use a little time on this, we can see that it actually the address actually correspond to the bytes on indices $1005$ to $1008$ (which we can verify by running it again). Thus, we can change this to wherever we want in the memory by constructing a string that crashes the thread with the desired address on these indices of the string.

\end{document}
