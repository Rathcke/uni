\documentclass[a4paper]{article}
\usepackage[utf8]{inputenc}
\usepackage[fleqn]{amsmath}
\usepackage{algorithmicx}
\usepackage{algorithm}
\usepackage{algpseudocode}
\usepackage{amssymb}
\usepackage{pdfpages}
\usepackage{mathtools}
\usepackage{amsfonts}
\usepackage{epstopdf}
\usepackage{lastpage}
\usepackage{tikz}
\usepackage{float}
\usepackage{textcomp}
\usetikzlibrary{patterns}
\usepackage{pdfpages}
\usepackage{gauss}
\usepackage{fancyvrb}
\usepackage[table]{colortbl}
\usepackage{fancyhdr}
\usepackage{graphicx}
\usepackage{caption}
\usepackage[margin=1in]{geometry}
\usepackage{subcaption}
\delimitershortfall-1sp
\newcommand\abs[1]{\left|#1\right|}

\definecolor{listinggray}{gray}{0.9}
\usepackage{listings}
\lstset{
	language=,
	literate=
		{æ}{{\ae}}1
		{ø}{{\o}}1
		{å}{{\aa}}1
		{Æ}{{\AE}}1
		{Ø}{{\O}}1
		{Å}{{\AA}}1,
	backgroundcolor=\color{listinggray},
	tabsize=2,
	rulecolor=,
	basicstyle=\scriptsize,
	upquote=true,
	aboveskip={0.2\baselineskip},
	columns=fixed,
	showstringspaces=false,
	extendedchars=true,
	breaklines=true,
	prebreak =\raisebox{0ex}[0ex][0ex]{\ensuremath{\hookleftarrow}},
	frame=single,
	showtabs=false,
	showspaces=false,
	showlines=true,
	showstringspaces=false,
	identifierstyle=\ttfamily,
    keywordstyle=\color[rgb]{0,0,1},
	commentstyle=\color[rgb]{0.133,0.545,0.133},
	stringstyle=\color[rgb]{0.627,0.126,0.941},
  moredelim=**[is][\color{blue}]{@}{@},
}
\newcommand{\comment}[1]{%
  \text{\phantom{(#1)}} \tag{#1}}
\lstdefinestyle{base}{
  emptylines=1,
  breaklines=true,
  basicstyle=\ttfamily\color{black},
}

\pagestyle{fancy}
\def\checkmark{\tikz\fill[scale=0.4](0,.35) -- (.25,0) -- (1,.7) -- (.25,.15) -- cycle;}
\def\E{\mbox{\textbf{E}}}
\def\Pr{\mbox{\textbf{Pr}}}
\newcommand*\circled[1]{\tikz[baseline=(char.base)]{
            \node[shape=circle,draw,inner sep=2pt] (char) {#1};}}
\newcommand*\squared[1]{%
  \tikz[baseline=(R.base)]\node[draw,rectangle,inner sep=0.5pt](R) {#1};\!}
\cfoot{Page \thepage\ of \pageref{LastPage}}
\DeclareGraphicsExtensions{.pdf,.png,.jpg,.eps}
\graphicspath{{image/}}
\lhead{RA}
\rhead{Chap 8}
\begin{document}
\section{Fundamental Data-structuring Problem}
Problem: Maintain a collction of \(\{S_1,S_2,..\}\) of sets of item so as to effeciently support
\begin{itemize}
\item make(S): creats empty structure S
\item Insert(i,S): insert i in S
\item Delete(k,S): delete the item indexed by k from 
\item Find(k,S) return item indeced by k in S
\item join(\(S_1,i,S_2\)): relplace the strucs \(S_1\) and \(S_2\) by new struc \(S=S_1\cup i\S_2\) st \(\forall j\in S_1. k(j) < k(i)\) and \(\forall j \in S_2. k(i) < k(j)\).
\item paste(\(S_1,S_2\)): replace \(S_1,S_2\) by the new struc \(S = S_1 \cup S_2\) st \(\forall i\in S_1 \wedge j \in S_2. k(i) < k(j)\).
\item Split(k,S): replace struc S by structs \(S_1\) and \(S_2\) st \(S_1 \{j\in S | k(j) < k\} \wedge S_2 = \{j \in S | k < k(j)\}\).
\end{itemize}

\section{Random Treaps}
Binary search tree with each node assigned random priority \(p(v)\), st it  maintains search order for keys and heap order for priorities.\\
\textbf{Thrm}: Let \(S=\{(k_1,p_1),..,(k_n,p_n)\}\) be any set of distinct key-prority pairs. Then, there exists a unique treap \(T(S)\) for it.\\
Proof: Recursive contruction, choose highest prority pair \(k_i,p_i\) in S, make it root. Construct right subtree for \(k > k_i\) and left subtree for \(k < k_i\) in same manner. Since all priorities are distinct the treap is unique.
\textbf{Lemma}: Let \(T\) be a random treap for a set \(S\) of size n. For elements \(x \in S\) with rank k,
\[E[depth(x)] = H_k+H_{n-k-1} - 1\]
Proof:
Same as Expected \(depth(x_k)\). Define \(X_{ik} = 1\) iff. \(x_i\) is an ancestor of \(x_k\).\\
For \(i < k\) \(depth(x_k)=\sum_{i\neq k}X_{ik}\).\\
 \(X_{ik} \Leftrightarrow p_i = \max \{p_i,..,p_k\}\), otherwise there would exist \(p_j\) which would be common ancestor of \(x_i\) and \(x_k\), which implies they are in different subtrees of \(x_j\).\\
From QuickSort analysis we have \(\Pr[X_{ik}] = 1/(k-i-1) \Rightarrow X=\sum_{i<k} X_{ik} = \sum_{i<k} 1/(k-i-1) = \sum_{d=2}^k 1/d = H_k -1\). \\
\(H_{n-k+1} -1\) from \(k<t\) case, and one from \(x_i\). So \(E[depth(x_i)] = (H_k - 1)+(H_{n-k+1} -1) +1\), \(+1\) from \(x_k\). \\
\(H_k\) harmonic number, so expected \(O(lg n)\) search time. \\ 
\textbf{Thrm} Expected number of rotations for Insert and Delete is 2.
Proof: case rank \(i<k\):\\
Ancestors when \(p_k=-\infty\) are \(p_i = \max \{p_i,..,p_k,-\infty\}\)\\
Ancestors before \(p_i = \max\{p_i,..,p_{k-1}\}\)\\
Let \(Y_{ik}=1\) 1 iff \(p_k > (p_i = \max\{p_i,..,p_{k-1}\})\), that is \(p_i\) became an ancestor after reduction of priority.
\[\Pr[Y_{ik} = Pr[p_k= \max \{p_i,..,p_k\}]\Pr[p_i = \{p_i,..,p_{k-1}] = \frac{1}{k-i+1}\cdot\frac{1}{k-i}\]
By same analysis as QuickSort.\\
So we have the expected number of new ancestors is given by
\[E[\sum_{i<k} Y_{ik}] = \sum_{i<k} \frac{1}{(k-i-1)(k-i)} = \sum_1^{k-1} \frac{1}{(d+1)d} = 1 - \frac{1}{k}\]
We can perform a similar analysis for ranks \(t > k\) and find
\[E[\sum_{t \geq k} Y_{ik}] = 1-\frac{1}{n-k+1}\]
From this we can see that
\[\sum_{i\neq k} Y_{ik} = 2 - 1/k - 1/(n-k+1) < 2\]

\section{Hashing \(O(1)\) Search Time}
Concept: Use primary hash-function with low collios
\end{document}
