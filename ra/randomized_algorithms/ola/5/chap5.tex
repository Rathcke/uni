\documentclass[a4paper]{article}
\usepackage[utf8]{inputenc}
\usepackage[fleqn]{amsmath}
\usepackage{algorithmicx}
\usepackage{algorithm}
\usepackage{algpseudocode}
\usepackage{amssymb}
\usepackage{pdfpages}
\usepackage{mathtools}
\usepackage{amsfonts}
\usepackage{epstopdf}
\usepackage{lastpage}
\usepackage{tikz}
\usepackage{float}
\usepackage{textcomp}
\usetikzlibrary{patterns}
\usepackage{pdfpages}
\usepackage{gauss}
\usepackage{fancyvrb}
\usepackage[table]{colortbl}
\usepackage{fancyhdr}
\usepackage{graphicx}
\usepackage{caption}
\usepackage[margin=1in]{geometry}
\usepackage{subcaption}
\delimitershortfall-1sp
\newcommand\abs[1]{\left|#1\right|}

\definecolor{listinggray}{gray}{0.9}
\usepackage{listings}
\lstset{
	language=,
	literate=
		{æ}{{\ae}}1
		{ø}{{\o}}1
		{å}{{\aa}}1
		{Æ}{{\AE}}1
		{Ø}{{\O}}1
		{Å}{{\AA}}1,
	backgroundcolor=\color{listinggray},
	tabsize=2,
	rulecolor=,
	basicstyle=\scriptsize,
	upquote=true,
	aboveskip={0.2\baselineskip},
	columns=fixed,
	showstringspaces=false,
	extendedchars=true,
	breaklines=true,
	prebreak =\raisebox{0ex}[0ex][0ex]{\ensuremath{\hookleftarrow}},
	frame=single,
	showtabs=false,
	showspaces=false,
	showlines=true,
	showstringspaces=false,
	identifierstyle=\ttfamily,
	keywordstyle=\color[rgb]{0,0,1},
	commentstyle=\color[rgb]{0.133,0.545,0.133},
	stringstyle=\color[rgb]{0.627,0.126,0.941},
  moredelim=**[is][\color{blue}]{@}{@},
}
\newcommand{\comment}[1]{%
  \text{\phantom{(#1)}} \tag{#1}}
\lstdefinestyle{base}{
  emptylines=1,
  breaklines=true,
  basicstyle=\ttfamily\color{black},
}

\pagestyle{fancy}
\def\checkmark{\tikz\fill[scale=0.4](0,.35) -- (.25,0) -- (1,.7) -- (.25,.15) -- cycle;}
\def\E{\mbox{\textbf{E}}}
\def\Pr{\mbox{\textbf{Pr}}}
\newcommand*\circled[1]{\tikz[baseline=(char.base)]{
            \node[shape=circle,draw,inner sep=2pt] (char) {#1};}}
\newcommand*\squared[1]{%
  \tikz[baseline=(R.base)]\node[draw,rectangle,inner sep=0.5pt](R) {#1};\!}
\cfoot{Page \thepage\ of \pageref{LastPage}}
\DeclareGraphicsExtensions{.pdf,.png,.jpg,.eps}
\graphicspath{{image/}}
\author{Ola Rønning (vdl761)\\Tobias Hallundbæk Petersen (xtv657)}
\title{Randomized Algorithms \\ Week 6}
\lhead{RA}
\rhead{Week 6}
\begin{document}
\textbf{Concepts}: Probabilistic method, probability amplification, Locasz local lemma, conditional probabilities.\\
\section{Probabiltic Method}
\begin{enumerate}
\item Any ran. var. \(x\) assume \(\geq 1\) value \(\geq E[x]\) and assume \(\geq 1\) value \(\leq E[x]\).
\item If \(o\) chosen at random from \(U\) satifies property \(p\) with non-zero probability, the there \(\exists o \in U\) with property \(p\).
\end{enumerate}
Application: Max Cut\\
Thrm: For any grah \(G(V,E)\), \(|V|=n\), \(|E|=m\), there exist partition of \(V\) to \(A,B\) st.
\[|\{(u,v)|u\in A \wedge v\in B\}| \geq m/2\]
Consider assign \(v\in V\) at random to \(A\) or \(B\) with \(1/2\) prob. \(\Rightarrow\) Prob \((u,v)=u\in A \wedge v\in B = 1/2\). By linarity of expectation we expect \(m/2\) edges satisfying \(p=(u,v)=u\in A \wedge v\in B\). \(\Rightarrow \exists p\) satisfying the thrm by Probabilistic Method 2.\\
Not necessarily an efficient ran. alg., if prob. is miniscule.

\section{Max-Sat}
Max-Sat:\(m\) clauses in \(cnf\) over \(n\) vars.\\
Problem: assign vars values st. max num. of cluase are satisfied.\\
\textbf{Thrm}: For any set of \(m\) clauses, there is a thruth assign. st var satisfy \(\geq m/2\) clauses.\\
Proof: Ran assign var to 0,1 indp. and equiprob. For \(1\leq i \leq ,\), \(Z_i=1\) iff \(i\)'th clause satisfied. With \(k\) literals, \(\Pr[\bar{Z}_i]=2^{-k}\), since all literals must zero. \(\Rightarrow \Pr(Z_i) = 1 - 2^{-k} \geq 1/2\) \(\Rightarrow \forall i.E[Z_i] \geq 1/2\). By linarity of expectation 
\[E[\sum_{i=1}^m Z_i \geq m/2\]
By probabilistic method 1. \(\exists\) assign. st \(\sum_{i=1}^m Z_i \geq m/2\).\\
\(3/4\)-approx Max Sat.\\
\(\alpha\)-ran-approx alg: for instance \(I\) \(m_{\cdot}(I)\) max num of satif. clauses \(E[M_A(I)]\) expected num of satisf. clause by \(\mathcal{A}\)
\[ inf_{I\in \mathcal{I}} \frac{E[m_A(I)]}{m_{\cdot}(I)}=\alpha\]
Thrm 1 is \(1-2^{-k}\)-approx alg if \(\forall i |C_i| \geq k\) \(\Rightarrow k\geq 2\) Thrm 1 \(\geq 3/4\)-approx alg.\\
Issue \(k=1\) (\(1/2\) approx)\\
Solution: New randomized rounding alg. Run both, return better.\\
Formulate problem as Linear programming relaxation and use randomize rounding.\\
Let \(z_j\in \{0,1\}\) be ind. sat. \(C_j\). For each var \(x_i\) let \(y_i\) be indp. st. \(y_i=1\) iff \(x_i=TRUE\). \(C_j^+\) be set pf indeces in \(C_j\) of vars in uncomplemented form and let \(C_j^-\) be indeces of complemented vars.\\
Linear Program
\[\max \sum_{j=1}^m z_j\\y_i,z_j \in \{0,1\}\]
subject to
\[\sum_{i\in C^+_j}y_i + \sum_{i \in C^-_j}(1-y_i) \geq z_j\]
Relaxation: \(y_i, z_j \in [0,1\). Let \(\hat{y}_i\) be relaxed \(y_i\) when solved. \(\hat{z}_j\) be val obt. for \(z_j \Rightarrow \max \sum_j \hat{z}_j \geq \max \sum_j z_j\).\\
Show \(E[Z] \geq (1-1/e)\sum_j \hat{z}_j\), then show best of two algs atleast \(3/4 \sum_j \hat{z}_j\).\\
Let \(\Pr[y_i=1] = \hat{y}_i\) and 
\[\beta_k = 1-(1-k^{-1})^k\]
Note \(\beta_k \geq (1-1/e)\) for all \(k \in \mathcal{Z}^+\).\\
\textbf{Lemma 1}: Let \(c_j\) be a clause with \(k\) literals. The prob that it is satis. by rand rounding is \(\geq \beta_k\hat{z}_j\)\\
Proof: Focus on \(c_j\) and assume \(c_j^+=c_j\), assume \(x_1 \vee .. \vee x_k\).
By constraint \(\hat{y}_1+..+\hat{y}_k \geq \hat{z}_j\). \(c_j\) remains unsatis if all \(y_i\) are rounded to zero.\\
\(\Pr[c_j=FALSE]=\prod_{i=1}^k(1-\hat{y}_i)\) since ran rounding is indep.\\
Remain to show
\[ 1-\prod_{i=1}^k (1 - \hat{y}_i) \geq \beta_k \hat{z}_j\]
LHS min when \(\hat{y}_i = \hat{z}_j/k \Rightarrow\)
\[1-(1-x/k)^k \geq \beta_{k} x\]
for \(x\in[0,1]\). Note \(g(x) = 1-(1-x/k)^k\) is concave, so it remains to show \(f(0)\geq g(0)\) and \(f(1) \geq g(1)\), for \(f(x)=\beta_k x\). Simple calc gives \(g(0)=f(0)=0\) and \(f(1)=g(1)=1-(1-1/k)^k\).\\
\textbf{Thrm 3}: Given instance \(I\) of Max-Sat. \(E[satis clauses]\) by a Linear Prog with randomized rounding is (1-1/e) times the max num of number clauses satifyable on \(I\).\\
Proof: Lemma 1 over all \(j\) with linearity of expectation.\\
\textbf{Thrm 4}: Let \(n_1\) be the expected of alg from thrm 1. Let \(n_2\) be expected of alg from Thrm 2 then 
\[\max(n_1,n_2)\geq 3/4 \sum_j \hat{z}_j\]
Proof: Suffice to show  \((n_1 + n_2)/2 \geq 3/4 \sum \hat{z}_j\).\\
Denote by \(S^k\) the set of clause \(c\) where \(|c|=k\)
\[n_1 = \sum_k \sum_{c_j \in S^k} (1-2^{-k} \geq \sum_k \sum_{c_j \in S^k} (1-2^{-k})\hat{z}_j\]
By lemma 1
\begin{align*}
n_2 &\geq \sum_k \sum_{c_j \in S^k} \beta_k \hat{z}_j \Rightarrow\\
\frac{n_1+n_2}{2} &\geq \sum_k \sum_{c_j \in S^k} \frac{(1-2^{-k})+\beta_k}{2}\hat{z}_j
\end{align*}
Since \( \forall k. (1-2^{-k}) + \beta_k \geq 3/2\) \(\Rightarrow\)
\[\frac{n_1+n_2}{2} \geq \frac{3}{4} \sum_k\sum_{c_j \in S^k} \hat{z}_j = \frac{3}{4} \sum_j \hat{z}_j\]
Expanding graph: number of any \(S\) is large than some constant times \(|S|\), ie \(|\tau(S)| \geq c|S|\). \(\tau(S)= \{w\in V| \exists v\in S,(v,w) \in E\}\).\\
An \((n,d,\alpha,c\) OR-concentrator is a bipartite multigraph \(G(L,R,E)\) with indp. set \(|L|=|R|=n\) st.
\begin{enumerate}
\item \(\forall v\in L. deg(v)\leq d\)
\item \(\forall S \subseteq L\) st. \(|S| \leq \alpha n\), there are \(|\tau(S)\leq R| \geq c|S|\)
\end{enumerate}
\textbf{Thrm 5}: \(\exists n_0\) st. \(\forall n > n_0\) there is an \((n,18,1/3,2)\) OR-concentrator.\\
Proof: Consider bipartite rand graph on vert. in \(L\) and \(R\) st \(v\in L\) choose \(\tau(v)\) by sampling \(d\) vertices indep and unif from \(R\). \(\Rightarrow\) \(\tau(v) \leq d\) since multi edges removed.\\
Let \(\mathcal{E}_s\) denote event that \(\tau(S) < c|S|\) neighbors in \(R\).\\
Bound \(\Pr[\mathcal{E}_s]\) then \(\sum_{s\in S: |S| \leq \alpha n} \Pr[\mathcal{E}_s]\) to upperbound prob that rand. graph fails to be OR-conctrator \((n,d,\alpha,c)\) as specified.\\
Fix \(S\subseteq L\) st \(|S|=s\) and \(T\subseteq R\) st \(|T|=cs\). there are \({n \choose s}\) ways of choosing \(S\), and \({n \choose cs}\) ways of choosing \(T\).\\
\(\Pr[\tau(S) \subseteq T] = (cs/n)^{ds}\), since \(\forall v\in S.\tau(v) \leq d\) \(\Rightarrow \Pr[\mathcal{E}_S] \leq {n \choose s} {n \choose cs} (cs/n)^{ds}\)\\
Use \({n \choose k} \leq (ne/k)^k\)
\[\Pr[\mathcal{E}_s]\leq (ne/s)^s(ne/cs)^{cs}(cs/n)^{ds}  = \left[\right(\frac{s}{n}\left)^{d-c-1} e^{1+c} c^{d-c}\right]^s\]
\(\alpha = 1/3\) and \(s \leq \alpha n\)\\
\begin{align*}
\Pr[\mathcal{E}_s] &\leq \left[\left(\frac{1}{3}\right)^{d-c-1} e^{1+d} c^{d-c}\right]^s\\
                   &\leq \left[\left(\frac{1}{3}\right)^{d} (3e)^{1+d}\right]^s
\end{align*}
Using \(c=2 \wedge d=18\)
\[\Pr[\mathcal{E}] \leq \left[\left(\frac{2}{3}\right)^{18}(3e)^3\right]^s\]
let \(r=(2/3)^{18}(3e)^3\) and note \(r < 1/2\) \(\Rightarrow\)
\[\sum \Pr[\mathcal{E}_s] \leq \sum_{s\geq 1} r^s = \frac{r}{1-r} < 1\]
\section{Parallel Routing}
\textbf{Thrm} Consider any randomized oblivous algorithm for permutation routing on the hypercube with \(N = 2^n\) nodes. If this algorithm uses \(k\) random bits then its expected runtime is \(\Omega(2^{-k} \sqrt{N/n}\).\\
Proof: Since we use \(k\) random bits atleast one of deterministic alg., denote A, that solves oblivous routing is chosen with probabilty atleast \(2^{-k}\). \\
We know there exists an permutation st. A uses \(\Omega(\sqrt{N/n})\) time.\\
So we obtain expected runtime \(\Omega(2^{-k} \sqrt{N/n}\) for this ran. algoirthm.\\
\textbf{Col}: Any randomized algorithm that solves oblivous routing on hypercube, must use \(\Omega(n)\) random bits to achive expected runtime \(O(n)\).\\
\textbf{Thrm}: For every n, there exists a randomized oblivous scheme for permutation routing on a hypercube that uses 3n random bits and runs in expcted time at most 15n.\\
Proof:
%TODO write proof 5.10
\section{Lovasz Local Lemma}

\textbf{LLL}: Let \(G(V,E)\) be a dependency graph for events \(E_1,..,E_n\) in a probability space. Suppose there exists \(x_i \in [0,1]\) for \(1\leq i \leq n\) st.
\[\Pr[E_i] \leq x_i \prod_{(i,j) \in E} (1-x_j)\]
then
\[\Pr[\cap_{i=1}^n \bar{E}_i] \geq \prod_{i=1}^n (1-x_i)\]
\textbf{Corollary} Let \(E_1,..,E_n\) be events from a probability space. with \(\Pr[E_i] \leq p\) for all \(i\). If each event is mutually independent of all other events expept atmost \(d\) st. \(ep(d+1) \leq 1\) then 
\[\Pr[\cap_{i=1}^n \bar{E}_i] > 0\]
Consider k-SAT (each clause has k literals) with m clauses, where each of the n variables occur as at most \(2^{k/50}\) literals.\\
Consider assinging 0,1 to each variable equiprobabily, and let \(E_i\) be the event that clause \(i\) is not satisfied, so \(\Pr[E_i] = 2^{-k}\) which is independent of all but atmost \(k2^{k/50}\) clauses. By above corollary there must be postive probabilty that all clauses are satisfied.
\section{Conditonal Probabilites}
Using set balancing problem: Given a \(n\times n\) matrix A with \(a_{ij} \in \{0,1\}\), find vector \(B\in \{-1,1\}^n\) st \(\min |\!|AB|\!|_{\infty} = \max \{|x_1|,..,|x_n|\}\).\\
Concept: Consider random algorithm to  find a string of lenght n a binary tree of heigh n. The random string is path from root to leaf, where at each level we append the value of the node.\\
For balancing problem conider a good if vector \(v\) of label from the root leading to it satisfies \(|\!|Av|\!| \leq 4\sqrt{n\ln n}\). \\
Consider bad event \(E_i\) if the inner product of the ith row of A with b exeeds \(4\sqrt{n\ln n}\). We know probabilty \(\Pr[E_i] \leq 2/n^2\), so \(\sum_i \Pr[E_i] \leq 2/n\). Let \(\Pr[E_i|a]\) be of bad event conditioned on alg being in node a and let \(\hat{P}[\sum_i \Pr[E_i | a]\) now \(P(a) \leq \hat{P}(a)\) for all a.
\textbf{Thrm}: The algorithm based on method of conditional probailities to determines a vector b st \(|\!| Ab|\!|_{\infty} \leq 4\sqrt{n \ln n}\) in polytime.\\
Proof: Follows from the three properties of \(\hat{P}\).
\begin{enumerate}
\item \(\hat{P}(r) < 1\)
\item For any node a with children b,c
\[\hat{P}(\min (b,c)) \leq \hat{P}(a)\]
\item For any node a, we can compute \(\hat{P}(a)\) in polynomial time.
\end{enumerate}
\end{document}
