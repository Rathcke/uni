\documentclass[a4paper]{article}
\usepackage[utf8]{inputenc}
\usepackage{amsmath}
\usepackage{amssymb}
\usepackage{caption}
\usepackage{mathtools}
\usepackage{amsfonts}
\usepackage{lastpage}
\usepackage{tikz}
\usepackage{float}
\usepackage{textcomp}
\usetikzlibrary{patterns}
\usepackage{pdfpages}
\usepackage{gauss}
\usepackage{fancyvrb}
\usepackage[table]{colortbl}
\usepackage{fancyhdr}
\usepackage{graphicx}
\usepackage[margin=2.5 cm]{geometry}

\definecolor{listinggray}{gray}{0.9}
\usepackage{listings}
\lstset{
	language=,
	literate=
		{æ}{{\ae}}1
		{ø}{{\o}}1
		{å}{{\aa}}1
		{Æ}{{\AE}}1
		{Ø}{{\O}}1
		{Å}{{\AA}}1,
	backgroundcolor=\color{listinggray},
	tabsize=3,
	rulecolor=,
	basicstyle=\scriptsize,
	upquote=true,
	aboveskip={0.2\baselineskip},
	columns=fixed,
	showstringspaces=false,
	extendedchars=true,
	breaklines=true,
	prebreak =\raisebox{0ex}[0ex][0ex]{\ensuremath{\hookleftarrow}},
	frame=single,
	showtabs=false,
	showspaces=false,
	showlines=true,
	showstringspaces=false,
	identifierstyle=\ttfamily,
	keywordstyle=\color[rgb]{0,0,1},
	commentstyle=\color[rgb]{0.133,0.545,0.133},
	stringstyle=\color[rgb]{0.627,0.126,0.941},
  moredelim=**[is][\color{blue}]{@}{@},
}

\lstdefinestyle{base}{
  emptylines=1,
  breaklines=true,
  basicstyle=\ttfamily\color{black},
}

\pagestyle{fancy}
\def\checkmark{\tikz\fill[scale=0.4](0,.35) -- (.25,0) -- (1,.7) -- (.25,.15) -- cycle;}
\newcommand*\circled[1]{\tikz[baseline=(char.base)]{
            \node[shape=circle,draw,inner sep=2pt] (char) {#1};}}
\newcommand*\squared[1]{%
  \tikz[baseline=(R.base)]\node[draw,rectangle,inner sep=0.5pt](R) {#1};\!}
\newcommand{\comment}[1]{%
  \text{\phantom{(#1)}} \tag{#1}}
\def\el{[\![}
\def\er{]\!]}
\def\dpip{|\!|}
\def\MeanN{\frac{1}{N}\sum^N_{n=1}}
\cfoot{Page \thepage\ of \pageref{LastPage}}
\DeclareGraphicsExtensions{.pdf,.png,.jpg}
\author{Nikolaj Dybdahl Rathcke (rfq695) \\ Victor Petren Bach Hansen (grn762)}
\title{Randomized Algorithms \\ Assignment 2 - Resubmission}
\lhead{Randomized Algorithms}
\rhead{Assignment 2 - Resubmission}

\begin{document}
\maketitle

\section*{Problem 3.1}
When want to count the number of empty bins when we throw $n$ balls into $n$ bins, consider a single bin $i$. The probability that the bin is empty is equal to throwing the ball into another bin each of the $n$ times:
\begin{align*}
  \mathbb{P}[\mbox{bin $i$ is empty}] &= \left(\frac{n-1}{n}\right)^n \\
                                      &= \left(1-\frac{1}{n}\right)^n \\
                                      &\approx 1/e \comment{Approaches for $n\rightarrow \infty$}
\end{align*}
Let $Z_i$ be $1$ if bin $i$ is empty and $0$ otherwise. Then we have exactly that:
$$
\mathbb{E}[Z_i]=\mathbb{P}[\mbox{bin $i$ is empty}]\approx 1/e
$$
Since every bin has the same probability of being empty, we can find the expected number of empty bins:
\begin{align*}
  \mathbb{E}\left[\sum_{i=1}^n Z_i\right] &= \sum_{i=1}^n \mathbb{E}[Z_i] \comment{By linearity of expectation} \\
                                          &\approx \sum_{i=1}^n 1/e \\
                                          &=n/e
\end{align*}
Which is what we wanted to show. \\
When we throw $m$ balls into $n$ bins, we can calculate the expected number of empty bins in the same manner. That is, look at the probability of bin $i$ being empty (that we hit another bin $m$ times):
\begin{align*}
  \mathbb{P}[\mbox{bin $i$ is empty}] &= \left(\frac{n-1}{n}\right)^m \\
                                      &= \left(1-\frac{1}{n}\right)^m
\end{align*}
If we want to show an approximation as before, we can approximate the probability above to $\frac{1}{e^{m/n}}$. Then again, we let $Z_i$ be $1$ if bin $i$ is empty and $0$ otherwise and as before we can calculate the expected number of empty bins:
\begin{align*}
  \mathbb{E}\left[\sum_{i=1}^n Z_i\right] &= \sum_{i=1}^n \mathbb{E}[Z_i] \comment{By linearity of expectation} \\
                                          &\approx \sum_{i=1}^n \frac{m1}{e^{m/n}} \\
                                          &=ne^{-m/n}
\end{align*}
This probability hold generally. If we let $r=m/n$, where $m$ are the balls we throw into the $n$ bins. Then the number of empty bins will approximately be $ne^{-r}$.

\end{document}
