\documentclass[12pt]{article}
\usepackage[utf8]{inputenc}
\usepackage{amsmath}
\usepackage{mathtools}
\usepackage{amsfonts}
\usepackage{lastpage}
\usepackage{tikz}
\usepackage{pdfpages}
\usepackage{gauss}
\usepackage{fancyvrb}
\usepackage{fancyhdr}
\usepackage{graphicx}
\pagestyle{fancy}
\fancyfoot[C]{\footnotesize Page \thepage\ of 15}
\DeclareGraphicsExtensions{.pdf,.png,.jpg}
\title{Database and Web Programming}
\author{Nikolaj Dybdahl Rathcke}
\chead{Nikolaj Dybdahl Rathcke - rfq695}

\begin{document}

Binære søgetræer datastruktur, Består af noder, som udover en key har\\ attributter(left, right, parent) og noget satellitdata -  universel pointer til root\\
\\
Binary search tree property\\
\\
Balancerede har højst en højde log(n) fra root til bund simple path - binært kan være lort.\\
\\
Balancerede gør at basic operationer (search, predecessor, successor, minimum, maximum, insert, delete) tager O(log n) - ikke for O(n) tid.\\
\\
Vis procedure for search/access, insert og delete. Make er triviel\\
\\
Gode hvis højden er lille.\\
\\
Rød-sorte træer er balancerede binære træ med en ekstra attribut (farve) 1 bit. \\
\\
Regler for rød-sorte træer: Alt har en farve, roden er sort, leafs er sorte, en node er rød er begge børn sorte, samme antal sorte noder ned.\\
\\
Hvorfor rødsorte træer er gode søgetræer O(log n) med lemma-\\
\\
Lemma 13.1 (rødsort træ har en højde på max 2 lg(n+1)\\
\\
Proof: Induktion at et subtree med node x har $2^bh(x)-1$ (claim)
\\
Basis: h = 0, is a leaf with no nodes.\\
Vis sandt for h $<$ k(højden)\\
\\
Forestil dig nu subtree med 2 børn.\\
\\
Rød-sorte træer med rotering (rotationerne). Basic funktioner fungere på samme måde og opretholder properties ved rotationer og omfarvning op igennem træet.



\end{document}
