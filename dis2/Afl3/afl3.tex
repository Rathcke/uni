\documentclass[12pt]{article}
\usepackage[utf8]{inputenc}
\usepackage{amsmath}
\usepackage{amssymb}
\usepackage{mathtools}
\usepackage{amsfonts}
\usepackage{lastpage}
\usepackage{tikz}
\usepackage{pdfpages}
\usepackage{gauss}
\usepackage{fancyvrb}
\usepackage{fancyhdr}
\usepackage{graphicx}
\usepackage[margin=2.5cm]{geometry}
\pagestyle{fancy}
\fancyfoot[C]{\footnotesize Page \thepage\ of 3}
\DeclareGraphicsExtensions{.pdf,.png,.jpg}
\author{Nikolaj Dybdahl Rathcke}
\chead{Nikolaj Dybdahl Rathcke (rfq695)}

\begin{document}

\section*{Opgave 2}
Idet vi har to cirkler med hver især 6 pladser som kan være på 2 måder, må hver cirkels antal måder børn kan sidde på være
\begin{align*}
N(m,n)&=\frac{1}{m}\sum_{d|m}\varphi(d)n^{m/d} \\
N(6,2)&=\frac{1}{6}(\varphi(1)2^6+\varphi(2)2^3+\varphi(3)2^2+\varphi(6)2^1) \\
&=\frac{1}{6}(2^6+2^3+2\cdot 2^2+2\cdot 2^1) \\
&=\frac{1}{6}84 \\
&=14
\end{align*}
Da vi har to af disse cirkler som hver især kan justeres på denne måde har vi altså $14^2=196$ måder karussellen kan blive fyldt på.

\end{document}
