\documentclass[12pt]{article}
\usepackage[utf8]{inputenc}
\usepackage[fleqn]{amsmath}
\usepackage{amssymb}
\usepackage{mathtools}
\usepackage{amsfonts}
\usepackage{lastpage}
\usepackage{tikz}
\usepackage{pdfpages}
\usepackage{gauss}
\usepackage{fancyvrb}
\usepackage{fancyhdr}
\usepackage{graphicx}
\usepackage[margin=2.5cm]{geometry}
\pagestyle{fancy}
\fancyfoot[C]{\footnotesize Page \thepage\ of 6}
\DeclareGraphicsExtensions{.pdf,.png,.jpg}
\chead{1806922265}
\title{Dis2 Eksamen 2015 \\ 1806922265}
\begin{document}
\maketitle
\section*{Opgave 1}
\subsection*{a}
Grunden til at
\begin{align}
\left\lfloor \frac{k}{2}\right\rfloor\left \lceil \frac{k}{2}\right\rceil =\frac{1}{4}(k^2-[k \mbox{ er ulige}])
\end{align}
er at hvis vi har et lige $k$, så har vi
\begin{align*}
\frac{k}{2}\frac{k}{2}&=\frac{k^2}{4} \\
&=\frac{1}{4}(k^2)
\end{align*}
Men når vi har et ulige $k$, så får vi
\begin{align*}
\frac{k-1}{2}\frac{k+1}{2}&=\frac{(k-1)(k+1)}{4} \\
&=\frac{1}{4}(k^2-1)
\end{align*}
Hvor vi har brugt at det er en kvadratsætning. Disse kan altså sættes sammen til det givne udtryk i (1) da $-1$ kun kommer med når $k$ er ulige.

\subsection*{b}
Vi vil bestemme det lukkede udtryk for
\begin{align*}
\sum_{k=1}^n [k \mbox{ er ulige}]
\end{align*}
Vi ser at summen stiger med $1$ hver gang $k$ er ulige, altså skal vi bare tælle de ulige tal mellem $1$ og $n$, så
\begin{align*}
\sum_{k=1}^n [k \mbox{ er ulige}] &= [1\leq 2k-1\leq n] \\
&=[\frac{1+1}{2}\leq k\leq \frac{n+1}{2}] \\
&=[1\leq k\leq \frac{n+1}{2}]
\end{align*}
Vi skal passe på $n$, da når $n$ er lige, så får vi en rest på $\frac{1}{2}$ i $\frac{n+1}{2}$ og da vi kune arbejder med heltal $k$, så mister vi altså et $k$. Derfor har vi
\begin{align*}
[1\leq k\leq \frac{n+1}{2}] &= \frac{n+1}{2}[k \mbox{ er ulige}]+\frac{n}{2}[k \mbox{ er lige}] \\
&=\left\lceil \frac{n}{2} \right\rceil
\end{align*}
Som er vores lukkede udtryk.

\subsection*{c}
Hvis vi bruger informationen fra de tidligere opgaver har vi at
\begin{align*}
a_n&=\sum_{k=1}^n \frac{1}{4}(k^2-[k \mbox{ er ulige}]) \\
&=\frac{1}{4}\left(\sum_{k=1}^n k^2 - \sum_{k=1}^n [k \mbox{ er ulige}]\right) \\
&=\frac{1}{4}\left(\sum_{k=1}^n k^2 -  \left\lceil \frac{n}{2} \right\rceil\right) \\
\end{align*}
Summen for kvadrattal kender vi, den er nemlig $\frac{n^3}{3}+\frac{n^2}{2}+\frac{n}{6}$, så vi får
\begin{align*}
a_n&=\frac{1}{4}\left (\frac{n^3}{3}+\frac{n^2}{2}+\frac{n}{6} -  \left\lceil \frac{n}{2} \right\rceil\right ) \\
&=\frac{n^3}{12}+\frac{n^2}{8}+\frac{n}{24} -  \frac{1}{4}\left\lceil \frac{n}{2} \right\rceil \\
\end{align*}
Hvilket er vores lukkede udtryk for $a_n$. \\
Vi kan desuden finde et $p(n)$, så $a_n=\lfloor p(n) \rfloor$, hvis vi kigger på hvor meget det sidste led gør for summen. Ved at fjerne ceilings bliver leddet $\frac{1}{8}$ mindre. Det vil altså sige at det totale led er $\frac{1}{8}$ større. Dette korrigeres der så for ved at tage floors af det totale udtryk, så vi får blot
\begin{align*}
p(n)&=\frac{n^3}{12}+\frac{n^2}{8}+\frac{n}{24} -  \frac{n}{8}
\end{align*}
Samt et $q(n)$ som opfylder, at $a_n=\lceil q(n) \rceil$. Det kan vi gøre ved at ændre udtrykket for $a_n$ til at indeholde en floor istedet for en ceiling, så vores $q(n)$ bliver
\begin{align*}
q(n)&=\frac{n^3}{12}+\frac{n^2}{8}+\frac{n}{24} -  \frac{n+1}{8}
\end{align*}
Dette virker da det sidste led bliver $\frac{1}{8}$ større, som gør det totale led bliver $\frac{1}{8}$ mindre, så derfor runder vi op for at korrigere for det.

\section*{Opgave 2}
\subsection*{a}
Vi kan udtrykke antallet af perler der skal til at fylde pladen ved
\begin{align*}
b_n=1+\sum_{k=1}^n 6n
\end{align*}
Der er inkluderet et centrum der ifølge figurene i opgaveteksten indgår.\\
Denne sum kan løses let da vi kender summen af blot $n$ (trekantstallene) til at være $\frac{n(n+1)}{2}$, så
\begin{align*}
b_n &= 1+6\sum_{k=1}^n n \\
&=1+6\left(\frac{n(n+1)}{2}\right) \\
&=1+3n^2+3n
\end{align*}
Som er vores lukkede udtryk.

\subsection*{b}
Vi kan udtrykke måden at fylde en enkelt cirkel med to forskellige slags perler ved
\begin{align}
N(m,n)=\frac{1}{m}\sum_{d \backslash m}\varphi (d)n^{m/d}
\end{align}
hvor $m$ er antal pladser der skal fyldes og $n$ er antallet af forskellige perler. \\
Idet der er i alt $6$ rotationer, hvor af hver "sjette del" af pladen kan fyldes med $2^{((n^2+n)/2)}$ (potensen er kvadrattallene). Da dette er antallet af unikke måder perlerne kan sættes på de $\frac{n^2+n}{2}$ stifter, kan vi se dette som det nye antal "farver" for hver sjettedel. Vi kan derfor bruge (2) til at udtrykke $c_n$. Vi husker på at vi har en perle i midten som kan antage to farver, altså
\begin{align*}
c_n&=2\cdot N(6,2^{((n^2+n)/2)}) \\
&=2\cdot\frac{1}{6}\sum_{d\backslash 6} \varphi(d)\left( 2^{(n^2+n)/2}\right)^{6/d} \\
&=\frac{1}{3}\sum_{d\backslash 6} \varphi(d)\left( 2^{(n^2+n)/2}\right)^{6/d} 
\end{align*}
Denne kan selvfølgelig skrives ud idet vi kender alle divisorer i $6$, men det gør blot udtrykket meget langt.

\subsection*{c}
Der er brugt computer til at finde begge værdier. Vi får, at 
\begin{align*}
c_2&=\frac{1}{3}\sum_{d\backslash 6} \varphi(d)\left( 2^{(2^2+2)/2}\right)^{6/d} \\
&=\frac{1}{3}\sum_{d\backslash 6} \varphi(d) 8^{6/d} \\
&=87600
\end{align*}
som vi ønskede. Desuden fås
\begin{align*}
c_6&=\frac{1}{3}\sum_{d\backslash 6} \varphi(d)\left( 2^{(2^6+6)/2}\right)^{6/d} \\
&=\frac{1}{3}\sum_{d\backslash 6} \varphi(d) 2097152^{6/d} \\
&=14178431955039102645844505448482340864
\end{align*}
Hvor $2097152=2^{21}$ og vores endelige tal er $38$-cifret som vi ønskede at få.

\section*{Opgave 3}
\subsection*{a}
Hvis vi sætter klodsen der rager en udover lig længde $2$, vil vi have den samme figur (på hovedet), men med en mindre i længde, altså $u_{n-1}$.\\
Hvis vi sætter klodsen til at være af længden $3$, tvinger det den næste klods til også at være af længden $3$, hvorved vi har samme figur, men med tre mindre i længden, altså $u_{n-3}$. \\
Vi ser at ved den første totale længde $n$ hvor det er muligt at bygge en mur med disse klodser er ved $n=2$. Altså må vi bruge dette som basistilfælde. Ved at sætte dem sammen får vi altså
\begin{align*}
u_n=u_{n-1}+u_{n-3}+[n=2]
\end{align*}
som vi ville nå frem til.\\
Herefter har vi at
\begin{align*}
U(z)&=\sum_n u_nz^n \\
&=\sum_n u_{n-1}z^n+\sum_n u_{n-3}z^n+\sum_n [n=2]z^n \\
&=\sum_n u_{n}z^{n+1}+\sum_n u_{n}z^{n+3}+z^2 \\
&=zU(z)+z^3U(z)+z^2
\end{align*}
Derved får vi, ved isolering, at
\begin{align*}
U(z)&=\frac{z^2}{1-z-z^3}
\end{align*}

\subsection*{b}
Vi ser at $a_n$ kan gives ved
\begin{align*}
a_n=[n=2]+[n=3]+2u_{n-3}
\end{align*}
Hvor de første to led er basistilfældende. Det sidste led er ved at overveje hvis vi lægger en klods med længde $2$ tvinger vi klodsen ovenpå til at være af længden $3$. Derved får vi en figur der er givet ved $u_{n-3}$. Hvis vi lagde en med længde $2$ ville vi tvinge en af længde $3$ ovenpå. Derfor får vi 2 af disse led.\\
Nu ser vi det sidste led jo kan skrives som
\begin{align*}
2\sum_n u_{n-3}z^n&=2\sum_n u_{n}z^{n+3} \\
&=2z^3\sum_n u_{n}z^n
\end{align*}
Nu kan vi genkende summen i udtrykket til at være $U(z)$, som vi kender, så vi skriver $a_n$ som
\begin{align*}
a_n&=[n=2]+[n+3]+2z^3U(z) \\
&=z^2+z^3+2z^3\frac{z^2}{1-z-z^3} \\
&=z^2+z^3+\frac{2z^5}{1-z-z^3} 
\end{align*}
Som vi ville vise.

\section*{Opgave 4}
\subsection*{a}
Hvis vi ser på ligningen (*), er det klart at der for $n=1$ kun er en mulighed for hvad $d_0$ kan være, nemlig $0$. Hvis man dernæst beregner $n=2$ kan vi se at vi får to ligninger indeholdende et $d_k$. Vi kender det ene, $d_1$, og derfor er der kun en ligning der skal løses. På denne måde kan vi blive ved, hvor vi hele tiden får en ukendt mere i ligningen og derfor er rækken defineret da vi kun har et valg hver gang.\\
Dette kan muligvis lettere ses når vi beregner de første fem værdier for $d_n$ ved at løse (*)
\begin{align*}
d_1&:\ [2\backslash 1]=\frac{1!}{1!\cdot 0!}d_1\ \Leftrightarrow\ 0=d_1\  \Leftrightarrow\ d_1 = 0 \\
d_2&:\ [2\backslash 2]=\frac{2!}{1!\cdot 1!}d_1 + \frac{2!}{2!\cdot 0!}d_2 \ \Leftrightarrow\ 1=2\cdot 0+d_2\  \Leftrightarrow\ d_2 = 1 \\\\
d_3&:\ [2\backslash 3]=\frac{3!}{1!\cdot 2!}d_1 +\frac{3!}{2!\cdot 1!}d_2 + \frac{3!}{3!\cdot 0!}d_3 \ \Leftrightarrow\ 0=3\cdot 0+3\cdot 1+d_3\  \Leftrightarrow\ d_3 = -3 \\
d_4&:\ [2\backslash 4]=\frac{4!}{1!\cdot 3!}d_1 +\frac{4!}{2!\cdot 2!}d_2 + \frac{4!}{3!\cdot 1!}d_3 +\frac{4!}{4!\cdot 0!}d_4\ \Leftrightarrow\ 1=4\cdot 0+6\cdot 1+4\cdot -3+ d_4\\
&\ \ \ \ \ \ \ \ \ \ \ \ \ \Leftrightarrow\ d_4 = 7 \\
d_5&:\ [2\backslash 5]=\frac{5!}{1!\cdot 4!}d_1 +\frac{5!}{2!\cdot 3!}d_2 + \frac{5!}{3!\cdot 2!}d_3 +\frac{5!}{4!\cdot 1!}d_4+\frac{5!}{5!\cdot 0!}d_5\ \\
&\ \ \ \ \ \ \ \ \ \ \ \ \ \ \Leftrightarrow\ 0=5\cdot 0+10\cdot 1+10\cdot -3+ 5\cdot 7 + d_5\ \Leftrightarrow\ d_5=-15
\end{align*}
Altså ved at løse for den eneste ukendte hver gang kan vi finde værdierne af $d_n$.

\subsection*{c}
Hvis vi kigger på ligningen fra (b) ser vi at sum leddene kun antager værdier når $k$ er lige, altså giver det kun mening at kigge på de tilfælde hvor $k$ er lige. \\
Vi kan altså omskrive ligningen fra (b) til
\begin{align*}
d_n &= (-1)^n\sum_{k=1}^n \binom {n}{2k}
\end{align*}
Vi kan beholde $n$ som vores øvre grænse, da hvis $2k$ overstiger $n$ giver binomialkoefficienten bare $0$.\\
Denne sum kender vi (når $k$ går fra $0$ til $n$), den er netop $2^{n-1}$. Vi husker også på at $k$ starter fra $1$. Altså trækker vi leddet $\binom n 0$ fra, så
\begin{align*}
d_n &=(-1)^n\left(2^{n-1}-\binom n 0 \right) \\
&=(-1)^n(2^{n-1}-1)
\end{align*}
Som er vores endelige lukkede udtryk.\\

\end{document}
