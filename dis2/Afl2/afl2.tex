\documentclass[12pt]{article}
\usepackage[utf8]{inputenc}
\usepackage{amsmath}
\usepackage{amssymb}
\usepackage{mathtools}
\usepackage{amsfonts}
\usepackage{lastpage}
\usepackage{tikz}
\usepackage{pdfpages}
\usepackage{gauss}
\usepackage{fancyvrb}
\usepackage{fancyhdr}
\usepackage{graphicx}
\usepackage[margin=2.5cm]{geometry}
\pagestyle{fancy}
\fancyfoot[C]{\footnotesize Page \thepage\ of 3}
\DeclareGraphicsExtensions{.pdf,.png,.jpg}
\author{Nikolaj Dybdahl Rathcke}
\chead{Nikolaj Dybdahl Rathcke (rfq695)}

\begin{document}

\section*{Opgave 1}

Hvis vi sætter $n=a^2$ og bruger at $\lceil x\rceil=\sum_j [1\leq j<x+1]$
DER TONSES

\begin{align*}
\sum_{k=0}^{a^2}\lceil\sqrt{2k}\rceil &= \sum_{j,k}\left[1\leq j< \sqrt{2k}+1\right]\left[0\leq k\leq a^2\right] \\
&=\sum_{j,k}\left[0\leq \frac{(j-1)^2}{2}< k\right]\left[0\leq k\leq a^2\right] \\
&=\sum_{0\leq j\leq a}\sum_k \left[\frac{(j-1)^2}{2}< k\leq a^2\right] \\
&=\sum_{0\leq j\leq a}\sum_k \left[(j-1)^2< 2k\leq 2a^2\right] \\
&=\sum_{0\leq j\leq a} \frac{2a^2-(j-1)^2}{2} \\ 
&=\sum_{0\leq j\leq a} \frac{2a^2-(j^2-2j+1)}{2} \\
&=-\frac{1}{2}+\sum_{0\leq j\leq a} \frac{2a^2-j^2+2j}{2} \\
&=-\frac{1}{2}+\sum_{0\leq j\leq a} a^2-\frac{j^2}{2}+j \\
&=-\frac{1}{2}+\sum_{0\leq j\leq \lfloor \sqrt{n}\rfloor} n-\frac{j^2}{2}+j \\
&=\frac{5a^3}{6}+\frac{5a^2}{4}+\frac{5a}{12}-\frac{1}{2}
\end{align*}

40/6
20/4
10/12
-1/2

80+60+10-6

144/12


10+15+5-6
24/12

\end{document}
&=\sum_{1\leq j\leq a} \frac{a^2-j^2}{2} \\
&=\frac{1}{2}\sum_{1\leq j\leq a} a^2-j^2 \\	
&=\frac{1}{2}(a^3-\frac{1}{3}a(a+\frac{1}{2})(a+1)) \\
&=\frac{a^3}{3}-\frac{a^2}{4}-\frac{a}{12}