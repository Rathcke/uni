\documentclass[12pt]{article}
\usepackage[utf8]{inputenc}
\usepackage{amsmath}
\usepackage{amssymb}
\usepackage{mathtools}
\usepackage{amsfonts}
\usepackage{lastpage}
\usepackage{tikz}
\usepackage{pdfpages}
\usepackage{gauss}
\usepackage{fancyvrb}
\usepackage{fancyhdr}
\usepackage{graphicx}
\usepackage[margin=3.5cm]{geometry}
\pagestyle{fancy}
\fancyfoot[C]{\footnotesize Page \thepage\ of 3}
\DeclareGraphicsExtensions{.pdf,.png,.jpg}
\author{Nikolaj Dybdahl Rathcke}
\chead{Nikolaj Dybdahl Rathcke (rfq695)}

\begin{document}

\section*{Miller-Rabin}
\subsection*{Algoritmen}
Miller-Rabin testen afgør om et givet tal, $n$, er et sammensat tal eller om det er "nok et primtal". Dette skyldes at det er en probabilistisk algoritme, altså at den afhænger af tilfældighed. \\
Et kald til Miller-Rabin algoritmen kræver udover tallet $n$ også en parameter $k$ som er antallet af gange algoritmen bliver gentaget. For hver ekstra gang algoritmen bliver kørt hvor der bliver returneret at $n$ nok er primtal, falder sandsynligheden for at $n$ ren faktiskt er et sammensat tal. Altså ses $k$ som en parameter for korrekthed. \\
Algoritmen kan deles i fire trin for $n\geq 5$.
\begin{enumerate}
\item Find de unikke tal $r$ og $s$, så $n-1=2^r\cdot m$ og $m$ er ulige.
\item Vælg et tilfældigt heltal $a$, hvor $1<a<n$.
\item Sæt $b=a^m$ (mod $n$). Hvis $b\equiv \pm 1$ (mod $n$), så er $n$ nok et primtal.
\item Hvis $b^{2^s}\equiv -1$ (mod $n$) for et $s$ hvor $1\leq s\leq r-1$, så er $n$ nok et primtal. Hvis ikke, så er $n$ et sammensat tal.
\end{enumerate}
En implementering af dette kan ses

\subsection*{Køretid}




\end{document}
