\section{Konklusion}
Man kan konkludere, at der er forskellige tilgange til at finde ud af, om et tal er et primtal. 
De probabilistiske prøver ved brug af gentagne forsøg at forøge sandsynligheden for, at et tal er 
et primtal. De to probabilistiske teste, der er blevet gennemgået i denne rapport, har haft to forskellige
tilgange til det. Miller-Rabin kunne finde ud af, om et tal var sammensat, og hvis det ved gentagne forsøg ikke
var vist at dette var tilfældet, antog man det for at være et primtal. Proth lavede fire tjek: Et der tjekkede for om tallet 
var et primtal, to der tjekkede om tallet var sammensat, og det sidste der bestemte, hvorvidt det forblev ukendt om tallet var et primtal eller ej. 
En helt anden tilgang til primtalstest så man igennem Pocklington testen, der udnyttede faktorisering af $N-1$.
De har alle tre hver især deres fordele. For Miller-Rabin giver man for tidsforbrugets skyld afkald på at kunne garantere,
at et tal er et primtal. Ved Proth kunne man ikke altid afgøre, om et tal var et primtal  eller ej, men til genæld var der god ydeevne. 
Afslutningsvis kunne Pocklington altid afgøre, om et tal var primisk, men til gengæld havde den langsom køretid.
