\documentclass[11pt,a4paper]{article}
\usepackage[utf8]{inputenc}
\usepackage[T1]{fontenc}
\usepackage[danish]{babel}
\usepackage{pdfpages}
\usepackage{stmaryrd}
\usepackage{listings}
\usepackage{amsmath,amssymb,amsfonts}
\usepackage{mathtools}
\usepackage{amsthm}
\usepackage{listings}

\DeclarePairedDelimiter\ceil{\lceil}{\rceil}
\DeclarePairedDelimiter\floor{\lfloor}{\rfloor}

\newcommand\defeq{\stackrel{\mbox{\normalfont\tiny def}}{=}}
\newcommand{\HRule}{\rule{\linewidth}{0.5mm}}

\let\oldemptyset\emptyset
\let\emptyset\varnothing

\newtheorem{lemma}{Lemma}
\newtheorem{theorem}{Teorem}

\begin{document}
\begin{titlepage}
\begin{center}

\textsc{\Large Diskret matematik 2\\[0.3cm](2014-2015)}\\[0.5cm]

\HRule \\[0.4cm]
{ \LARGE \bfseries 8. Primtalstest, moduloregning i store beregninger}\\[0.4cm]

\HRule \\[1.2cm]

% Author and supervisor
\begin{minipage}{0.6\textwidth}
  \begin{flushleft}
    \textbf{Gruppemedlemmer:}\\
    Christoffer Nave \textsc{Øhrstrøm}\\
    Mads \textsc{Filbert}\\
    Nikolaj Dybdahl \textsc{Ratchke}\\
    Victor Petren Bach \textsc{Hansen}
  \end{flushleft}
\end{minipage}

\end{center}

\begin{center}

\vfill

{\large 9. januar 2015}

\end{center}
\end{titlepage}

\tableofcontents
\thispagestyle{empty}
\newpage
\setcounter{page}{1}
\section{Introduktion}
Denne rapport har som formål at give læseren et vidensgrundlag om primtal, og om måden man tester 
om et tal er et primtal.
Vi vil gennemgå en række algoritmer, der tester for om et tal er et primtal. Vi vil starte med at 
gennemgå primtalsfaktorisering, da mange primtalstests baserer sig på faktoriseringen
af $N-1$ og $N+1$. Herefter vil der blive redegjort for Pocklingtons test, som er en deterministisk primtalstest.
Efterfølgende vil der blive gjort rede for de to probabilistiske algoritmer: Proth og Miller-Rabin.
Afslutningsvis vil vi lave sammenligning af implementeringerne af de tre algoritmer. 

\section{N-1 teste}
Pocklingtons metode til at teste primtal er baseret på, at $n-1$ ofte kan faktoriseres trivielt. Primtal tests der tester for $n-1$ og $n+1$, hvor 
disse er trivielt faktoriseret er meget almindelige, da de gør det let at teste, om et tal er et primtal. Derfor består listen 
af de størst kendte primtal også primært af tal, for hvilke dette er gældende. I dette projekt vil der bliver fokuseret på $n-1$ tests.
\subsection{Faktorisering}
I næste sektion beskriver vi Pocklington testen, som bygger på primtalsfaktorisering, hvor
der arbejdes ud fra, at man kan afgøre om et tal er et primtal ud fra bl.a. primtalsfaktorerne
for et andet tal. Vi har implementeret algoritmen ``trial division'', da den er simpel
at implementere og således er mulig at implementere inden for projektets rammer. Trial division
fungerer ved at finde alle primtalsfaktorer af et tal $n$ op til $\floor{\sqrt{n}}$ og dividere
tallet med den pågældende faktor, når en sådan faktor findes. Trial division starter da forfra
med at finde primtalsfaktorer for det nye tal.
Dette er dog en meget langsom algoritme til primtalsfaktorisering, og vi vil derfor
også beskrive matematikken bag algoritmen ``quadratic sieve'', som er væsentlig mere effektiv end
``trial division''. Quadratic sieve er i midlertidig ikke blevet implementeret, da
vi mente, at dette ville tage for lang tid.
\subsubsection{Quadratic sieve}
Quadratic sieve bygger på, at vi kan udtrykke $n$ som differensen mellem to kvadrattal. Ved
at bygge videre på denne ide, så søger algoritmen at finde to tal $x, y$, der opfylder
\begin{align}
  x & \not \equiv \pm \; y \pmod{n} \label{eqn:qs-condition1} \\
  x^2 & \equiv y^2 \pmod{n} \label{eqn:qs-condition2}
\end{align}
Ud fra disse to betingelser er det klart, at $(x - y)(x + y) \equiv 0 \pmod{n}$, hvilket vil
sige, at $(x - y)(x + y)$ er en divisor i $n$. Den første af betingelserne udelukker dog, at
$(x - y)$ og $(x + y)$ alene er divisorer i $n$. Således må $\text{gcd}(x - y, n)$ og
$\text{gcd}(x + y, n)$ være ikke trivielle divisorer i $n$. (gcd er funktionen til at finde
den største fælles divisor for to tal.)\\
Vi vil nu vise, hvordan quadratic sieve fungerer ved at finde primtalsfaktoriseringen for
$n = 1649$ ved at anvende ovenstående beskrivelse. (\ref{eqn:qs-condition2}) giver, at vi skal finde
to kvadrattal, som er kongruente med $n$. $x$ bestemmes ved først at lave en iterativ søgning,
hvor vi starter med $x = \ceil{\sqrt{n}} = 41$. Vi husker, at algoritmen bygger på, at
$n = x^2 - y^2$, hvorfor vi kan finde $y^2 = x^2 - n$.
\begin{align*}
  41^2 - n &= 32 \\
  42^2 - n &= 115 \\
  43^2 - n &= 200
\end{align*}
Vi har endnu ikke fundet to kvadrattal, der opfylder (\ref{eqn:qs-condition1}) og (\ref{eqn:qs-condition2}),
men alligevel vælger vi at stoppe søgningen ved $x = 43$. Vi bemærker nemlig at produktet
$32 \cdot 200 = 6400 = 80^2$ er et kvadrattal, og vi får derfor følgende identitet:
\begin{align*}
  41^2 \cdot 43^2 \equiv 80^2 \pmod{n} \iff (41 \cdot 43)^2 \equiv 80^2 \pmod{n}
\end{align*}
Vi har altså $x = 41 \cdot 43 = 1763$ og $y = 80$. Vi anvender nu gcd til at finde de
to primtalsfaktorer for $n = 1649$.\\
\begin{align*}
  \text{gcd}(x - y, n) &= \text{gcd}(1763 - 80, 1649) = \text{gcd}(1683, 1649) = 17 \\
  \text{gcd}(x + y, n) &= \text{gcd}(1763 + 80, 1649) = \text{gcd}(1843, 1649) = 97
\end{align*}
En simpel undersøgelse kan vise, at $17$ og $97$ er primtal, hvorfor vi nu har fundet
primtalsfaktoriseringen for $n = 1649 = 17 \cdot 97$.\\
Eksemplet har vist, hvordan quadratic sieve fungerer, men vi mangler at forklare,
hvordan den iterative process vælger $y_1,\cdots,y_t$, så $x_i^2 - n = y_i^2$
for $1 \le i \le t$ og produktet af alle $y_i$'er er et kvadrattal.
Dette kan gøres ved at se på ``smooth numbers''. Vi siger, at et tal er $B$-smooth,
hvis alle dets primtalsfaktorer er mindre eller lig $B$. Vi lader $\pi(B)$ betegne
antallet af primtal i intervallet $[1 ; B]$.
\begin{lemma}
(Uden bevis.) Hvis $M = m_1, m_2, \dots, m_k$ er $B$-smooth heltal og $k > \pi(B)$, så findes der
en ikke tom delsekvens af $M$, hvor produktet af tallene i delsekvensen er et kvadrattal.
\end{lemma}
Vi kan altså finde det ønskede kvadrattal ved at finde mere end $\pi(B)$ $B$-smooth
tal. Målet vil vu være at finde delsekvensen, der giver kvadrattallet. I den forbindelse
bemærker vi, at hvert $m_i$ kan repræsenteres af en vektor $v_i$ med $\pi(B)$ elementer,
hvor det $j$'te element i $v_i$ er eksponenten for det $j$'te primtal. Det ses da trivielt,
at produktet af en delsekvens er et kvadrattal, hvis alle elementerne i vektoren, der
er summen af vektorerne i delsekvensen, er lige. Vi kan således konstruere en matrix, hvor
den $i$'te række er $v_i$, og ``mod 2'' er anvendt på hvert element. Rækkerne,
der opfylder det givne kriterie, kan da findes ved at anvende gaussisk elimination.
Vi har altså nu beskrevet teorien bag quadratic sieve, hvor det dog er blev udeladt,
hvordan man effektivt kan finde $B$-smooth tal.

\subsection{Pocklington}
Pocklingtons metode til at teste primtal er baseret på primtalsfaktorisering af $n-1$. 
Testen er deterministisk. Dvs., at testens resultatet med sikkerhed er korrekt.
\begin{theorem}
\label{theorem:pocklington2}
Lad $N > 1$. Hvis der for hvert primtal $q$ i $N-1$'s primtals faktorisering findes et heltal $a$, for hvilket det gælder: 
\begin{enumerate}
\item $a^{N-1}\equiv 1 \pmod{N}$
\item $a^{(N-1)/q} \not \equiv 1 \pmod{N}$
\end{enumerate}
så vil $N$ være et primtal.
\end{theorem}
For at få en forståelse for teoremet vil vi først fremføre en intuitiv forklaring af teoremet. Hvis
$N$ er et primtal, vil alle tal under $N$ være indbyrdes primiske med $N$ - altså eulers $\varphi$ funktion (antallet af tal mindre end $N$ der er indbyrdes primiske med $N$) være lig $N-1$.\\
Derved kan man tolke den første betingelse igennem gruppeteori, hvor man kigger på gruppen $(Z/ZN)*$, som er restklassen
for $N$. 
Den første betingelse siger således blot, at ordenen af $a$ i $(Z/ZN)*$ er $N-1$. Ordenen er det antal gange
man i en gruppe skal anvende gruppens operator på et element for at opnå identitetselementet. Hvis ordenen er $N-1$, skal $a$ altså i
dette tilfælde multipliceres med sig selv $N-1$ gange for at få $1$, som er identitetselementet for en multiplikativ gruppe.
Man kan forestille sig det lidt som om, man går en tur rundt i gruppen $(Z/ZN)*$, hvor man på vejen kommer forbi $N-1$ elementer
for at komme frem til $1$.
Man kan dog ikke være helt sikker på, at dette medfører, at der er $N-1$ elementer, der er indbyrdes primiske med $N$. Man skal først 
sikre sig, at der ikke er nogle elementer i gruppen, man ikke når, og det er netop hvad den anden betingelse sørger for. 
Hvis man igen tænker på det som en tur rundt i gruppen, så sikrer den altså, at man ikke kommer forbi det samme element i gruppen
to gange. Dette var den intuitive forståelse af teoremet. Et mere formelt bevis kan formuleres på følgende måde.
\begin{proof}[Bevis for teorem \ref{theorem:pocklington2}]
Som tidligere nævnt skal vi for at vise, at $N$ er primisk, vise, at $\varphi(N)=N-1$ eller, at $N-1 \backslash \varphi(N)$,
hvor $\varphi(N)$, som tidligere nævnt, er eulers totientfunktion. Hvis vi antager, at $N-1$ ikke deler $\varphi(N)$, så 
vil der være et primtal $q$ og en eksponent $r>0$, så $q^r$ deler $N-1$, men ikke $\varphi(N)$. For dette $q$ vil der være et $a$ for hvilket
de to betingelser gælder. Lad os nu sige, at $m$ er ordenen af $a$ modulo $N$. Da vil den første betingelse medføre, at $m \backslash N-1$, og den
anden betingelse vil medføre, at $m$ ikke deler $(N-1)/q$. Derved fås $q^r \backslash m$ og $m \backslash \varphi(N)$, men det må betyde, at $q^r \backslash \varphi(n)$, hvilket er en modstrid. Dette beviser, at teorem \ref{theorem:pocklington2} er korrekt.
\end{proof}
Vi mangler dog et sidste teorem for at komme frem til pocklington testen. Det sidste teorem er Pocklingtons teorem:
\begin{theorem}
\label{theorem:pocklingtons-theorem}
(Pocklingtons teorem). Lad $N-1$ kunne skrives på formen $q^kR$, hvor $q$ er et primtal, der ikke deler $R$.
$q$ er altså et af primtallene fra $N-1$'s primtalsfaktorisering. Hvis der findes et heltal $a$,
der opfylder de to betingelser fra Lucas testen, så kan alle primtalsfaktorer af $N$ skrives på formen $q^kr+1$.
\end{theorem}
\begin{proof}
Lad $p$ være en hvilken som helst af primtalsfaktorerne for $N$, og $m$ være ordenen af 
$a$ modulo $p$. Ligesom i det tidligere bevis for teorem \ref{theorem:pocklington2} må $q^k$ dele $m$,
og da $m$ er ordenen af $a$ modulo $p$, så deler $m$ $p-1$, og deraf følger teoremet.
\end{proof}
Hvis man kombinerer Pocklingtons teorem med teorem \ref{theorem:pocklington2}, får man Pocklington testen som beskrevet ud fra teorem \ref{theorem:pocklington1}.
\begin{theorem}
\label{theorem:pocklington1}
Faktoriser $N-1$ så $N-1=AB$ hvor $A$ og $B$ er indbyrdes primiske, $A>B$ og hvor vi kender primtals faktoriseringen
af $A$. Da gælder det:
Hvis der for hvert primtal $p$ i $A$'s faktorisering findes et heltal $a$ der har egenskaberne
\begin{enumerate}
\item $a^{n-1}\equiv 1 \pmod{N}$
\item $\text{gcd}(a^{(N-1)/p}-1, N) = 1$
\end{enumerate}
så vil $n$ være et primtal.
\end{theorem}
Observer at $A>B$ betingelsen medfører, at $A>\sqrt{N}$, og netop dette gør teorem \ref{theorem:pocklington1}
mere effektiv end testen baseret på \ref{theorem:pocklington2}. 

\documentclass[11pt,a4paper]{article}

\usepackage[utf8]{inputenc}
\usepackage[T1]		 {fontenc}
\usepackage{pdfpages}
\usepackage{listings}
\usepackage{amsmath,amssymb,amsfonts}
\usepackage{mathtools}
\usepackage{amsthm}
\usepackage{listings}

\DeclarePairedDelimiter\ceil{\lceil}{\rceil}
\DeclarePairedDelimiter\floor{\lfloor}{\rfloor}

\newcommand\defeq{\stackrel{\mbox{\normalfont\tiny def}}{=}}
\newcommand{\HRule}{\rule{\linewidth}{0.5mm}}

\let\oldemptyset\emptyset
\let\emptyset\varnothing

\newtheorem{smoothNumbers}{Lemma}
\newtheorem{prothTeorem}{Teorem}

\begin{document}
\subsection*{Proth tal}
Et Proth tal er et positivt heltal på formen $N=h*2^k+1$ hvor det er krævet at $k$ er ulige og $k>2^h$ gælder. \\
\begin{prothTeorem}
\label{thm:1}ved vi at
(Proths Teorem). Lad $N$ være et Proth tal som defineret før, hvis der for et heltal $a$ gælder
$$a^{(n-1)/2} \equiv -1\:(mod\:N)$$
hvor det er antaget at $(\frac{a}{N}) = -1$, så er $N$ et primtal.
\end{prothTeorem}
\begin{proof}
Vi starter med at antage at $N$ er et primtal.\\
Ved brug af Euler's kriterium kan vi omkskrive $(\frac{a}{N}) = -1$  til
$$a^{(n-1)/2} \equiv -1\:(mod\:N)$$
Under antagelse af at kongruensen i teorem \ref{thm:1} holder, så er $N-1=h*2^k$ og $\gcd(k,2^n)=1$, derfor
$$a^{N-1}=\left(a^{(N-1)/2} \right)^2 \equiv (-1)^2 \equiv 1\: (mod\: N)$$
Siden $N$ er ulige og deler $a^{(N-1)/2}$ må der nødvendigvis også gælde $\gcd(a^{(N-1)/2}-1,N)=1$. Fra Pocklingtons teorem [reference her], ved vi at enhver primtalsfaktor $p$ af $N$ er på formen
$$p=r2^k+1>2^k$$
Men 
$$N=h*2^k+1$$
Derfor
$$\sqrt{N}<2^n<p$$
Hvilket betyder at $N$ er et primtal.
\end{proof}
\subsection*{Proth test}
Proth testen er en simpel probabilistisk primtalstest, der fungerer ved at udvælge et vilkårligt heltal $a$, hvorom der ikke gælder $a \not \equiv 0\:(mod\:N)$. Her udregnes 
$$b\equiv a^{(N-1)/2}\:(mod\:N)$$
hvilket giver 4 tilfælde
\begin{enumerate}
\item Hvis $b\equiv -1\: (mod\:N)$, så er $N$ et primtal (udfra Teorem \ref{thm:1})
\item Hvis $b\not\equiv \pm 1\: (mod\:N)$ og $b^2\equiv 1\: (mod\:N)$, så er $N$ et sammensat tal, idet $\gcd(b \pm 1,N)$ er ikke-trivielle faktorer af $N$.
\item Hvis $b^2\not\equiv 1\: (mod\:N)$, så kan man udfra Fermats lille teorem slutte at $N$ er et sammensat tal
\item Hvis $b\equiv 1\: (mod\:N)$ forbliver det ukendt om hvorvidt $N$ er primtal
\end{enumerate}
Denne procedure gentages indtil at det er blevet bestemt om $N$ er et primtal eller ej. I tilfældet af at $N$ er et primtal, har testen en sandsynlighed på $\frac{1}{2}$ for at returnere at $N$ er et primtal i en enkelt iteration.
\end{document}
\documentclass[12pt]{article}
\usepackage[utf8]{inputenc}
\usepackage{amsmath}
\usepackage{amssymb}
\usepackage{mathtools}
\usepackage{amsfonts}
\usepackage{lastpage}
\usepackage{tikz}
\usepackage{pdfpages}
\usepackage{gauss}
\usepackage{fancyvrb}
\usepackage{fancyhdr}
\usepackage{graphicx}
\usepackage[margin=3.5cm]{geometry}
\pagestyle{fancy}
\fancyfoot[C]{\footnotesize Page \thepage\ of 3}
\DeclareGraphicsExtensions{.pdf,.png,.jpg}
\author{Nikolaj Dybdahl Rathcke}
\chead{Nikolaj Dybdahl Rathcke (rfq695)}

\begin{document}


\section{Miller-Rabin}
Miller-Rabin testen afhænger af nogle ligninger, som er sande for primtal. Hvis vi har et primtal $p>2$, så ved vi, at $p-1$ er lige, og vi kan derfor skrive det unikt som $2^r \cdot m$, hvor $m$ er ulige. For ethvert heltal $a$ i gruppen $\mathbb{Z}/p\mathbb{Z}$ gælder der så, at
\begin{align}
  a^m \equiv 1 \pmod{p}
  \label{eqn:mr1}
\end{align}
eller
\begin{align}
  a^{2^s \cdot m} \equiv -1 \pmod{p}
  \label{eqn:mr2}
\end{align}
for $0 \leq s \leq r-1$. Dette vil vi også gerne vise er sandt. Til det bruger vi en simpel omskrivning af
Fermats lille teorem:

Lad $p$ være et primtal. Hvis $p$ ikke er en divisor i $a$, da er
\begin{align*}
  a^{p-1} \equiv 1 \pmod{p}
\end{align*}
\label{fermats-little-theorem}

Bemærk at $p$ ikke er en divisor i $a$, da $a$ er et heltal i gruppen, $\mathbb{Z}/p\mathbb{Z}$, hvorfor $0 < a < p$.\\
Til beviset vil vi også gøre bruge af følgende lemma:

Hvis vi har et kvadrat $x^2$ kongruent med $1$ modulo $p$, så gælder det, at
\begin{align*}
  &x^2 \equiv 1 \pmod{p} \iff (x-1)(x+1) \equiv 0 \pmod{p}
\end{align*}
Med andre ord er $p$ en divisor i $(x-1)(x+1)$.
\label{lemma:mr1}

Dette betyder så, at
\begin{align*}
  x \equiv \pm 1 \pmod{p}
\end{align*}
Ved at kombinere teorem \ref{fermats-little-theorem} og lemma \ref{lemma:mr1} kan vi altså blive ved med at tage kvadratroden af $a^{p-1}$.
Hvis vi får $-1$, så holder (\ref{eqn:mr2}). Hvis vi aldrig får $-1$, så er det fordi, vi ikke har flere potenser af $2$, og så vil (\ref{eqn:mr1}) holde.
Miller-Rabin testen afhænger dog af kontrapositionen - nemlig at
\begin{align*}
  a^m \not \equiv 1 \pmod{p}
\end{align*}
eller
\begin{align*}
  a^{2^s \cdot m} \not \equiv -1 \pmod{p}
\end{align*}
for $0\leq s\leq r-1$. 
\subsection{Algoritmen}
Miller-Rabin testen afgør, om et tal $n$ er et sammensat tal eller, om det sandsynligvis er et primtal.
Dette skyldes, at det er en probabilistisk algoritme - altså at den afhænger af tilfældighed.
Miller-Rabin algoritmens argumenter kræver, udover tallet $n$, også en parameter $k$, der er antallet af gange, algoritmen bliver gentaget.
For hver ekstra gang algoritmen bliver kørt, hvor der bliver returneret at $n$ nok er primtal, falder sandsynligheden for, at $n$ ren faktiskt er et sammensat tal.
Altså ses $k$ som en parameter for korrekthed. Tallene $a$ kaldes for ``vidner'', hvis de angiver, at $n$ er et sammensat tal.
Algoritmen kan deles i fire trin for $n \geq 5$.
\begin{enumerate}
  \item Find de unikke tal $r$ og $m$ så $n-1=2^r\cdot m$,, og $m$ er ulige.
  \item Vælg et tilfældigt heltal $a$ hvor $1 < a < n$.
  \item Sæt $b = a^m \mod n$. Hvis $b \equiv \pm 1 \pmod{n}$, så er $n$ nok et primtal.
  \item Hvis $b^{2^s}\equiv -1 \pmod{n}$ for et $s$, hvor $1 \leq s \leq r-1$, så er $n$ nok et primtal. Hvis ikke så er $n$ et sammensat tal.
\end{enumerate}
Algoritmen er blevet implementeret i Ruby, og filen \texttt{millerrabin.rb} (implementeringen) kan ses i bilag \ref{app:miller-rabin}.
Algoritmen er blevet implementeret med henblik på at kunne understøtte flere kørsler for at opnå flere baser $a$. I bilag \ref{app:mr-test} 
ses outputtet, når den køres på testsættet ``numbers'', hvor de første 10 tal er primtal, og de sidste 10 er sammensatte tal.
\subsection{Præcision og køretid}
Det kan vises, at der mindst er $3/4$ vidner $a$ for et ulige sammensat heltal. Altså vil der for $k$ kørsler på $n$, der siger $n$ er et primtal,
højest være en sandsynlighed på $4^{-k}$, for at $n$ faktiskt er et sammensat. En effektiv implementering af modular eksponentiering kører
i $\mathcal{O} (\log n)$ tid. Dette vil give os en samlet køretid på $\mathcal{O} (k\log^3 n)$.

\end{document}
\section{Sammenligning af algoritmer}
Implementeringerne af de tre algoritmer kan findes i mappen \texttt{code}.\\
For at teste de tre algoritmer har vi brugt 12 tal, der alle er Proth tal. Tallene skifter desuden mellem at være sammensatte tal og primtal. \\
For at gøre vilkårene så lige som muligt er alle algoritmerne blevet implementeret i det samme programmeringssprog (Ruby),
og testene er blevet udført på den samme computer. For hver test er der udover korrektheden af algoritmen også blevet målt tidsforbruget i sekunder.\\
Ved de probabilistiske test har vi kørt testen 64 gange for at give en høj præcision.
\begin{table}
\begin{center}
\begin{tabular}{c | c | c | c  }
\hline 
   Test tal  & Proth & Pocklington & Miller-Rabin \\ \hline
   5 & (0.000074, 1) & (0.000064, 1) & (0.000744, 1) \\ \hline
   9 & (0.000021, 1) & (0.000055, 1) & (0.000034, 1) \\ \hline
   $9*(2^7)+1$ & (0.000039, 1) & (0.000070, 1) & (0.002038, 1) \\ \hline
   $9*(2^9)+1$ & (0.000023, 1) & (0.030828, 1) & (0.000053, 1) \\ \hline
   $9*(2^{33})+1$ & (0.000020, 1) & (N/A) & (0.014823, 1) \\ \hline
   $9*(2^{31})+1$ & (0.000041, 1) & (N/A) & (0.000195, 1) \\ \hline
   $9*(2^{134})+1$ & (0.000092, 1) & (N/A) & (0.063023, 1) \\ \hline
   $9*(2^{132})+1$ & (0.000068, 1) & (N/A) & (0.001409, 1) \\ \hline
   $9*(2^{366})+1$ & (0.000330, 1) & (N/A) & (0.294020, 1) \\ \hline
   $9*(2^{368})+1$ & (0.000855, 1) & (N/A) & (0.004647, 1) \\ \hline
   $9*(2^{782})+1$ & (0.001410, 1) & (N/A) & (1.485159, 1) \\ \hline
   $9*(2^{780})+1$ & (0.001677, 1) & (N/A) & (0.023595, 1)\\ \hline
\end{tabular}
\end{center}
\caption{Testresultater}
\end{table}

\textbf{Tabel 1} viser resultaterne af vores teste. Tuplerne beskriver (tid, korrekt), hvor korrekt er 1, hvis testen 
gav det rigtige resultat og 0 ellers.\\
Hvis vi kigger på resultaterne for Proth, kan man se, at den i de fleste tilfælde er den hurtigste. Dette var forventet,
da vi i alle vores testscenarier brugte Proth tal. Til gengæld vil Proth algoritmen ikke i alle tilfælde give det rigtige svar,
da den ikke nødvendigvis virker på ikke-Proth tal.  
På dette punkt er Miller-Rabin bedre, da den kan bruges på alle heltal. Desuden, som der kan ses på resultaterne, er Miller-Rabin hurtigere,
når den anvendes på de sammensatte tal, end når den anvendes på primtal. Dette skyldes, at den skal køre alle 64 gange for at udlede,
at tallet "nok" er et primtal, men at den returnerer med det samme, hvis den finder et vidne for at tallet er sammensat.\\
Forskellen mellem de to probabilistiske test er, at Miller-Rabins usikkerhed ligger i, at den ikke kan garantere, at tallet er et primtal,
mens Proth usikkerhed er, at den ikke nødvendigvis kan finde et vidne for, om det er sammensat tal eller et primtal.\\
Da Pocklington er en deterministisk primtalstest, er det forventeligt, at den altid vil give det ønskede resultat.\\
Problemet med Pocklington er at finde $a$'er for hvilke betingelserne for teoremet gælder. Vi har ikke kunnet
finde nogen smart måde at finde disse $a$'er på. Derfor gør vi det ved en iterativ process, hvor vi kigger på en mulighed
af gangen. Dette er meget langsommeligt, hvis man arbejder med store tal, hvor $a$ ikke bliver fundet hurtigt. I
dette tilfælde vil der være mange iterationer og dermed vil algoritmen være langsom.
Det har endda den konsekvens, at hvis tallet man tjekker om er primisk ikke er et primtal, så vil testen kører alle $a$'er igennem
op til det tal, der testes. Derved er algoritmen meget ineffektiv.\\
Testene kan køres ved at køre filen \texttt{test.rb}. Denne skal dog stoppes i eksekveringen, da Pocklington testen 
er meget lang tid om at terminere. Derfor er mange af resultaterne for Pocklington markeret med N/A (not applicable).

\section{Konklusion}
Man kan konkludere, at der er forskellige tilgange til at finde ud af, om et tal er et primtal. 
De probabilistiske prøver ved brug af gentagne forsøg at forøge sandsynligheden for, at et tal er 
et primtal. De to probabilistiske teste, der er blevet gennemgået i denne rapport, har haft to forskellige
tilgange til det. Miller-Rabin kunne finde ud af, om et tal var sammensat, og hvis det ved gentagne forsøg ikke
var vist at dette var tilfældet, antog man det for at være et primtal. Proth lavede fire tjek: Et der tjekkede for om tallet 
var et primtal, to der tjekkede om tallet var sammensat, og det sidste der bestemte, hvorvidt det forblev ukendt om tallet var et primtal eller ej. 
En helt anden tilgang til primtalstest så man igennem Pocklington testen, der udnyttede faktorisering af $N-1$.
De har alle tre hver især deres fordele. For Miller-Rabin giver man for tidsforbrugets skyld afkald på at kunne garantere,
at et tal er et primtal. Ved Proth kunne man ikke altid afgøre, om et tal var et primtal  eller ej, men til genæld var der god ydeevne. 
Afslutningsvis kunne Pocklington altid afgøre, om et tal var primisk, men til gengæld havde den langsom køretid.

\nocite{*}
\bibliographystyle{plain}
\bibliography{bibliography}
\newpage
\appendix
%\input{appendix.tex}
\end{document}
