\documentclass[11pt,a4paper]{article}

\usepackage[utf8]{inputenc}
\usepackage[T1]		 {fontenc}
\usepackage{pdfpages}
\usepackage{listings}
\usepackage{amsmath,amssymb,amsfonts}
\usepackage{mathtools}
\usepackage{amsthm}
\usepackage{listings}

\DeclarePairedDelimiter\ceil{\lceil}{\rceil}
\DeclarePairedDelimiter\floor{\lfloor}{\rfloor}

\newcommand\defeq{\stackrel{\mbox{\normalfont\tiny def}}{=}}
\newcommand{\HRule}{\rule{\linewidth}{0.5mm}}

\let\oldemptyset\emptyset
\let\emptyset\varnothing

\newtheorem{smoothNumbers}{Lemma}
\newtheorem{prothTeorem}{Teorem}

\begin{document}
\subsection*{Proth tal}
Et Proth tal er et positivt heltal på formen $N=h*2^k+1$ hvor det er krævet at $k$ er ulige og $k>2^h$ gælder. \\
\begin{prothTeorem}
\label{thm:1}ved vi at
(Proths Teorem). Lad $N$ være et Proth tal som defineret før, hvis der for et heltal $a$ gælder
$$a^{(n-1)/2} \equiv -1\:(mod\:N)$$
hvor det er antaget at $(\frac{a}{N}) = -1$, så er $N$ et primtal.
\end{prothTeorem}
\begin{proof}
Vi starter med at antage at $N$ er et primtal.\\
Ved brug af Euler's kriterium kan vi omkskrive $(\frac{a}{N}) = -1$  til
$$a^{(n-1)/2} \equiv -1\:(mod\:N)$$
Under antagelse af at kongruensen i teorem \ref{thm:1} holder, så er $N-1=h*2^k$ og $\gcd(k,2^n)=1$, derfor
$$a^{N-1}=\left(a^{(N-1)/2} \right)^2 \equiv (-1)^2 \equiv 1\: (mod\: N)$$
Siden $N$ er ulige og deler $a^{(N-1)/2}$ må der nødvendigvis også gælde $\gcd(a^{(N-1)/2}-1,N)=1$. Fra Pocklingtons teorem [reference her], ved vi at enhver primtalsfaktor $p$ af $N$ er på formen
$$p=r2^k+1>2^k$$
Men 
$$N=h*2^k+1$$
Derfor
$$\sqrt{N}<2^n<p$$
Hvilket betyder at $N$ er et primtal.
\end{proof}
\subsection*{Proth test}
Proth testen er en simpel probabilistisk primtalstest, der fungerer ved at udvælge et vilkårligt heltal $a$, hvorom der ikke gælder $a \not \equiv 0\:(mod\:N)$. Her udregnes 
$$b\equiv a^{(N-1)/2}\:(mod\:N)$$
hvilket giver 4 tilfælde
\begin{enumerate}
\item Hvis $b\equiv -1\: (mod\:N)$, så er $N$ et primtal (udfra Teorem \ref{thm:1})
\item Hvis $b\not\equiv \pm 1\: (mod\:N)$ og $b^2\equiv 1\: (mod\:N)$, så er $N$ et sammensat tal, idet $\gcd(b \pm 1,N)$ er ikke-trivielle faktorer af $N$.
\item Hvis $b^2\not\equiv 1\: (mod\:N)$, så kan man udfra Fermats lille teorem slutte at $N$ er et sammensat tal
\item Hvis $b\equiv 1\: (mod\:N)$ forbliver det ukendt om hvorvidt $N$ er primtal
\end{enumerate}
Denne procedure gentages indtil at det er blevet bestemt om $N$ er et primtal eller ej. I tilfældet af at $N$ er et primtal, har testen en sandsynlighed på $\frac{1}{2}$ for at returnere at $N$ er et primtal i en enkelt iteration.
\end{document}