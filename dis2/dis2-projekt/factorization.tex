\subsection{Faktorisering}
I næste sektion beskriver vi Pocklington testen, som bygger på primtalsfaktorisering, hvor
der arbejdes ud fra, at man kan afgøre om et tal er et primtal ud fra bl.a. primtalsfaktorerne
for et andet tal. Vi har implementeret algoritmen ``trial division'', da den er simpel
at implementere og således er mulig at implementere inden for projektets rammer. Trial division
fungerer ved at finde alle primtalsfaktorer af et tal $n$ op til $\floor{\sqrt{n}}$ og dividere
tallet med den pågældende faktor, når en sådan faktor findes. Trial division starter da forfra
med at finde primtalsfaktorer for det nye tal.
Dette er dog en meget langsom algoritme til primtalsfaktorisering, og vi vil derfor
også beskrive matematikken bag algoritmen ``quadratic sieve'', som er væsentlig mere effektiv end
``trial division''. Quadratic sieve er i midlertidig ikke blevet implementeret, da
vi mente, at dette ville tage for lang tid.
\subsubsection{Quadratic sieve}
Quadratic sieve bygger på, at vi kan udtrykke $n$ som differensen mellem to kvadrattal. Ved
at bygge videre på denne ide, så søger algoritmen at finde to tal $x, y$, der opfylder
\begin{align}
  x & \not \equiv \pm \; y \pmod{n} \label{eqn:qs-condition1} \\
  x^2 & \equiv y^2 \pmod{n} \label{eqn:qs-condition2}
\end{align}
Ud fra disse to betingelser er det klart, at $(x - y)(x + y) \equiv 0 \pmod{n}$, hvilket vil
sige, at $(x - y)(x + y)$ er en divisor i $n$. Den første af betingelserne udelukker dog, at
$(x - y)$ og $(x + y)$ alene er divisorer i $n$. Således må $\text{gcd}(x - y, n)$ og
$\text{gcd}(x + y, n)$ være ikke trivielle divisorer i $n$. (gcd er funktionen til at finde
den største fælles divisor for to tal.)\\
Vi vil nu vise, hvordan quadratic sieve fungerer ved at finde primtalsfaktoriseringen for
$n = 1649$ ved at anvende ovenstående beskrivelse. (\ref{eqn:qs-condition2}) giver, at vi skal finde
to kvadrattal, som er kongruente med $n$. $x$ bestemmes ved først at lave en iterativ søgning,
hvor vi starter med $x = \ceil{\sqrt{n}} = 41$. Vi husker, at algoritmen bygger på, at
$n = x^2 - y^2$, hvorfor vi kan finde $y^2 = x^2 - n$.
\begin{align*}
  41^2 - n &= 32 \\
  42^2 - n &= 115 \\
  43^2 - n &= 200
\end{align*}
Vi har endnu ikke fundet to kvadrattal, der opfylder (\ref{eqn:qs-condition1}) og (\ref{eqn:qs-condition2}),
men alligevel vælger vi at stoppe søgningen ved $x = 43$. Vi bemærker nemlig at produktet
$32 \cdot 200 = 6400 = 80^2$ er et kvadrattal, og vi får derfor følgende identitet:
\begin{align*}
  41^2 \cdot 43^2 \equiv 80^2 \pmod{n} \iff (41 \cdot 43)^2 \equiv 80^2 \pmod{n}
\end{align*}
Vi har altså $x = 41 \cdot 43 = 1763$ og $y = 80$. Vi anvender nu gcd til at finde de
to primtalsfaktorer for $n = 1649$.\\
\begin{align*}
  \text{gcd}(x - y, n) &= \text{gcd}(1763 - 80, 1649) = \text{gcd}(1683, 1649) = 17 \\
  \text{gcd}(x + y, n) &= \text{gcd}(1763 + 80, 1649) = \text{gcd}(1843, 1649) = 97
\end{align*}
En simpel undersøgelse kan vise, at $17$ og $97$ er primtal, hvorfor vi nu har fundet
primtalsfaktoriseringen for $n = 1649 = 17 \cdot 97$.\\
Eksemplet har vist, hvordan quadratic sieve fungerer, men vi mangler at forklare,
hvordan den iterative process vælger $y_1,\cdots,y_t$, så $x_i^2 - n = y_i^2$
for $1 \le i \le t$ og produktet af alle $y_i$'er er et kvadrattal.
Dette kan gøres ved at se på ``smooth numbers''. Vi siger, at et tal er $B$-smooth,
hvis alle dets primtalsfaktorer er mindre eller lig $B$. Vi lader $\pi(B)$ betegne
antallet af primtal i intervallet $[1 ; B]$.
\begin{lemma}
(Uden bevis.) Hvis $M = m_1, m_2, \dots, m_k$ er $B$-smooth heltal og $k > \pi(B)$, så findes der
en ikke tom delsekvens af $M$, hvor produktet af tallene i delsekvensen er et kvadrattal.
\end{lemma}
Vi kan altså finde det ønskede kvadrattal ved at finde mere end $\pi(B)$ $B$-smooth
tal. Målet vil vu være at finde delsekvensen, der giver kvadrattallet. I den forbindelse
bemærker vi, at hvert $m_i$ kan repræsenteres af en vektor $v_i$ med $\pi(B)$ elementer,
hvor det $j$'te element i $v_i$ er eksponenten for det $j$'te primtal. Det ses da trivielt,
at produktet af en delsekvens er et kvadrattal, hvis alle elementerne i vektoren, der
er summen af vektorerne i delsekvensen, er lige. Vi kan således konstruere en matrix, hvor
den $i$'te række er $v_i$, og ``mod 2'' er anvendt på hvert element. Rækkerne,
der opfylder det givne kriterie, kan da findes ved at anvende gaussisk elimination.
Vi har altså nu beskrevet teorien bag quadratic sieve, hvor det dog er blev udeladt,
hvordan man effektivt kan finde $B$-smooth tal.
