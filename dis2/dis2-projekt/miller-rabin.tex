\section{Miller-Rabin}
Miller-Rabin testen afhænger af nogle ligninger, som er sande for primtal. Hvis vi har et primtal $p>2$, så ved vi, at $p-1$ er lige, og vi kan derfor skrive det unikt som $2^r \cdot m$, hvor $m$ er ulige. For ethvert heltal $a$ i gruppen $\mathbb{Z}/p\mathbb{Z}$ gælder der så, at
\begin{align}
  a^m \equiv 1 \pmod{p}
  \label{eqn:mr1}
\end{align}
eller
\begin{align}
  a^{2^s \cdot m} \equiv -1 \pmod{p}
  \label{eqn:mr2}
\end{align}
for $0 \leq s \leq r-1$. Dette vil vi også gerne vise er sandt. Til det bruger vi en simpel omskrivning af
Fermats lille teorem:
\begin{theorem}
Lad $p$ være et primtal. Hvis $p$ ikke er en divisor i $a$, da er
\begin{align*}
  a^{p-1} \equiv 1 \pmod{p}
\end{align*}
\label{fermats-little-theorem}
\end{theorem}
Bemærk at $p$ ikke er en divisor i $a$, da $a$ er et heltal i gruppen, $\mathbb{Z}/p\mathbb{Z}$, hvorfor $0 < a < p$.\\
Til beviset vil vi også gøre bruge af følgende lemma:
\begin{lemma}
Hvis vi har et kvadrat $x^2$ kongruent med $1$ modulo $p$, så gælder det, at
\begin{align*}
  &x^2 \equiv 1 \pmod{p} \iff (x-1)(x+1) \equiv 0 \pmod{p}
\end{align*}
Med andre ord er $p$ en divisor i $(x-1)(x+1)$.
\label{lemma:mr1}
\end{lemma}
Dette betyder så, at
\begin{align*}
  x \equiv \pm 1 \pmod{p}
\end{align*}
Ved at kombinere teorem \ref{fermats-little-theorem} og lemma \ref{lemma:mr1} kan vi altså blive ved med at tage kvadratroden af $a^{p-1}$.
Hvis vi får $-1$, så holder (\ref{eqn:mr2}). Hvis vi aldrig får $-1$, så er det fordi, vi ikke har flere potenser af $2$, og så vil (\ref{eqn:mr1}) holde.
Miller-Rabin testen afhænger dog af kontrapositionen - nemlig at
\begin{align*}
  a^m \not \equiv 1 \pmod{p}
\end{align*}
eller
\begin{align*}
  a^{2^s \cdot m} \not \equiv -1 \pmod{p}
\end{align*}
for $0\leq s\leq r-1$. 
\subsection{Algoritmen}
Miller-Rabin testen afgør, om et tal $n$ er et sammensat tal eller, om det sandsynligvis er et primtal.
Dette skyldes, at det er en probabilistisk algoritme - altså at den afhænger af tilfældighed.
Miller-Rabin algoritmens argumenter kræver, udover tallet $n$, også en parameter $k$, der er antallet af gange, algoritmen bliver gentaget.
For hver ekstra gang algoritmen bliver kørt, hvor der bliver returneret at $n$ nok er primtal, falder sandsynligheden for, at $n$ ren faktiskt er et sammensat tal.
Altså ses $k$ som en parameter for korrekthed. Tallene $a$ kaldes for ``vidner'', hvis de angiver, at $n$ er et sammensat tal.
Algoritmen kan deles i fire trin for $n \geq 5$.
\begin{enumerate}
  \item Find de unikke tal $r$ og $m$ så $n-1=2^r\cdot m$,, og $m$ er ulige.
  \item Vælg et tilfældigt heltal $a$ hvor $1 < a < n$.
  \item Sæt $b = a^m \mod n$. Hvis $b \equiv \pm 1 \pmod{n}$, så er $n$ nok et primtal.
  \item Hvis $b^{2^s}\equiv -1 \pmod{n}$ for et $s$, hvor $1 \leq s \leq r-1$, så er $n$ nok et primtal. Hvis ikke så er $n$ et sammensat tal.
\end{enumerate}
Algoritmen er blevet implementeret i Ruby og kan findes i filen \texttt{millerrabin.rb}.
Algoritmen er blevet implementeret med henblik på at kunne understøtte flere kørsler for at opnå flere baser $a$. Outputtet kan findes i sammenligningen, hvor vi har brugt et testsæt af 12 tal, hvor det første tal er et primtal og derefter (burde) den alternere mellem et sammensat tal og et primtal.

\subsection{Præcision og køretid}
Det kan vises, at der mindst er $3/4$ vidner $a$ for et ulige sammensat heltal. Altså vil der for $k$ kørsler på $n$, der siger $n$ er et primtal,
højest være en sandsynlighed på $4^{-k}$, for at $n$ faktiskt er et sammensat. En effektiv implementering af modular eksponentiering kører
i $\mathcal{O} (\log n)$ tid. Dette vil give os en samlet køretid på $\mathcal{O} (k\log^3 n)$.
