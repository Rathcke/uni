\documentclass[12pt]{article}
\usepackage[utf8]{inputenc}
\usepackage{amsmath}
\usepackage{amssymb}
\usepackage{mathtools}
\usepackage{amsfonts}
\usepackage{lastpage}
\usepackage{tikz}
\usepackage{pdfpages}
\usepackage{gauss}
\usepackage{fancyvrb}
\usepackage{fancyhdr}
\usepackage{graphicx}
\usepackage[margin=3.5cm]{geometry}
\pagestyle{fancy}
\fancyfoot[C]{\footnotesize Page \thepage\ of 3}
\DeclareGraphicsExtensions{.pdf,.png,.jpg}
\author{Nikolaj Dybdahl Rathcke}
\chead{Nikolaj Dybdahl Rathcke (rfq695)}

\begin{document}

\section{Sammenligning}
\subsection*{Test}

\subsubsection*{Forventning}
\textbf{Pocklington}:\\
Da Pocklington er en deterministisk primtalstest er det forventeligt at den altid vil give det ønskede resultat.
Problemet med Pocklington er at finde \textit{a}'er for hvilke betingelserne for teoremet gælder. Vi har ikke kunnet
finde nogen smart måde at finde disse \textit{a}'er på. Derfor gør vi det ved en iterativ process hvor vi kigger på en mulighed
af gangen. Dette er meget langsommeligt hvis man arbejder med store tal hvor \textit{a} ikke bliver fundet hurtigt. I
dette tilfælde vil der være mange iterationer og dermed vil algoritmen være langsom.
Det har endda den konsekvens at hvis tallet man tjekker om er primisk ikke er et primtal så vil testen kører alle \textit{a}'er igennem op til det tal der testes, dermed er algoritmen meget ineffektiv tests. Det kan derfor forventes at algoritmen vil
være langsom for store tal.\\\\
\textbf{Proth}:\\

\textbf{Miller-Rabin}:\\


\subsubsection*{Test af implementationer}
For at teste de tre algoritmer har vi brugt 12 der alle er Proth tal. Tallene skifter desuden mellem at være sammensatte tal og være primtal. \\
For at gøre vilkårene så lige som muligt er alle algoritmerne blevet implementeret i Ruby og testene er blevet udført på den samme computer. 
For hver test er der udover korrektheden af algoritmen også blevet målt tidsforbruget i sekunder.\\
Ved de probabilistiske test har vi kørt testen 64 gange for give en højere præcision.\\ 
\begin{table}
\begin{center}
\begin{tabular}{c | c | c | c  }
\hline 
     & Proth & Pocklington & Miller-Rabin \\ \hline
   5 & (0.000074, 1) & (0.000064, 1) & (0.000744, 1) \\ \hline
   9 & (0.000021, 1) & (0.000055, 1) & (0.000034, 1) \\ \hline
   $9*(2^7)+1$ & (0.000039, 1) & (0.000070, 1) & (0.002038, 1) \\ \hline
   $9*(2^9)+1$ & (0.000023, 1) & (0.030828, 1) & (0.000053, 1) \\ \hline
   $9*(2^{33})+1$ & (0.000020, 1) & (N/A) & (0.014823, 1) \\ \hline
   $9*(2^{31})+1$ & (0.000041, 1) & (N/A) & (0.000195, 1) \\ \hline
   $9*(2^{134})+1$ & (0.000092, 1) & (N/A) & (0.063023, 1) \\ \hline
   $9*(2^{132})+1$ & (0.000068, 1) & (N/A) & (0.001409, 1) \\ \hline
   $9*(2^{366})+1$ & (0.000330, 1) & (N/A) & (0.294020, 1) \\ \hline
   $9*(2^{368})+1$ & (0.000855, 1) & (N/A) & (0.004647, 1) \\ \hline
   $9*(2^{782})+1$ & (0.001410, 1) & (N/A) & (1.485159, 1) \\ \hline
   $9*(2^{780})+1$ & (0.001677, 1) & (N/A) & (0.023595, 1)\\ \hline
\end{tabular}
\end{center}
\caption{Testresultater}
\end{table}

\textbf{Tabel 1} viser resultaterne af vores tests. Tuplerne beskriver (tid, korrekt) hvor korrekt er 1 hvis testen 
gav det rigtige resultat og 0 ellers.\\
Hvis vi kigger på resultaterne for Proth kan man se at den i de fleste tilfælde er den hurtigste. Dette var forventet
da vi i alle vores testcases brugte Proth tal. Til gengæld vil Proth algoritmen ikke i alle tilfælde give det, da den ikke nødvendigvis virker på ikke-Proth tal.  
På dette punkt er Miller-Rabin bedre, da den kan bruges på alle heltal. Desuden, som der kan ses på resultaterne, er Miller-Rabin hurtigere når den anvendes på de sammensatte tal end når den anvendes på primtal. Dette skyldes, at den skal køre alle 64 gange for at udlede at tallet "nok" er et primtal, men at den returnerer med det samme hvis den finder et vidne for at tallet er sammensat.\\
Forskellen mellem de to probabilistiske test er, at Miller-Rabins usikkerhed ligger i at den ikke kan garantere det er et primtal, mens Proth usikkerhed er at den ikke nødvendigvis kan finde et vidne for om det er sammensat eller et primtal.\\
Som det også var forventet, så er Pocklington langsom på store tal og testen blev derfor stoppet før termineringen. Den er dog determineret - altså at resultatet er garanteret korrekt.\\
Testene kan køres ved at udføres ved at køre filen \texttt{test.rb}. Denne skal dog stoppes i eksekveringen da Pocklington testen 
er meget lang tid om at terminere. 

\end{document}
