\documentclass[12pt]{article}
\usepackage[utf8]{inputenc}
\usepackage{amsmath}
\usepackage{amssymb}
\usepackage{mathtools}
\usepackage{amsfonts}
\usepackage{lastpage}
\usepackage{tikz}
\usepackage{pdfpages}
\usepackage{gauss}
\usepackage{fancyvrb}
\usepackage{fancyhdr}
\usepackage{graphicx}
\usepackage[margin=3.5cm]{geometry}
\pagestyle{fancy}
\fancyfoot[C]{\footnotesize Page \thepage\ of 3}
\DeclareGraphicsExtensions{.pdf,.png,.jpg}
\author{Nikolaj Dybdahl Rathcke}
\chead{Nikolaj Dybdahl Rathcke (rfq695)}

\begin{document}

\section{Miller-Rabin}
\subsection*{Konceptet}
Miller-Rabin testen afhænger af nogle ligninger som er sande for primtal. Hvis vi har et primtal $p>2$, så ved vi at $p-1$ er lige og vi kan derfor skrive det unikt som $2^r\cdot m$, hvor $m$ er ulige. For ethvert heltal $a$ i gruppen $\mathbb{Z}/p\mathbb{Z}$ gælder der så, at
\begin{align*}
&a^m\equiv 1 \ (\mbox{mod }p) \\
\mbox{eller}\ \ \ \ \ \ &\ \\
&a^{2^s\cdot m}\equiv -1 \ (\mbox{mod }p)
\end{align*}
For et $s\in 0\leq s\leq r-1$. Dette vil vi også gerne vise er sandt. Til det bruger vi, når $p$ er et primtal, at
\begin{align}
a^{p-1}\equiv 1 \ (\mbox{mod }p)
\end{align}
Som er en simpel omskrivning af Fermats lille theorem, der gælder når $a$ ikke er dividerbart af $p$ som det aldrig kan være når $0<a<p$	.\\
Desuden ser vi på et andet lemma, nemlig at hvis vi har et kvadrat er lig $1$ modulo $p$, så
\begin{align*}
&x^2\equiv 1 \ (\mbox{mod }p) \\
\Leftrightarrow\ \ &(x-1)(x+1)\equiv 0 \ (\mbox{mod }p)
\end{align*}
Altså at $p\ \backslash\ (x-1)(x+1)$. Dette betyder så at
\begin{align*}
x\equiv 1\mbox{ or }-1 \ (\mbox{mod }p)
\end{align*}
Bruger vi dette sammen med (1) kan vi altså blive ved med at tage kvadratroden af $a^{p-1}$. Hvis vi får $-1$ holder den anden ligning. Hvis vi aldrig får $-1$ så er det fordi vi ikke har flere potenser af $2$ og så vil den første ligning holde.\\
Miller-Rabin testen afhænger dog af kontrapositionen, nemlig at
\begin{align*}
&a^m\not\equiv 1 \ (\mbox{mod }p) \\
\mbox{eller}\ \ \ \ \ \ &\ \\
&a^{2^s\cdot m}\not\equiv -1 \ (\mbox{mod }p)
\end{align*}
For $0\leq s\leq r-1$. 


\subsection*{Algoritmen}
Miller-Rabin testen afgør derfor om et givet tal, $n$, er et sammensat tal eller om det er "nok et primtal". Dette skyldes at det er en probabilistisk algoritme, altså at den afhænger af tilfældighed. \\
Et kald til Miller-Rabin algoritmen kræver udover tallet $n$ også en parameter $k$ som er antallet af gange algoritmen bliver gentaget. For hver ekstra gang algoritmen bliver kørt hvor der bliver returneret at $n$ nok er primtal, falder sandsynligheden for at $n$ ren faktiskt er et sammensat tal. Altså ses $k$ som en parameter for korrekthed. Tallene $a$ kaldes for "vidner" hvis de angiver $n$ er et sammensat tal. \\
Algoritmen kan deles i fire trin for $n\geq 5$.
\begin{enumerate}
\item Find de unikke tal $r$ og $m$, så $n-1=2^r\cdot m$ og $m$ er ulige.
\item Vælg et tilfældigt heltal $a$, hvor $1<a<n$.
\item Sæt $b=a^m$ (mod $n$). Hvis $b\equiv \pm 1$ (mod $n$), så er $n$ nok et primtal.
\item Hvis $b^{2^s}\equiv -1$ (mod $n$) for et $s$ hvor $1\leq s\leq r-1$, så er $n$ nok et primtal. Hvis ikke, så er $n$ et sammensat tal.
\end{enumerate}
Den er implementeret i Ruby og filen \texttt{millerrabin.rb} kan ses i Appendix 1. Algoritmen er implementeret med henblik på at kunne understøtte flere kørsler for at opnå flere baser $a$. \\
Vedhæftet er også outputtet når den køres på testsættet "numbers", hvor de første 10 tal er primtal og de sidste 10 er sammensatte tal.

\subsection*{Præcision og Køretid}
Det kan vises at der mindst er $3/4$ vidner $a$ for et ulige sammensat heltal. Altså vil der for $k$ kørsler på $n$ der siger $n$ er et primtal højst være en sandsynlighed på $4^{-k}$ for at $n$ rent faktiskt er sammensat.\\
En effektiv implementering af modular eksponentiering, $b^e\ \mbox{mod}\ m$ kan implementeres i $\mathcal{O} (lg(e))$. Da vores eksponent, $m$, i det ydre loop maksimalt kan være $n/2$, har vi altså en køretid på $\mathcal{O} (k\ lg(n))$. \\
Desuden kører den også det indre loop $k$ gange, hvor $r$ maksimalt kan være $lg(n)$. Desuden er eksponenten $2$ så dette lægge ikke synderligt meget til køretiden.
Altså har vi en samlet køretid på $\mathcal{O} (k\ lg(n)^2)$.\\
\\
Det kan vises at der mindst er $3/4$ vidner $a$ for et ulige sammensat heltal. Altså vil der for $k$ kørsler på $n$ der siger $n$ er et primtal højst være en sandsynlighed på $4^{-k}$ for at $n$ rent faktiskt er sammensat.\\
En effektiv implementering af modular eksponentiering kører i $\mathcal{O} (\log n)$ tid. \\
Dette giver en samlet køretid på $\mathcal{O} (k\log^3 n)$.


\newpage
\section*{Appendix 1} \label{App:App1}
\begin{verbatim}
require 'benchmark'
require 'openssl'
 
def millerrabin(n,k)
    if n < 5
        return "n need to be greater than 4"
    end
    if n % 2 == 0
        return "#{n} is composite"
    end
    m = n-1;
    r = 0;
    while (m % 2 == 0)
        r += 1;
        m = m/2;
    end
    for i in 1..k do
        a = rand(2..n-2);
        b = a.to_bn.mod_exp(m, n)                   # b = a^m % n
        if (b == 1 or b == n-1)
            next;
        end
        for j in (1..(r-1))
            b = b.to_bn.mod_exp(2,n)                # b = b^2 % n
            if (b == 1)
              return "#{n} is composite"
            end
            if (b == n-1)
                break;
            end
        end
        if (b != n-1)
            return "#{n} is composite";
        end
    end
    return "#{n} is probably prime";
end
 
numbers = [48112959837082048697, 54673257461630679457,
29497513910652490397, 40206835204840513073, 12764787846358441471,
71755440315342536873, 45095080578985454453, 27542476619900900873,
66405897020462343733, 36413321723440003717, 24918723678509309615,
37813311192850760755, 50608046447142653959, 32395252670977894437,
19601290870353624837, 13875069371204789371, 41875470038172930717,
23238616604259962647, 31242090723917058139, 92427220233223221283]

Output:
48112959837082048697 is probably prime
54673257461630679457 is probably prime
29497513910652490397 is probably prime
40206835204840513073 is probably prime
12764787846358441471 is probably prime
71755440315342536873 is probably prime
45095080578985454453 is probably prime
27542476619900900873 is probably prime
66405897020462343733 is probably prime
36413321723440003717 is probably prime
24918723678509309615 is composite
37813311192850760755 is composite
50608046447142653959 is composite
32395252670977894437 is composite
19601290870353624837 is composite
13875069371204789371 is composite
41875470038172930717 is composite
23238616604259962647 is composite
31242090723917058139 is composite
92427220233223221283 is composite
\end{verbatim}

\end{document}
