\subsection{Proth}
Et Proth tal er et positivt heltal på formen $N = h \cdot 2^k + 1$, hvor det er krævet, at $k$ er ulige, og $k > 2^h$.
\begin{theorem}
\label{thm:1}
(Proths Teorem). Lad $N$ være et Proth tal. Hvis der for et heltal $a$ gælder
\begin{align*}
  a^{(n-1)/2} \equiv -1 \pmod{N}
\end{align*}
hvor det er antaget, at Jacobi symbolet $(\frac{a}{N}) = -1$, så er $N$ et primtal.
\end{theorem}
\begin{proof}
Vi starter med at antage, at $N$ er et primtal. Vi får nu brug for Eulers kriterium, som vi (uden bevis)
ser i lemma \ref{eulers-criterion}.
\begin{lemma}
Lad $p$ være et ulige primtal, og lad $a$ være et heltal, så $p$ og $a$ er indbyrdes primiske. Da har vi
\begin{align*}
  \left( \frac{a}{p} \right) \equiv a^{(p-1)/2} \pmod{p}
\end{align*}
\label{eulers-criterion}
\end{lemma}
Ved brug af Euler's kriterium kan vi omkskrive $(\frac{a}{N}) = -1$  til
\begin{align*}
  a^{(N-1)/2} \equiv -1 \pmod{N}
\end{align*}
Under antagelse af at kongruensen i teorem \ref{thm:1} holder, så er $N-1 = h \cdot 2^k$ og $\gcd(k, 2^n) = 1$. Derfor får vi
\begin{align*}
  a^{N-1} = \left( a^{(N-1)/2} \right)^2 \equiv (-1)^2 \equiv 1 \pmod{N}
\end{align*}
Siden $N$ er ulige og deler $a^{(N-1)/2}$, må der nødvendigvis også gælde, $\gcd(a^{(N-1)/2}-1,N)=1$.
Fra Pocklingtons teorem (teorem \ref{theorem:pocklingtons-theorem}), ved vi, at enhver primtalsfaktor $p$ for $N$ er på formen
\begin{align*}
  p =r \cdot 2^k + 1 > 2^k
\end{align*}
Dog har vi også $N = h \cdot 2^k+1$, hvorfor $\sqrt{N} < 2^n < p$. Dette betyder at $N$ er et primtal.
\end{proof}
\subsubsection{Proth test}
Proth testen er en simpel probabilistisk primtalstest, der fungerer ved at udvælge et vilkårligt heltal $a$,
hvorom der ikke gælder $a \not \equiv 0 \pmod{N}$. Her udregnes
\begin{align*}
  b \equiv a^{(N-1)/2} \pmod{N}
\end{align*}
hvilket giver 4 tilfælde:
\begin{enumerate}
  \item Hvis $b \equiv -1 \pmod{N}$, så er $N$ et primtal (udfra Teorem \ref{thm:1})
  \item Hvis $b \not \equiv \pm 1 \pmod{N}$, og $b^2 \equiv 1 \pmod{N}$, så er $N$ et sammensat tal, idet $\gcd(b \pm 1, N)$ er ikke-trivielle faktorer af $N$
  \item Hvis $b^2 \not \equiv 1 \pmod{N}$, så kan man udfra Fermats lille teorem slutte, at $N$ er et sammensat tal
  \item Hvis $b \equiv 1 \pmod{N}$, forbliver det ukendt, hvorvidt $N$ er primtal
\end{enumerate}
Denne procedure gentages indtil, det er blevet bestemt, om $N$ er et primtal eller ej. I tilfældet af at $N$ er et primtal,
har testen en sandsynlighed på $\frac{1}{2}$ for at returnere, at $N$ er et primtal i en enkelt iteration.
