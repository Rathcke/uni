\documentclass[12pt]{article}
\usepackage[dot, autosize, outputdir="dotgraphs/"]{dot2texi}
\usepackage{tikz}
\usetikzlibrary{shapes}
\usepackage[utf8]{inputenc}
\usepackage{amsmath}
\usepackage{mathtools}
\usepackage{amsfonts}
\usepackage{lastpage}
\usepackage{pdfpages}
\usepackage{gauss}
\usepackage{fancyvrb}
\usepackage{fancyhdr}
\usepackage{graphicx}
\pagestyle{fancy}
\fancyfoot[C]{\footnotesize Page \thepage\ of 8}
\DeclareGraphicsExtensions{.pdf,.png,.jpg}
\title{Oversætter}
\author{Nikolaj Dybdahl Rathcke}
\chead{Nikolaj Dybdahl Rathcke (rfq695)}

\begin{document}
\section*{Oversætter - Week 2}

\subsection*{1 - Writing Context-Free Grammars}
Write unambiguous grammars for the following languages over the alphabet $\Sigma = \{a, b, c\}$

\subsubsection*{a}
Words that match regular expression $a^*b^*$ which contain more a's than b's
\begin{align*}
S&\rightarrow A \\
A&\rightarrow aA \\
A&\rightarrow aB \\
B&\rightarrow aBb \\
B&\rightarrow \varepsilon
\end{align*}

\subsubsection*{b}
Palindromes
\begin{align*}
S&\rightarrow T \\
T&\rightarrow aTa \\
T&\rightarrow bTb \\
T&\rightarrow cTc \\
T&\rightarrow a|b|c \\
T&\rightarrow \varepsilon
\end{align*}

\subsubsection*{c}
Write mosmlyacc grammar files for your grammars to check them. A grammar which does not cause conflicts is certain to be unambiguous.\\
However, for (b), it will not be possible to get a grammar without conflicts. Why?\\
\\
Grammer files are uploaded separately as "1a.grm" and "1b.grm". \\
Since there is a look-ahead on 1, if you have a palindrome that has an odd length, it's not possible to know what production you need to make, e.g it's not possible to know if it's $T\rightarrow aTa$ or $T\rightarrow a$.

\newpage

\subsection*{2 - LL(1)-Parser Construction}
Construct an LL(1) parser, taking the following grammar as a starting point:
\begin{align*}
Z &\rightarrow b\:|\:X Y Z \\
Y &\rightarrow \varepsilon\:|\: c \\
X &\rightarrow Y\:|\: a
\end{align*}
with the terminal symbols a, b, and c.
\subsubsection*{a}
Determine which nonterminals are nullable and calculate first sets of all right-hand sides of the productions.\\
\\
$Y$ is nullable since 
$$NULLABLE(Y)=NULLABLE(\varepsilon) \vee NULLABLE(c) =true$$
$X$ is nullable because 
$$NULLABLE(X)=NULLABLE(Y) \vee NULLABLE(a) =true$$
$Z$ is not nullable since
\begin{align*}
NULLABLE(Z)&=NULLABLE(a) \vee NULLABLE(XYZ) \\
&=NULLABLE(a)\:\vee \\
&\:\:\:\:\:(NULLABLE(X) \wedge NULLABLE(Y) \wedge NULLABLE(Z))\\
&=false \vee (true \wedge true \wedge NULLABLE(Z)) \\
&=NULLABLE(Z)
\end{align*}
which is an infinite loop, so $Z$ is not nullable.\\
\newpage
Or it can be calculated by fixed-point iteration\\
\\
\begin{tabular}{c|c|c|c|c}
\hline 
Right-hand side & Initialisation & Iteration 1 & Iteration 2 & Iteration 3 \\ 
\hline 
b & false & false & false & false \\ 
XYZ & false & false & false & false \\ 
$\varepsilon$ & false & true & true & true \\ 
c & false & false & false & false \\  
Y & false & false & true & true \\ 
a & false & false & false & false \\ 
\hline 
Nonterminals &  &  &  &  \\ 
\hline
Z & false & false & false & false \\ 
Y & false & true & true & true \\  
X & false & false & true & true \\ 
\hline 
\end{tabular}\\
\\
We want to find first sets of the right hand sides of the production. We get that
\begin{align*}
FIRST(b)&=\{b\} \\
FIRST(XYZ)&=\{a,b,c\} \\
FIRST(\varepsilon)&=\emptyset \\
FIRST(c)&= \{c\} \\
FIRST(Y)&=\{c\} \\
FIRST(a)&=\{a\}
\end{align*}
These can be calculated by fixed point iteration as well\\
\\
\begin{tabular}{c|c|c|c|c}
\hline 
Right-hand side & Initialisation & Iteration 1 & Iteration 2 & Iteration 3 \\ 
\hline 
b & $\emptyset$ & \{b\} & \{b\} & \{b\} \\ 
XYZ & $\emptyset$ & $\emptyset$ & \{a,b,c\} & \{a,b,c\} \\ 
$\varepsilon$ & $\emptyset$ & $\emptyset$ & $\emptyset$ & $\emptyset$ \\ 
c & $\emptyset$ & \{c\} & \{c\} & \{c\} \\  
Y & $\emptyset$ & $\emptyset$ & \{c\} & \{c\} \\ 
a & $\emptyset$ & \{a\} & \{a\} & \{a\} \\ 
\hline
\end{tabular}

\subsubsection*{b}
Calculate follow sets for all nonterminals (adding an extra start production to recognise the end of the input, denoted by "\textdollar") \\
\\
We use the algorithm from page 59 in the book and use it on the grammar with $Z'\rightarrow Z\$ $ added to it. We then get these constraints
\begin{align*}
\{\$\} &\subseteq FOLLOW(Z) \\
\{a,b,c\} &\subseteq FOLLOW(X) \\
FOLLOW(Z) &\subseteq FOLLOW(X) \\
\{a,b,c\} &\subseteq FOLLOW(Y) \\
FOLLOW(X) &\subseteq FOLLOW(Y)
\end{align*}
Now solving the constraints give us
\begin{align*}
FOLLOW(X)&=\{a,b,c,\$\} \\
FOLLOW(Y)&=\{a,b,c,\$\} \\
FOLLOW(Z)&=\{\$\}
\end{align*}

\subsubsection*{c}
Determine the look-ahead sets of all productions and put together a parse table for a predictive parser.\\
\\
Since $X\rightarrow Y$ and $Y\rightarrow \varepsilon$ are nullable, we get the following look-ahead sets for the productions
\begin{align}
LA(Z'\rightarrow Z\$ ) &= \{a,b,c,\$\}  \\
LA(Z\rightarrow b)&=\{b\} \\
LA(Z\rightarrow XYZ)&=\{a,b,c\} \\
LA(Y\rightarrow \varepsilon)&=\{a,b,c,\$\} \\
LA(Y\rightarrow c) &= \{c\} \\
LA(X\rightarrow Y) &= \{a,b,c,\$\} \\
LA(X\rightarrow a) &= \{a\}
\end{align}
\newpage
$ $\\
We create the parse table
\begin{center}
\begin{tabular}{c|c|c|c|c} 
Stack & a & b & c & \$ \\ 
\hline 
Z' & Z\$, 1 & Z\$, 1 & Z\$, 1 & Z\$, 1\\ 
\hline 
X & Y, 6 $\vee$ a, 7  & Y, 6 & Y, 6 & Y, 6 \\ 
\hline 
Y & $\varepsilon$, 4 & $\varepsilon$, 4  & $\varepsilon$, 4 $\vee$ c, 5  & $\varepsilon$, 4 \\ 
\hline 
Z & XYZ, 3 & XYZ, 3 $\vee$ b, 2 & XYZ, 3 & error \\ 
\hline 
a & pop & error & error & error \\ 
\hline 
b & error & pop & error & error \\ 
\hline 
c & error & error & pop & error \\ 
\hline 
\$ & error & accept & error & accept \\ 
\hline 
\end{tabular}
\end{center}
$ $\\
There are 3 conflicts.
\newpage

\subsection*{3 - SLR Parser Construction}
Make up a very small grammar which contains left-recursion, to demonstrate that left-recursion is not a problem for LR-Parsing.

\subsubsection*{a}
Show that your grammar does not generate conflicts (by providing a parse table).\\
\\
I have created a grammar below which is using left recursion
\begin{align*}
S&\rightarrow A\$ &(0) \\
A&\rightarrow Aa &(1) \\
A&\rightarrow b &(2)
\end{align*}
From the output of "3a.grm" we can create the parse table.
\begin{center}
\begin{tabular}{c|c|c|c|c|c}
 & a & b & \$ & A & S\\ 
\hline 
0 & s1 & s1 & s2 & g2 & g2  \\
\hline 
1 &  & s3 &  & g4 &  \\ 
\hline 
2 &  &  & a & & \\ 
\hline 
3 &  & r2 &  &  &  \\ 
\hline 
4 & s5 &  & r3 &  \\ 
\hline 
5 & r1 &  &  &  &  \\ 
\hline 
\end{tabular} 
\end{center}
And we can tell there are no conflicts.

\subsubsection*{b}
Compare your grammar to an equivalent one that uses right-recursion. How does the parse stack grow when parsing input?\\
\\
The equivalent right recursive grammar is
\begin{align*}
S&\rightarrow A\$ &(0) \\
A&\rightarrow bB &(1)\\
B&\rightarrow aB &(2)\\
B&\rightarrow \varepsilon&(3)
\end{align*}
\newpage
$ $\\
Parse stack for the left recursive grammar
\begin{center}
\begin{tabular}{|c|c|c|}
\hline
Stack & Input & Action \\ 
\hline 
$\varepsilon$ & baa\$ & shift \\ 
\hline 
b & aa\$ & reduce 2 \\ 
\hline 
A & aa\$ & shift \\ 
\hline 
Aa & a\$ & reduce 1 \\ 
\hline 
A & a\$ & shift \\ 
\hline
Aa & \$ & reduce 1 \\
\hline
A & \$ & reduce 0 \\
\hline
S & \$ & accept \\
\hline
\end{tabular}\\
\end{center}
And the parse stack for the right recursive grammar
\begin{center}
\begin{tabular}{|c|c|c|}
\hline 
Stack & Input & Action \\ 
\hline 
$\varepsilon$ & baa\$ & shift \\ 
\hline 
b & aa\$ & shift \\ 
\hline 
ba & a\$ & shift \\ 
\hline 
baa & \$ & reduce 3 \\ 
\hline 
baaB & \$ & reduce 2 \\ 
\hline
baB & \$ & reduce 2 \\
\hline
bB & \$ & reduce 1 \\
\hline
A & \$ & reduce 0 \\
\hline
S & \$ & accept \\
\hline
\end{tabular}\\
\end{center}
So right-recursive reduces when it has read the whole input while it for left-recursive reduces along the way. Thus, the stack is smaller for left recursive.

\end{document}
