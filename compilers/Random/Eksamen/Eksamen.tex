\documentclass[12pt]{article}
\usepackage[dot, autosize, outputdir="dotgraphs/"]{dot2texi}
\usepackage{tikz}
\usetikzlibrary{shapes}
\usepackage[utf8]{inputenc}
\usepackage{amsmath}
\usepackage{mathtools}
\usepackage{amsfonts}
\usepackage{lastpage}
\usepackage{pdfpages}
\usepackage{gauss}
\usepackage{fancyvrb}
\usepackage{fancyhdr}
\usepackage{graphicx}
\pagestyle{fancy}
\fancyfoot[C]{\footnotesize Page \thepage\ of 5}
\DeclareGraphicsExtensions{.pdf,.png,.jpg}
\title{Oversætter}
\author{Nikolaj Dybdahl Rathcke}
\chead{Nikolaj Dybdahl Rathcke (rfq695)}

\begin{document}
\section*{Compiler - Exam}

\subsection*{Task 1 - Grammar transformation for LL(1)}
We have the following grammar $G_{T_{UP}}$ with the 4 terminals: id ( ) ,
\begin{align*}
S &\rightarrow id \\
S &\rightarrow (\:T\:) \\
T &\rightarrow S\:,\:T \\
T &\rightarrow S
\end{align*}
for tuple expressions for LL(1) parsing.
\subsubsection*{a}
Give a short reason why this grammar is not LL(1), and transform the grammar (using a well-known transformation) to obtain a grammar $G'_{T_{UP}}$ suitable for LL(1) parsing.\\
\\
To make the grammer LL(1), we want to
\begin{enumerate}
\item Eliminate ambiguity
\item Eliminate left-recursion
\item Perform left factorisation where required
\end{enumerate}
Since the two productions for $T$ begins with the same symbol, this means they have overlapping \textit{FIRST} sets and is not LL(1). Therefore we need to left-factor $T$ by making the a simple production with the same common prefix, $S$ and creating another nonterminal, so we get the following grammar $G'_{T_{UP}}$
\begin{align*}
S &\rightarrow id \\
S &\rightarrow (\:T\:) \\
T &\rightarrow S\:R \\
R &\rightarrow \:,\:T \\
R &\rightarrow \varepsilon
\end{align*}

\subsubsection*{b}
For $G'_{T_{UP}}$, determine \textit{FIRST} sets for all right-hand sides.\\
\\
We calculate these through fixed-point iteration using the rules from \textbf{Algorithm 2.5}\footnote{Torben Ægidius Mogensen, Introduction to Compiler Design, page 55}.  \\
\\
\begin{tabular}{c|c|c|c|c}
\hline 
Right-hand side & Initialisation & Iteration 1 & Iteration 2 & Iteration 3 \\ 
\hline 
id & $\emptyset$ & \{id\} & \{id\} & \{id\} \\ 
( T ) & $\emptyset$ & \{$\:$($\:$\} & \{$\:$($\:$\} & \{$\:$($\:$\} \\ 
S R & $\emptyset$ & $\emptyset$ & \{id, ($\:$\} & \{id, ($\:$\} \\ 
, T & $\emptyset$ & \{$\:$,$\:$\} & \{$\:$,$\:$\} & \{$\:$,$\:$\} \\  
$\varepsilon$ & $\emptyset$ & $\emptyset$ & $\emptyset$ & $\emptyset$ \\ 
\hline
\end{tabular} \\
\\
And then we reach a fixed point.\\
So we have the following \textit{FIRST} sets
\begin{align*}
FIRST(id) &=\{id\} \\
FIRST(\:(\:T\:)\:) &= \{\:(\:\} \\
FIRST(S\:R) &= \{id,\:(\:\} \\
FIRST(,\:T) &= \{\:,\:\} \\
FIRST(\varepsilon) &= \emptyset
\end{align*}

\subsubsection*{c}
Add a start production $S' \rightarrow S\$$ and determine \textit{FOLLOW} sets for all nonterminals.\\
\\
This yields the grammar
\begin{align*}
S' &\rightarrow S\$ \\
S &\rightarrow id \\
S &\rightarrow (\:T\:) \\
T &\rightarrow S\:R \\
R &\rightarrow \:,\:T \\
R &\rightarrow \varepsilon
\end{align*}
To calculate the \textit{FOLLOW} sets, we follow an algorithm\footnote{Torben Ægidius Mogensen, Introduction to Compiler Design, page 59} and get the following constraints \\
\\
For the production $S' \rightarrow S\$$ we add the following constraint since '\$' is seen as a terminal and is therefore the \textit{FIRST} set of what follows $S$ but '\$' is not nullable.
\begin{align*}
\{\$\} \subseteq \{S\}
\end{align*}
For the production $S \rightarrow id$ there are no nonterminals, so no constraints are added.\\\\
For the production $S \rightarrow (\:T\:)$ we add the following constraint since ')' is the \textit{FIRST} set of what follows $T$ but is not nullable.
\begin{align*}
\{\:)\:\} \subseteq \{T\}
\end{align*}
For the production $T \rightarrow S\:R$ we make a split for each occurence. Looking at the nonterminal $S$ we get the constraint
$$ \{\:,\:\} \subseteq \{S\} $$
since it is the \textit{FIRST} set of $R$. Furthermore, since $R$ is nullable, we add the following constraint
$$ \{T\} \subseteq \{S\} $$
Now looking at $R$ there is nothing in the \textit{FIRST} set, so no constraint is added. However, that nothing follows means that it is nullable, so we add the constraint
$$ \{T\} \subseteq \{R\} $$
For the production $R \rightarrow \:,\:T$ we get nothing in the \textit{FIRST} set for what follows $T$ meaning it is the same case as above so we add the following constraint
$$ \{R\} \subseteq \{T\} $$
For the last production $R \rightarrow \varepsilon$ there are no nonterminals, so no constraints are added.\\
\\
Now we need to solve these constraints. For each constraint on the form $terminal \subseteq nonterminal$ we put those into $FOLLOW(nonterminal)$ so we get
\begin{align*}
FOLLOW(S) &= \{\$,\:','\} \\ 
FOLLOW(T) &= \{\:)\:\} \\
FOLLOW(R) &= \emptyset
\end{align*}
Now for each constraint on the form $nonterminal \subseteq nonterminal$ we add the content from the \textit{FOLLOW} set of the first nonterminal to the second until a fixed point is reached and we get
\begin{align*}
FOLLOW(S) &= \{\$,\:',',\:)\} \\
FOLLOW(T) &= \{\:)\:\} \\
FOLLOW(R) &= \{\:)\:\}
\end{align*}
Whereas a fixed point is reached and we have our \textit{FOLLOW} sets.

\subsubsection*{d}
Determine the look-ahead sets for all productions and point out that $G'_{T_{UP}}$ is LL(1).\\
\\
Having calculated \textit{FIRST} and \textit{FOLLOW} we can use the following rules to determine look-ahead (LA) sets.
$$
LA(X\rightarrow \alpha) = \left\{ \begin{array}{rl}
FIRST(\alpha) \cup FOLLOW(X) &\mbox{ if NULLABLE(X)} \\
FIRST(\alpha) &\mbox{ otherwise}
\end{array} \right.
$$
So we get the following look ahead sets (disregarding the added start production) \\
$$LA(S \rightarrow id) = \{id\}$$
Since the right hand side is not nullable.
$$LA(S \rightarrow (\:T\:)) = \{\:(\:\}$$
Since the right hand side is not nullable.
$$LA(T \rightarrow S\:R) = \{id,\:(\}$$
Since the right hand side is not nullable.
$$LA(R \rightarrow \:,\:T) = \{\:,\:\}$$
Since the right hand side is not nullable.
$$LA(R \rightarrow \varepsilon) = \{\:)\:\}$$
Since the right hand side is nullable.\\
\\
Since for each nonterminal $X$ in the grammar, the look ahead sets for each production for this nonterminal are disjoint, it means the grammar is LL(1).

\newpage

\subsection*{Task 2 - Extend the equality operation in Paladim to work on arrays}
\end{document}

