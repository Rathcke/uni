\documentclass[12pt]{article}
\usepackage[dot, autosize, outputdir="dotgraphs/"]{dot2texi}
\usepackage{tikz}
\usetikzlibrary{shapes}
\usepackage[utf8]{inputenc}
\usepackage{amsmath}
\usepackage{mathtools}
\usepackage{amsfonts}
\usepackage{lastpage}
\usepackage{pdfpages}
\usepackage{gauss}
\usepackage{fancyvrb}
\usepackage{fancyhdr}
\usepackage{graphicx}
\pagestyle{fancy}
\fancyfoot[C]{\footnotesize Page \thepage\ of 5}
\DeclareGraphicsExtensions{.pdf,.png,.jpg}
\title{Oversætter}
\author{Nikolaj Dybdahl Rathcke}
\chead{Nikolaj Dybdahl Rathcke (rfq695)}

\begin{document}
\section*{Oversætter - Week 4}

\subsection*{Exercise 1}

\subsubsection*{1a}
Translate expression  $gcd(x+y, y+1) * 2$   to intermediate code. Place the result in  $t0$ and assume that $vtable=[x\rightarrow v, y\rightarrow w], ftable=[gcd\rightarrow \_GCD\_]$.\\
\\
The intermediate code looks like this
\begin{verbatim}
t1 := 1
t2 := v
t3 := w
t4 := t2+t3
t5 := t4+t1
t6 := CALL _GCD_(t4, t5)
t7 := 2
t0 := t6*t2
\end{verbatim} 

\subsubsection*{1b}
Translate the following statement, i.e., Stat in grammar, to intermediate code, and then to MIPS code:
\begin{verbatim}
 while (b != 0) and (a/b != 0) {
     if b < a then { a := a - b }
              else { b := b - a }
 }
\end{verbatim}
where initially $vtable=[a->v, b->w]$\\
\\
In intermediate code:
\begin{verbatim}
t1:= v
t2:= w
LABEL while
    t3 := t2 < 0
    t4 := 0 < t2
    t5 := t3 || t4
    IF t5 THEN bNotZero ELSE whileFalse
    LABEL bNotZero
        t3 := t1 / t2
        t4 := t3 < 0
        t6 := 0 < t3
        t7 := t4 || t6
        IF t5 && t7 THEN whileTrue ELSE whileFalse
        LABEL whileTrue
            t0 := t2 < t1
            IF t0 THEN ifTrue ELSE ifFalse
            LABEL ifTrue
                t1 := t1 - t2
                GOTO while
            LABEL ifFalse
                t2 := t2 - t1
                GOTO while
LABEL whileFalse

\end{verbatim}
And in MIPS code and assuming $a=v_0$ and $b=v_1$	:
\begin{verbatim}
add	    $t1, $v0, $zero
add	    $t2, $v1, $zero

while:
beqz    $t2, end
divu    $t0, $t1, $t2
beqz    $t0, end
slt	    $t0, $t2, $t1
beqz    $t0, ifTrue
sub	    $t2, $t2, $t1
j       while

ifTrue:
sub	    $t1, $t1, $t2
j       while

end:
\end{verbatim}

\subsubsection*{1c}
Make pattern/replacement pairs for each of the following intermediate-language instructions:
\begin{align*}
(i) rd &:= rs = rt\\
(ii) rd &:=\:!r
\end{align*}
For (i) we get (with $rd=a_0$, $rs=a_1$ and $rt=a_2$)
\begin{verbatim}
slt		$t0, $a0, $a1
slt 	$t1, $a1, $a0
or		$t2, $t0, $t1
xori	$a0, $t2, 1
\end{verbatim}
for (ii) we get ($rd=a0$, $r=a1$)
\begin{verbatim}
slt		$t0, $a1, $zero
slt 	$t1, $zero, $a1
or		$t2, $t0, $t1
xori	$a0, $t2, 1
\end{verbatim}

\newpage

\subsection*{Exercise 2}
\subsubsection*{2a}
Translate the call map(f, x) in Paladim (using while loops)
\begin{verbatim}
function map(f : function, x : array of array of int)
var i : int;
    j : int;
begin
    i := 0;
    j := 0;
    while (i < len(0,x)) do
      begin
        while (j < len(1,x)) do
          begin
            x[i][j] := f(x[i][j]);
            j := j +1;
          end;
        i := i + 1;
      end;
end;
\end{verbatim}

\subsubsection*{2b}
Implement the call map(f, x) in Mips. Assume\\\\
vtable is [x $\rightarrow$ regx]  \\
ftable  is [f $\rightarrow$ "\_f\_"] and HP is the heap pointer.\\\\
You may also assume that you have a Mips instruction that calls a function, e.g., Mips.CALL(f, [re]).
\begin{verbatim}
addi    $t1, HP, 16
lw      $t2, 0(HP)
lw      $t3, 4(HP)
mul     $t4, $t2, $t3
sll     $t4, $t4, 2
beq     $t4, $zero, end
add     $t4, $t4, HP

while:
lw      $t5, 0($t0)
Mips.CALL(_f_, $t5)
addi    $t5, $t5, 4
slt     $t0, $t5, $t4
bne     $t0, $zero, while

end:

\end{verbatim}

\subsubsection*{2c}
Which version would be more efficient and Why?\\
\\
There is more branching in the paladim code because of the two while-loops, whereas the in the MIPS code there is only 1 while loop. Therefore the MIPS code is faster.
\end{document}

