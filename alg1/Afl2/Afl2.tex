\documentclass[12pt]{article}
\usepackage[utf8]{inputenc}
\usepackage{amsmath}
\usepackage{mathtools}
\usepackage{amsfonts}
\usepackage{lastpage}
\usepackage{tikz}
\usepackage{pdfpages}
\usepackage{gauss}
\usepackage{fancyvrb}
\usepackage{fancyhdr}
\usepackage{graphicx}
\pagestyle{fancy}
\fancyfoot[C]{\footnotesize Page \thepage\ of 2}
\DeclareGraphicsExtensions{.pdf,.png,.jpg}
\title{MatIntroNat}
\author{Nikolaj Dybdahl Rathcke}
\lhead{Nikolaj Dybdahl Rathcke (rfq695)}

\begin{document}
\section{Anden Frivillige opgave}

\subsection{}
Beregn ordenen af elementet $\bar{4}$ i den multiplikative gruppe $(\mathbb{Z}/9\mathbb{Z})^{\times}$ og angiv et andet element $\bar{a}$ i $(\mathbb{Z}/9\mathbb{Z})^{\times}$ af samme orden som $\bar{4}$.\\
\\
Da det er under multiplikation vil vi se hvor mange gange 4 skal multipliceres med sig selv før den modulo 9 giver identitetselementet. Vi vil egentlig løse ligningen $4^x$ mod $9 = 1$, hvor 1 er identitetselementet.\\
Dette giver os at elementet $\bar{4}$ har ordenen 3.
Ligeledes har elementet $\bar{7}$ ordenen 3, da $7^3$ mod $9 = 1$, mens de forrige potenser af $7$ ikke opfylder ligningen.

\subsection{}
Lad $\psi:\:C_{15}\rightarrow C_{10}$ være en ikke-triviel gruppehomomorfi. Vis, at kernen for $\psi$ har orden $3$, og at billedet for $\psi$ har orden 5. Vis, at der findes en sådan homomorfi.\\
\\
lol

\subsection{}
Lad $G$ være en abelsk gruppe. Vis, at $\{g \in G\:|\:|g|<\infty\}$ er en undergruppe af $G$. Giv et eksempel hvor dette sæt ikke er en undergruppe når $G$ er ikke-abelsk.\\
\\
Vi kalder $\{g \in G\:|\:|g|<\infty\}$ for $H$.\\
Da $H$ er endelig er det nok at se om $H$ er lukket under multiplikation og den ikke er tom.

\end{document}
