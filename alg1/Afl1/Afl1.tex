\documentclass[12pt]{article}
\usepackage[utf8]{inputenc}
\usepackage{amsmath}
\usepackage{mathtools}
\usepackage{amsfonts}
\usepackage{lastpage}
\usepackage{tikz}
\usepackage{pdfpages}
\usepackage{gauss}
\usepackage{fancyvrb}
\usepackage{fancyhdr}
\usepackage{graphicx}
\pagestyle{fancy}
\fancyfoot[C]{\footnotesize Page \thepage\ of 2}
\DeclareGraphicsExtensions{.pdf,.png,.jpg}
\title{MatIntroNat}
\author{Nikolaj Dybdahl Rathcke}
\lhead{Nikolaj Dybdahl Rathcke (rfq695)}

\begin{document}
\section{Første Frivillige opgave}

\subsection{}
Hvilke tal forekommer som ordener af elementer i gruppen $\mathbb{Z}_{12}=\mathbb{Z}/12\mathbb{Z}$? Samme spørgsmål for den symmetriske grupper $S_3$.\\
\\
Da det er en gruppe under addition skal vi finde det (først) heltal multiplikation af elementet der mod 12 giver 0. Vi opstiller følgende tabel\\
\begin{center}
\begin{tabular}{c|c|c|c|c|c|c|c|c|c|c|c|c}
Element & $\bar{0}$ & $\bar{1}$ & $\bar{2}$ & $\bar{3}$ & $\bar{4}$ & $\bar{5}$ & $\bar{6}$ & $\bar{7}$ & $\bar{8}$ & $\bar{9}$ & $\bar{10}$ & $\bar{0}$ \\ 
\hline 
Orden & 1 & 12 & 6 & 4 & 3 & 12 & 2 & 12 & 3 & 4 & 6 & 12 \\ 
\end{tabular}
\end{center}
Altså indgår tallene $\{1,2,3,4,6,12\}$ som orderner af element i gruppen $\mathbb{Z}_{12}=\mathbb{Z}/12\mathbb{Z}$\\
\\
For den symmetriske gruppe $S_3$ er ordenen for et element lig det mindste fælles multiplum af længderne af cyklerne der fås ved cykel dekomposition. Altså beregner vi først cykel dekompositionerne af elementerne i $S_3$ ved brug af algoritmen s. 30 i Dummit and Foote.\\
\begin{center}
\begin{tabular}{|c|c|}
\hline 
Værdier af $\sigma_i$ & Cykel dekomposition af $\sigma_i$ \\ 
\hline 
$\sigma_1(1)=1,\sigma_1(2)=2,\sigma_1(3)=3$  & (1)(2)(3)$=$1 \\  
$\sigma_2(1)=1,\sigma_2(2)=3,\sigma_2(3)=2$  & (1)(2 3)$=$(2 3) \\ 
$\sigma_3(1)=2,\sigma_3(2)=1,\sigma_3(3)=3$  & (1 2)(3)$=$(1 2) \\ 
$\sigma_4(1)=2,\sigma_4(2)=3,\sigma_4(3)=1$  & (1 2 3) \\ 
$\sigma_5(1)=3,\sigma_5(2)=1,\sigma_5(3)=2$  & (1 3 2) \\ 
$\sigma_6(1)=3,\sigma_6(2)=2,\sigma_6(3)=1$  & (1 3)(2)$=$(1 3) \\ 
\hline
\end{tabular} 
\end{center}
Hvor længderne og LCM(ordenerne) er henholdsvis
\begin{align*}
&\sigma_1=\{1,1,1\}, &LCM = 1 \\
&\sigma_2=\{2\}, &LCM = 2 \\
&\sigma_3=\{2,1\}, &LCM = 2 \\
&\sigma_4=\{3,1\}, &LCM = 3 \\
&\sigma_5=\{3,1\}, &LCM = 3 \\
&\sigma_6=\{2,1\}, &LCM = 2
\end{align*}
Altså indgår tallene $\{1,2,3\}$ som orderner i gruppen $S_3$.

\newpage 
\subsection{}
Opg 18, [DF], side 40: \\
Let $G$ be any group. Prove that the map from $G$ to itself defined by $\psi: g\rightarrow g^2$ is a homomorphism if and only if $G$ is abelian.\\
\\
Hvis det er en homomorfi, betyder det at for alle $a,b \in G$ gælder $\psi(ab)=\psi(a)\psi(b)$. Vi vil vise det begge veje, at hvis det er en homomorfi er den abelsk og hvis den er abelsk er det en homomorfi. Vi starter med at antage $\psi$ er en homomorfi. Derved får vi
\begin{align*}
\psi(ab)&=\psi(a)\psi(b) \\
(ab)^2&=a^2b^2 \\
(ab)(ab)&=aabb \\
ababb^{-1}&=aabbb^{-1} && b \text{ har et invers element vi kan gange på på begge sider} \\
aba&=aab  && \text{der gælder at } b*b^{-1}=1 \\
a^{-1}aba&=a^{-1}aab  && a \text{ har et invers element vi kan gange på på begge sider} \\
ba&=ab && \text{der gælder at } a*a^{-1}=1
\end{align*}
Hvilket viser G er en abelsk gruppe. Nu antages at G er abelsk og vi vil vise $\psi$ er en homomorfi. Vi har, at
\begin{align*}
\psi(ab)&=(ab)(ab) \\
&=a^2b^2 && \text{Da G er abelsk} \\
&=\psi(a)\psi(b)
\end{align*}
Altså må $\psi$ være en homomorfi.

\end{document}
