\documentclass[a4paper, fleqn]{article}
\usepackage[utf8]{inputenc}
\usepackage{amsmath}
\usepackage{amssymb}
\usepackage{caption}
\usepackage{mathtools}
\usepackage{amsfonts}
\usepackage{lastpage}
\usepackage{tikz}
\usepackage{float}
\usepackage{textcomp}
\usetikzlibrary{patterns}
\usepackage{pdfpages}
\usepackage{gauss}
\usepackage{fancyvrb}
\usepackage[table]{colortbl}
\usepackage{fancyhdr}
\usepackage{graphicx}
\usepackage[margin=2.5 cm]{geometry}

\setlength\parindent{0pt}
\setlength\mathindent{75pt}

\definecolor{listinggray}{gray}{0.9}
\usepackage{listings}
\lstset{
	language=,
	literate=
		{æ}{{\ae}}1
		{ø}{{\o}}1
		{å}{{\aa}}1
		{Æ}{{\AE}}1
		{Ø}{{\O}}1
		{Å}{{\AA}}1,
	backgroundcolor=\color{listinggray},
	tabsize=3,
	rulecolor=,
	basicstyle=\scriptsize,
	upquote=true,
	aboveskip={0.2\baselineskip},
	columns=fixed,
	showstringspaces=false,
	extendedchars=true,
	breaklines=true,
	prebreak =\raisebox{0ex}[0ex][0ex]{\ensuremath{\hookleftarrow}},
	frame=single,
	showtabs=false,
	showspaces=false,
	showlines=true,
	showstringspaces=false,
	identifierstyle=\ttfamily,
	keywordstyle=\color[rgb]{0,0,1},
	commentstyle=\color[rgb]{0.133,0.545,0.133},
	stringstyle=\color[rgb]{0.627,0.126,0.941},
  moredelim=**[is][\color{blue}]{@}{@},
}

\lstdefinestyle{base}{
  emptylines=1,
  breaklines=true,
  basicstyle=\ttfamily\color{black},
}

\pagestyle{fancy}
\def\checkmark{\tikz\fill[scale=0.4](0,.35) -- (.25,0) -- (1,.7) -- (.25,.15) -- cycle;}
\newcommand*\circled[1]{\tikz[baseline=(char.base)]{
            \node[shape=circle,draw,inner sep=2pt] (char) {#1};}}
\newcommand*\squared[1]{%
  \tikz[baseline=(R.base)]\node[draw,rectangle,inner sep=0.5pt](R) {#1};\!}
\newcommand{\comment}[1]{%
  \text{\phantom{(#1)}} \tag{#1}}
\def\el{[\![}
\def\er{]\!]}
\def\dpip{|\!|}
\def\MeanN{\frac{1}{N}\sum^N_{n=1}}
\cfoot{Page \thepage\ of \pageref{LastPage}}
\DeclareGraphicsExtensions{.pdf,.png,.jpg}
\author{Nikolaj Dybdahl Rathcke (Student ID: 74763954)}
\title{Cryptography and Coding Theory \\ Assignment 1}
\lhead{Cryptography and Coding Theory}
\rhead{Assignment 1}

\begin{document}
\maketitle

\section{Question 1}
\subsection{(i)}
TODO

\subsection{(ii)}
TODO

\subsection{(iii)}
TODO

\subsection{(iv)}
TODO

\section{Question 2}

\subsection{(i)}
The prime $p$ is the number $503$. There are several ways to check if a number is prime. One well-known way is to see if a number $n$ is prime is by checking if any of the primes less than the square root of $n$ divides $n$. We could also use a primality test such as miller-rabin, which works well in this case as it is correct if it outputs a $n$ is composite, but could be wrong if it outputs that $n$ is prime (I was so lucky to have this lying around). There are also deterministic versions of primality tests. \\
Choosing a cyclic group $(Z/pZ)^*$ for a prime $p$ is important as it means $\phi(p)=p-1$, i.e. that all members in the group are coprime with $p$. Thus, when we pick the generator $g$, we ensure that the final secret number can any value from $1$ to $p-1$.

\subsection{(ii)}
An algorithm\footnote{Algorithm 2.5.16, Elementary Number Theory - Primes, Congruences and Secrets by W. Stein in 2011} I found which finds the smallest element $g$ that generates $(Z/pZ)^*$ gives the following approach:
\begin{enumerate}
  \item  $[p = 2?]$ If $p=2$ output $1$ and terminate. Otherwise set $a=2$.
  \item  $[\mbox{Prime Divisors}]$ Compute the prime divisors $p_1,\ldots, p_r$ of $p-1$.
  \item  $[\mbox{Generator?}]$ If for every $p_i$, we have $a^{(p-1)/p_i} \not\equiv 1$ (mod $p$), then $a$ is a generator of $(Z/pZ)^*$, so output $a$ and terminate.
  \item  $[\mbox{Try next}]$ Set $a=a+1$ and go to Step 3.
\end{enumerate}
The prime factors, $p_i$ of $p-1=502$ is $2$ and $251$. This means there are $\varphi(502)=502(1-\frac{1}{2})(1-\frac{1}{251}=250$ primtive roots, so we should hopefully not need too many iterations of the algorithm. Table \ref{tab1} shows the calculations made in step $3$ of the algorithm:
\begin{table}
  \centering
  \begin{tabular}{|c|c|c|}
  \hline
  $a$ & $a^{502/2}$ mod $503$ & $a^{502/251}$ mod $503$ \\
  \hline
  $2$ & $1$ & \\
  \hline
  $3$ & $1$ & \\
  \hline
  $4$ & $1$ & \\
  \hline
  $5$ & $502$ & $25$ \\
  \hline
  \end{tabular}
  \caption{Table showing the calculation made in step $3$ of the above algorithm.}
  \label{tab1}
\end{table}
And the algorithm terminates. The number $5$ is a generator for the cyclic group.

\subsection{(iii)}
Let us randomly choose $a=41$, meaning we get $5^{41} \equiv 304$ mod $503$. So we have $A=304$ and our public key is $(503, 5, 304)$.

\section{Question 3}

\subsection{(i)}
TODO

\subsection{(ii)}
TODO

\section{Question 4}
TODO


\end{document}
