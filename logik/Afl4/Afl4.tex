\documentclass[12pt]{article}
\usepackage[utf8]{inputenc}
\usepackage[fleqn]{amsmath}
\usepackage{amssymb}
\usepackage[fleqn]{mathtools}
\usepackage{amsfonts}
\usepackage{lastpage}
\usepackage{tikz}
\usepackage{pdfpages}
\usepackage{gauss}
\usepackage{fancyvrb}
\usepackage{fancyhdr}
\usepackage{graphicx}
\usepackage{boxproof}
\usepackage{a4wide}
\usepackage{daymonthyear}

\def\meta#1{\mbox{$\langle\hbox{#1}\rangle$}}
\def\macrowitharg#1#2{{\tt\string#1\bra\meta{#2}\ket}}

{\escapechar-1 \xdef\bra{\string\{}\xdef\ket{\string\}}}

\def\intro#1{{#1}{\cal I}}
\def\elim#1{{#1}{\cal E}}

\showboxbreadth 999
\showboxdepth 999
\tracingoutput 1


\let\imp\to
\def\elim#1{{{#1}{\cal E}}}
\def\intro#1{{{#1}{\cal I}}}
\def\lt{<}
\def\eqdef{=}
\def\eps{\mathrel{\epsilon}}
\def\biimplies{\leftrightarrow}
\def\flt#1{\mathrel{{#1}^\flat}}
\def\setof#1{{\left\{{#1}\right\}}}
\let\implies\to
\def\KK{{\mathsf K}}
\let\squashmuskip\relax
\pagestyle{fancy}
\fancyfoot[C]{\footnotesize Page \thepage\ of 6}
\DeclareGraphicsExtensions{.pdf,.png,.jpg}
\author{Nikolaj Dybdahl Rathcke}
\chead{Nikolaj Dybdahl Rathcke (rfq695)}

\begin{document}
\section*{Logic in Computer Science - Assignment 4}
\subsection*{Exercise 3.2.2}
\subsubsection*{a}
We have a formula $\phi=G\:a$.
\subsubsection*{i}
We find the following path that satisfies $\phi$
$$q_3\to q_4\to q_3\to q_4\to ...$$
\subsubsection*{ii}
It does \textit{\textbf{not}} hold for all paths as
$$q_3\to q_2\to q_2\to ...$$
does not contain $a$ in every node.

\subsubsection*{b}
We have a formula $\phi=a\:U\:b$.
\subsubsection*{i}
The following path satisfies $\phi$
$$q_3\to q_2\to q_2\to ...$$
\subsubsection*{ii}
It does \textit{\textbf{not}} as we can pick a path 
$$q_3\to q_1\to q_2\to q_2\to ...$$
where $a$ does not hold in $q_1$ and $b$ does not hold either.

\subsubsection*{c}
We have a formula $\phi=a\:U\:X\:(a\land\neg b)$.
\subsubsection*{i}
The following path satisfies $\phi$
$$q_3\to q_4\to q_3\to q_4\to ...$$
\subsubsection*{ii}
It does \textit{\textbf{not}} hold as we can go to either $q_1$ or $q_2$ and the path will never contain $X(a\land \neg b)$. 

\subsubsection*{d}
We have a formula $\phi=X\:\neg b \land G(\neg a\lor\neg b)$.
\subsubsection*{i}
The following path satisfies $\phi$
$$q_3\to q_1\to q_2\to q_2\to ...$$
\subsubsection*{ii}
It does \textit{\textbf{not}} hold as we can pick the path
$$q_3\to q_2\to q_2\to ...$$
which will never contain $X\:\neg b$.

\subsubsection*{e}
We have a formula $\phi=X(a\land b)\land F(\neg a\land\neg b)$.
\subsubsection*{i}
The following path satisfies $\phi$
$$q_3\to q_4\to q_3\to q_1\to q_2\to q_2\to ...$$
\subsubsection*{ii}
It does \textit{\textbf{not}} hold as we can pick a path
$$q_3\to q_2\to q_2\to ...$$
And it will neither contain $X(a\land b)$ or $F(\neg a\land\neg b)$ and both are required.
\newpage
\subsection*{Exercise 2.3.7}
To prove the equivalence $\phi\:W\:\psi\land F\:\psi\equiv\phi\:U\:\psi$ we suppose there exists a path that satisfies $\phi\:W\:\psi\land F\:\psi$. Then, from clause 5, we know that 
\begin{align}
&\pi\models F\:\psi \\
&\pi\models\phi\:W\:\psi
\end{align}
Using (1) with clause 10, we have that \textit{for some $i\geq 1$ that $\pi^i\models\psi$}.\\
Since the above applies, this means using (2) with clause 12 we have that \textit{for some $i\geq 1$ we have $\pi^i\models\psi$ and for $j=1,..,i-1$ we have $\pi^j\models\phi$}.\\
Combining what we have found above, we can use clause 11 to show $\pi\models\phi\:U\:\psi$ as the two conditions needed is what we have found above.
\newpage
\subsection*{PCP Exercise}
\subsubsection*{Exercise 1}
ProofWeb has been used for this task. The code can be seen below and it is also uploaded separately in a \texttt{.txt} file for raw text.
\begin{Verbatim}[fontsize=\small]
Require Import ProofWeb.

Variable P : D * D -> Prop.
Variable f0 f1 : D -> D.
Variable e : D.

Theorem PCP_ex_simp_b : 
(P(f1 e,f1(f0(f1 e))) /\
 P(f0(f1 e), f0(f0 e)) /\
 P(f1(f1(f0 e)),f1(f1 e)))
->
((all v, all w, (P(v,w) -> P(f1 v,f1(f0(f1 w))))) /\
 (all v, all w, (P(v,w) -> P(f0(f1 v), f0(f0 w)))) /\
 (all v, all w, (P(v,w) -> P(f1(f1(f0 v)),f1(f1 w)))))
->
exi z, P(z,z).

Proof.
imp_i H1.
imp_i H2.
insert C1 ((all v, all w, (P(v,w) -> P(f0(f1 v), f0(f0 w)))) /\ 
            (all v, all w, (P(v,w) -> P(f1(f1(f0 v)),f1(f1 w))))).
f_con_e2 H2.
exi_i (f1(f1(f0(f0(f1(f1(f1(f0(f1 e))))))))).
imp_e (P(f0(f1(f1(f1(f0(f1 e))))), (f0(f0(f1(f1(f1(f0(f1 e))))))))).
all_e (all w, (P(f0(f1(f1(f1(f0(f1 e))))), w) -> 
                P(f1(f1(f0(f0(f1(f1(f1(f0(f1 e)))))))), f1(f1 w)))).
all_e (all v, all w, (P(v,w) -> P(f1(f1(f0 v)),f1(f1 w)))).
f_con_e2 C1.
imp_e (P((f1(f1(f0(f1 e)))), (f1(f1(f1(f0(f1 e))))))).
all_e (all w, (P((f1(f1(f0(f1 e)))), w) -> P(f0(f1(f1(f1(f0(f1 e))))), f0(f0 w)))).
all_e (all v, all w, (P(v,w) -> P(f0(f1 v), f0(f0 w)))).
f_con_e1 C1.
imp_e (P((f1 e), (f1(f0(f1 e))))).
all_e (all w, (P((f1 e), w) -> P (f1(f1(f0((f1 e)))), f1(f1 w)))).
all_e (all v, all w, (P(v,w) -> P(f1(f1(f0 v)),f1(f1 w)))).
f_con_e2 C1.
f_con_e1 H1.

Qed.
\end{Verbatim}

\subsubsection*{Exercise 2}
We want to find the predicate logic for the PCP instance
$$((001,0),(01,011),(01,101),(10,001))$$
This means we write,from [H\&R, page 134], $\psi=\phi_1\land\phi_2\to \phi_3$ where we find $\phi_1$ to
\begin{align*}
\phi_1 &\stackrel{def}{=}\bigwedge_{i=1}^{k} P(f_{s_i}(e),f_{t_i}(e)) \\
&=(P(f_{001}(e),f_0(e))\land P(f_{01}(e),f_{011}(e))\land P(f_{01}(e),f_{101}(e))\land P(f_{10}(e),f_{001}(e)))
\end{align*}
And $\phi_2$ is found to be
\begin{align*}
\phi_2 \stackrel{def}{=} \:&\forall v\forall w(P(v,w)\to \bigwedge_{i=1}^{k} P(f_{s_i}(v),f_{t_i}(w))) \\
=\:&\forall v\forall w(P(v,w)\to (P(f_{001}(v),f_0(w))\land P(f_{01}(v),f_{011}(w))\land P(f_{01}(v),f_{101}(w))\\
&\phantom{=} \land P(f_{10}(v),f_{001}(w)))
\end{align*}
And lastly $\phi_3$ is defined to be
\begin{align*}
\phi_3 &\stackrel{def}{=} \exists zP(z,z)
\end{align*}
Which defines our predicate logic formula $\psi$.


\end{document}