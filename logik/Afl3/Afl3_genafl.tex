\documentclass[12pt]{article}
\usepackage[utf8]{inputenc}
\usepackage{amsmath}
\usepackage{amssymb}
\usepackage{mathtools}
\usepackage{amsfonts}
\usepackage{lastpage}
\usepackage{tikz}
\usepackage{pdfpages}
\usepackage{gauss}
\usepackage{fancyvrb}
\usepackage{fancyhdr}
\usepackage{graphicx}
\usepackage{boxproof}
\usepackage{a4wide}
\usepackage{daymonthyear}

\def\meta#1{\mbox{$\langle\hbox{#1}\rangle$}}
\def\macrowitharg#1#2{{\tt\string#1\bra\meta{#2}\ket}}

{\escapechar-1 \xdef\bra{\string\{}\xdef\ket{\string\}}}

\def\intro#1{{#1}{\cal I}}
\def\elim#1{{#1}{\cal E}}

\showboxbreadth 999
\showboxdepth 999
\tracingoutput 1


\let\imp\to
\def\elim#1{{{#1}{\cal E}}}
\def\intro#1{{{#1}{\cal I}}}
\def\lt{<}
\def\eqdef{=}
\def\eps{\mathrel{\epsilon}}
\def\biimplies{\leftrightarrow}
\def\flt#1{\mathrel{{#1}^\flat}}
\def\setof#1{{\left\{{#1}\right\}}}
\let\implies\to
\def\KK{{\mathsf K}}
\let\squashmuskip\relax
\pagestyle{fancy}
\fancyfoot[C]{\footnotesize Page \thepage\ of 2}
\DeclareGraphicsExtensions{.pdf,.png,.jpg}
\title{Elementær Talteori}
\author{Nikolaj Dybdahl Rathcke}
\chead{Nikolaj Dybdahl Rathcke (rfq695)}

\begin{document}
\section*{Logic in Computer Science - Assignment 3}

\subsection*{Structural Induction}
\subsubsection*{a}
We want to show, by structural induction, the following hypothesis.\\
\textbf{Inductive hypothesis}: We claim that any tree with height $h$ has at most $2^h$ propositional atoms (leaves).\\
\textbf{Base case}: We see that if $\phi=\bot$ or $\phi=p$ then our height is $0$ and contains $1$ propositional atom. This holds to be less or equal to $2^h$, as $2^0=1$ and $1\leq 2^0$.\\
\textbf{Inductive step}: We let a $T$ be a tree of height $k+1$.\\
Since any propositional logic of form $\phi\oplus\psi$ has two children, we have two subtrees of height $k$. By out inductive hypothesis, we know both of these trees have at most $2^k$ leaves, and as such, the amount of leaves in $T$ is equal to the number of leaves in each subtree. This amount is guaranteed to be less than or equal to $2^k+2^k=2^{k+1}$.\\
For the case where $\phi=\neg\phi$, we have only one subtree of height $k$. The maximum amount of leaves for this subtree is $2^k$.\\
As this is strictly less than $2^{k+1}$, we can conclude that the sum of leaves of both subtrees is at most $2^{k+1}$, thus it holds for $k+1$ and the hypothesis is proved.

\subsubsection*{b}
We want to show the following using structural induction.\\
\textbf{Induction hypothesis}: Any propositional logic of height $h$ contains strictly less than $2^h$ propositional atoms if the logic contains a $\neg$ anywhere. \\
\textbf{Base case}: Our base is that $\neg p$ or $\neg\bot$ has a height of $1$ and $p$ adds $0$ to the height. It has $1$ propositional atom, which is strictly less than $2^1=2$.\\
\textbf{Inductive step}: Lets assume we have a subtree $T$ with height $h+1$. We have two subtrees of height $h$ and we assume one of these subtrees have a negation in it.\\
Let the subtree containing a negation be $T_2$ with height $h$. As the negation has only one child with height $h-1$, we know the subtree $T_2$ has less than or equal to $2^{h-1}$ (using proof from (a)).\\
For the other subtree, $T_3$ of also at most height $h$, we can also use (a) to show that it has at most $2^{h}$.\\
Now let $l(T)$ be the number of leaves in $T$, we want to show
\begin{align*}
l(T)>l(T_2)+l(T_3)
\end{align*}
We know from (a) that $l(T)\leq 2^{h+1}$, $l(T_3)\leq 2^h$ and $l(T_2)\leq 2^{h-1}$, so we can rewrite the above to a worst case
\begin{align*}
2^{h+1}>2^{h-1}+2^h
\end{align*}
or
\begin{align*}
2^h>2^{h-1}
\end{align*}
By using $2^{h+1}=2^h+2^h$ and eliminating $2^h$ on both sides. This is seen to be true. Had our $T_3$ had a negation we would have
\begin{align*}
2^{h+1}>2^{h-1}+2^{h-1}
\end{align*}
Which is obviously true. \\
Thus, we can prove our induction hypothesis.

\end{document}