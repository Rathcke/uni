\documentclass[12pt]{article}
\usepackage[utf8]{inputenc}
\usepackage{amsmath}
\usepackage{amssymb}
\usepackage{mathtools}
\usepackage{amsfonts}
\usepackage{lastpage}
\usepackage{tikz}
\usepackage{pdfpages}
\usepackage{gauss}
\usepackage{fancyvrb}
\usepackage{fancyhdr}
\usepackage{graphicx}
\usepackage{boxproof}
\usepackage{a4wide}
\usepackage{daymonthyear}

\def\meta#1{\mbox{$\langle\hbox{#1}\rangle$}}
\def\macrowitharg#1#2{{\tt\string#1\bra\meta{#2}\ket}}

{\escapechar-1 \xdef\bra{\string\{}\xdef\ket{\string\}}}

\def\intro#1{{#1}{\cal I}}
\def\elim#1{{#1}{\cal E}}

\showboxbreadth 999
\showboxdepth 999
\tracingoutput 1


\let\imp\to
\def\elim#1{{{#1}{\cal E}}}
\def\intro#1{{{#1}{\cal I}}}
\def\lt{<}
\def\eqdef{=}
\def\eps{\mathrel{\epsilon}}
\def\biimplies{\leftrightarrow}
\def\flt#1{\mathrel{{#1}^\flat}}
\def\setof#1{{\left\{{#1}\right\}}}
\let\implies\to
\def\KK{{\mathsf K}}
\let\squashmuskip\relax
\pagestyle{fancy}
\fancyfoot[C]{\footnotesize Page \thepage\ of 6}
\DeclareGraphicsExtensions{.pdf,.png,.jpg}
\title{Elementær Talteori}
\author{Nikolaj Dybdahl Rathcke}
\chead{Nikolaj Dybdahl Rathcke (rfq695)}

\begin{document}

\subsection*{Completeness exercise}
We want to prove completeness of
$$(p\to q)\to q \vdash (q\to p)\to p$$
We follow the steps described in the assignment text.\\
We start with
$$(p\to q)\to q \models (q\to p)\to p$$
Now we move the premises to the right hand side, so we get
$$\models ((p\to q)\to q)\to((q\to p)\to p)$$
We proceed by creating a truth table
\begin{center}
\begin{tabular}{|c|c||c|c|c|c||c|}
\hline 
$p$ & $q$ & $p\to q$ & $(p\to q)\to q$ & $q\to p$ & $(q\to p)\to p$ & $((p\to q)\to q)\to((q\to p)\to p)$ \\ 
\hline 
T & T & T & T & T & T & T \\ 
\hline 
T & F & F & T & T & T & T \\ 
\hline 
F & T & T & T & F & T & T \\ 
\hline 
F & F & T & F & T & F & T \\ 
\hline 
\end{tabular} 
\end{center}
This means we have the following constructed sequents
\begin{align*}
p,q&\vdash((p\to q)\to q)\to((q\to p)\to p) &:\alpha_1\\
p,\neg q&\vdash((p\to q)\to q)\to((q\to p)\to p) &:\alpha_2\\
\neg p,q&\vdash((p\to q)\to q)\to((q\to p)\to p) &:\alpha_3\\
\neg p,\neg q&\vdash((p\to q)\to q)\to((q\to p)\to p)&:\alpha_4 \\
\end{align*}
As explained in the assignment text, we start by proving $\alpha_1$, meaning we prove
\begin{align*}
p&\vdash \gamma_{\textit{l}}[p] &(a) \\
q&\vdash \gamma_{\textit{l}}[q] &(b)\\
p,q&\vdash \gamma_{\textit{l}}[p\to q]&(c)\\
p,q&\vdash \gamma_{\textit{l}}[(p\to q)\to q]&(d)\\
p,q&\vdash \gamma_{\textit{l}}[q\to p]&(e)\\
p,q&\vdash \gamma_{\textit{l}}[(q\to p)\to p]&(f)\\
p,q&\vdash \gamma_{\textit{l}}[((p\to q)\to q)\to((q\to p)\to p)]&(g)
\end{align*}
In (a), we prove $p\vdash\gamma_{\textit{l}}[p]$ where $\gamma_{\textit{l}}[p]=p$ meaning it is trivially true.\\
In (b), we prove $q\vdash\gamma_{\textit{l}}[q]$ where $\gamma_{\textit{l}}[q]=q$ meaning it is trivially true.\\
In (c), we prove $p,q\vdash\gamma_{\textit{l}}[p\to q]$ where $\gamma_{\textit{l}}[p\to q]=p\to q$. A box proof is provided below
\begin{proofbox}
   \: p 	 \=\mbox{premise}\\
   \: q \= \mbox{premise}
   \[
      \: p		  \=\mbox{assumption}\\
      \: q \=   (2)
   \]
   \: p\to q \= \to I(3-4)
\end{proofbox}
\\
In (d), we prove $p,q\vdash\gamma_{\textit{l}}[(p\to q)\to q]$ where $\gamma_{\textit{l}}[(p\to q)\to q]=(p\to q)\to q$. A box proof is provided below
\begin{proofbox}
   \: p 	 \=\mbox{premise}\\
   \: q \= \mbox{premise}
   \[
     \: p\to q		  \=\mbox{assumption} \\
     \: q \= \to E(1,3)
   \]
   \: (p\to q)\to q \= \to I(3-4)
\end{proofbox}
\\
In (e), we prove $p,q\vdash\gamma_{\textit{l}}[q\to p]$ where $\gamma_{\textit{l}}[q\to p]=q\to p$. This is the exact same proof as in (c), so we reuse this proof with the atoms $p$ and $q$ switched.\\
In (f), we prove $p,q\vdash\gamma_{\textit{l}}[(q\to p)\to p]$ where $\gamma_{\textit{l}}[(q\to p)\to p]=(q\to p)\to p$. This is the exact same proof as in (d), so we reuse this proof with the atoms $p$ and $q$ switched.\\
In (g), we prove $p,q\vdash\gamma_{\textit{l}}[((p\to q)\to q)\to((q\to p)\to p)]$ where $\gamma_{\textit{l}}[((p\to q)\to q)\to((q\to p)\to p)]=((p\to q)\to q)\to((q\to p)\to p)$. A boxproof is provided below
\begin{proofbox}
   \: p 	 \=\mbox{premise}\\
   \: q \= \mbox{premise}\\
   \: (p\to q)\to q \= \mbox{proved in (d)}\\
   \: (q\to p)\to p \= \mbox{proved in (f)}
   \[
     \: (p\to q)\to q		  \=\mbox{assumption} \\
     \: (q\to p)\to p \= (4)
   \]
   \: ((p\to q)\to q)\to ((q\to p)\to p) \= \to I(5-6)
\end{proofbox}
We now prove $\alpha_2$ in the same manner.
\begin{align*}
p&\vdash \gamma_{\textit{l}}[p] &(a) \\
\neg q&\vdash \gamma_{\textit{l}}[q] &(b)\\
p,\neg q&\vdash \gamma_{\textit{l}}[p\to q]&(c)\\
p,\neg q&\vdash \gamma_{\textit{l}}[(p\to q)\to q]&(d)\\
p,\neg q&\vdash \gamma_{\textit{l}}[q\to p]&(e)\\
p,\neg q&\vdash \gamma_{\textit{l}}[(q\to p)\to p]&(f)\\
p,\neg q&\vdash \gamma_{\textit{l}}[((p\to q)\to q)\to((q\to p)\to p)]&(g)
\end{align*}
In (a), we prove $p\vdash\gamma_{\textit{l}}[p]$ where $\gamma_{\textit{l}}[p]=p$ meaning it is trivially true.\\
In (b), we prove $\neg q\vdash\gamma_{\textit{l}}[q]$ where $\gamma_{\textit{l}}[q]=\neg q$ meaning it is trivially true.\\
In (c), we prove $p,\neg q\vdash\gamma_{\textit{l}}[p\to q]$ where $\gamma_{\textit{l}}[p\to q]=\neg(p\to q)$. A box proof is provided below
\begin{proofbox}
   \: p 	 \=\mbox{premise}\\
   \: \neg q \= \mbox{premise}
   \[
      \: (p\to q)		  \=\mbox{assumption}\\
      \: q \= \to E(1,3) \\
      \: \bot \= \neg E(4,2) 
   \]
   \: \neg(p\to q) \= \neg I(3-5)
\end{proofbox}
\\
In (d), we prove $p,\neg q\vdash\gamma_{\textit{l}}[(p\to q)\to q]$ where $\gamma_{\textit{l}}[(p\to q)\to q]=(p\to q)\to q$. A box proof is provided below
\begin{proofbox}
   \: p 	 \=\mbox{premise}\\
   \: \neg q \= \mbox{premise}
   \[
     \: p\to q		  \=\mbox{assumption} \\
     \: q \= \to E(1,3)
   \]
   \: (p\to q)\to q \= \to I(3-4)
\end{proofbox}
\\
In (e), we prove $p,\neg q\vdash\gamma_{\textit{l}}[q\to p]$ where $\gamma_{\textit{l}}[q\to p]=q\to p$. A box proof is provided below
\begin{proofbox}
   \: p 	 \=\mbox{premise}\\
   \: \neg q \= \mbox{premise}
   \[
     \: q		  \=\mbox{assumption} \\
     \: \bot \= \neg E(3,2)
     \: p
   \]
   \: q\to p \= \to I(3-5)
\end{proofbox}
\\
In (f), we prove $p,q\vdash\gamma_{\textit{l}}[(q\to p)\to p]$ where $\gamma_{\textit{l}}[(q\to p)\to p]=(q\to p)\to p$. A box proof is provided below
\begin{proofbox}
   \: p 	 \=\mbox{premise}\\
   \: \neg q \= \mbox{premise}
   \[
     \: (q\to p)		  \=\mbox{assumption} \\
     \: p \= (1)
   \]
   \: (q\to p)\to p \= \to I(3-4)
\end{proofbox}
\\
In (g), we prove $p,\neg q\vdash\gamma_{\textit{l}}[((p\to q)\to q)\to((q\to p)\to p)]$ where $\gamma_{\textit{l}}[((p\to q)\to q)\to((q\to p)\to p)]=((p\to q)\to q)\to((q\to p)\to p)$. A boxproof is provided below
\begin{proofbox}
   \: p 	 \=\mbox{premise}\\
   \: \neg q \= \mbox{premise}\\
   \: (p\to q)\to q \= \mbox{proved in (d)} \\
   \: (q\to p)\to p \= \mbox{proved in (f)} \\
   \[
     \: (p\to q)\to q		  \=\mbox{assumption} \\
     \: (q\to p)\to p \= (4)
   \]
   \: ((p\to q)\to q)\to((q\to p)\to p) \= \to I(5-6)
\end{proofbox}
We now prove $\alpha_3$ in the same manner.
\begin{align*}
\neg p&\vdash \gamma_{\textit{l}}[p] &(a) \\
q&\vdash \gamma_{\textit{l}}[q] &(b)\\
\neg p,q&\vdash \gamma_{\textit{l}}[p\to q]&(c)\\
\neg p,q&\vdash \gamma_{\textit{l}}[(p\to q)\to q]&(d)\\
\neg p,q&\vdash \gamma_{\textit{l}}[q\to p]&(e)\\
\neg p,q&\vdash \gamma_{\textit{l}}[(q\to p)\to p]&(f)\\
\neg p,q&\vdash \gamma_{\textit{l}}[((p\to q)\to q)\to((q\to p)\to p)]&(g)
\end{align*}
In (a), we prove $\neg p\vdash\gamma_{\textit{l}}[p]$ where $\gamma_{\textit{l}}[p]=\neg p$ meaning it is trivially true.\\
In (b), we prove $q\vdash\gamma_{\textit{l}}[q]$ where $\gamma_{\textit{l}}[q]=q$ meaning it is trivially true.\\
In (c), we prove $\neg p,q\vdash\gamma_{\textit{l}}[p\to q]$ where $\gamma_{\textit{l}}[p\to q]=p\to q$. A box proof is provided below
\begin{proofbox}
   \: \neg p 	 \=\mbox{premise}\\
   \: q \= \mbox{premise}
   \[
      \: p		  \=\mbox{assumption}\\
      \: \bot \= \neg E(3,1) \\
      \: q \= \bot E(4)
   \]
   \: p\to q \= \to I(3-5)
\end{proofbox}
\\
In (d), we prove $\neg p,q\vdash\gamma_{\textit{l}}[(p\to q)\to q]$ where $\gamma_{\textit{l}}[(p\to q)\to q]=(p\to q)\to q$. A box proof is provided below
\begin{proofbox}
   \: \neg p 	 \=\mbox{premise}\\
   \: q \= \mbox{premise}
   \[
     \: p\to q		  \=\mbox{assumption} \\
     \: q \= (2)
   \]
   \: (p\to q)\to q \= \to I(3-4)
\end{proofbox}
\\
In (e), we prove $\neg p,q\vdash\gamma_{\textit{l}}[q\to p]$ where $\gamma_{\textit{l}}[q\to p]=\neg(q\to p)$. A box proof is provided below
\begin{proofbox}
   \: \neg p 	 \=\mbox{premise}\\
   \: q \= \mbox{premise}
   \[
     \: q\to p		  \=\mbox{assumption} \\
     \: p \= \to E(2,3) \\
     \: \bot \= \neg I(4,1)
   \]
   \:  \neg(q\to p) \= \neg E(3-5)
\end{proofbox}
\\
In (f), we prove $\neg p,q\vdash\gamma_{\textit{l}}[(q\to p)\to p]$ where $\gamma_{\textit{l}}[(q\to p)\to p]=(q\to p)\to p$. A box proof is provided below
\begin{proofbox}
   \: \neg p 	 \=\mbox{premise}\\
   \: q \= \mbox{premise}
   \[
     \: q\to p		  \=\mbox{assumption} \\
     \: p \= \to E(2,3)
   \]
   \:  (q\to p)\to p \= \to I(3-4)
\end{proofbox}
\\
In (g), we prove $\neg p,q\vdash\gamma_{\textit{l}}[((p\to q)\to q)\to((q\to p)\to p)]$ where $\gamma_{\textit{l}}[((p\to q)\to q)\to((q\to p)\to p)]=((p\to q)\to q)\to((q\to p)\to p)$. A boxproof is provided below
\begin{proofbox}
   \: \neg p 	 \=\mbox{premise}\\
   \: q \= \mbox{premise}\\
   \: (p\to q)\to q \= \mbox{proved in (d)} \\
   \: (q\to p)\to p \= \mbox{proved in (f)}
   \[
     \: (p\to q)\to q	  \=\mbox{assumption} \\
     \: (q\to p)\to p\= (4)
   \]
   \: ((p\to q)\to q)\to((q\to p)\to p) \= \to I(5-6)
\end{proofbox}
We now prove $\alpha_4$ in the same manner.
\begin{align*}
\neg p&\vdash \gamma_{\textit{l}}[p] &(a) \\
\neg q&\vdash \gamma_{\textit{l}}[q] &(b)\\
\neg p,\neg q&\vdash \gamma_{\textit{l}}[p\to q]&(c)\\
\neg p,\neg q&\vdash \gamma_{\textit{l}}[(p\to q)\to q]&(d)\\
\neg p,\neg q&\vdash \gamma_{\textit{l}}[q\to p]&(e)\\
\neg p,\neg q&\vdash \gamma_{\textit{l}}[(q\to p)\to p]&(f)\\
\neg p,\neg q&\vdash \gamma_{\textit{l}}[((p\to q)\to q)\to((q\to p)\to p)]&(g)
\end{align*}
In (a), we prove $\neg p\vdash\gamma_{\textit{l}}[p]$ where $\gamma_{\textit{l}}[p]=\neg p$ meaning it is trivially true.\\
In (b), we prove $\neg q\vdash\gamma_{\textit{l}}[q]$ where $\gamma_{\textit{l}}[q]=\neg q$ meaning it is trivially true.\\
In (c), we prove $\neg p,\neg q\vdash\gamma_{\textit{l}}[p\to q]$ where $\gamma_{\textit{l}}[p\to q]=p\to q$. A box proof is provided below
\begin{proofbox}
   \: \neg p 	 \=\mbox{premise}\\
   \: \neg q \= \mbox{premise}
   \[
      \: p		  \=\mbox{assumption}\\
      \: \bot \= \to E(3,1) \\
      \: q \= \bot E(4)
   \]
   \: p\to q \= \to I(3-5)
\end{proofbox}
\\
In (d), we prove $\neg p,\neg q\vdash\gamma_{\textit{l}}[(p\to q)\to q]$ where $\gamma_{\textit{l}}[(p\to q)\to q]=\neg((p\to q)\to q)$. A box proof is provided below
\begin{proofbox}
   \: \neg p 	 \=\mbox{premise}\\
   \: \neg q \= \mbox{premise} \\
   \: p\to q	 \= \mbox{proved in (c)}
   \[
     \: (p\to q)\to q		  \=\mbox{assumption} \\
     \: q \= \to E(3,4) \\
     \: \bot \= \neg E(5,2)
   \]
   \: \neg((p\to q)\to q) \= \neg I(4-6)
\end{proofbox}
\\
In (e), we prove $\neg p,\neg q\vdash\gamma_{\textit{l}}[q\to p]$ where $\gamma_{\textit{l}}[q\to p]=q\to p$. A box proof is provided below
\begin{proofbox}
   \: \neg p 	 \=\mbox{premise}\\
   \: \neg q \= \mbox{premise}
   \[
     \: q		  \=\mbox{assumption} \\
     \: \bot \= \neg E(3,2) \\
     \: p \= \bot E(4)
   \]
   \:  q\to p \= \to I(3-5)
\end{proofbox}
\\
In (f), we prove $\neg p,\neg q\vdash\gamma_{\textit{l}}[(q\to p)\to p]$ where $\gamma_{\textit{l}}[(q\to p)\to p]=\neg((q\to p)\to p)$. A box proof is provided below
\begin{proofbox}
   \: \neg p 	 \=\mbox{premise}\\
   \: \neg q \= \mbox{premise} \\
   \: (q\to p) \= \mbox{proved in (e)}
   \[
     \: (q\to p)\to p		  \=\mbox{assumption} \\
     \: p \= \to E(3,4) \\
     \: \bot \= \neg E(5,1)
   \]
   \:  \neg((q\to p)\to p) \= \neg I(4-6)
\end{proofbox}
\newpage
In (g), we prove $\neg p,\neg q\vdash\gamma_{\textit{l}}[((p\to q)\to q)\to((q\to p)\to p)]$ where $\gamma_{\textit{l}}[((p\to q)\to q)\to((q\to p)\to p)]=((p\to q)\to q)\to((q\to p)\to p)$. A boxproof is provided below
\begin{proofbox}
   \: p 	 \=\mbox{premise}\\
   \: q \= \mbox{premise}\\
   \: \neg((p\to q)\to q) \= \mbox{proved in (d)} \\
   \: \neg((q\to p)\to p) \= \mbox{proved in (f)}
   \[
     \: (p\to q)\to q	  \=\mbox{assumption} \\
     \: \bot \= \neg E(5,3) \\
     \: (q\to p)\to p \= \bot E(6)
   \]
   \: ((p\to q)\to q)\to((q\to p)\to p) \= \to I(5-7)
\end{proofbox}
These proofs $\alpha_1,\alpha_2,\alpha_3$ and $\alpha_4$ together can be used to prove that 
$$\vdash ((p\to q)\to q)\to((q\to p)\to p)$$
as all combination of truth values are used.\\
We extend this to to the proof
$$(p\to q)\to q\vdash (q\to p)\to p$$
As so we have proved completeness of the formula.

\end{document}
