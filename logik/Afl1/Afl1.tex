\documentclass[12pt]{article}
\usepackage[utf8]{inputenc}
\usepackage{amsmath}
\usepackage{amssymb}
\usepackage{mathtools}
\usepackage{amsfonts}
\usepackage{lastpage}
\usepackage{tikz}
\usepackage{pdfpages}
\usepackage{gauss}
\usepackage{fancyvrb}
\usepackage{fancyhdr}
\usepackage{graphicx}
\usepackage{boxproof}
\usepackage{a4wide}
\usepackage{daymonthyear}

\def\meta#1{\mbox{$\langle\hbox{#1}\rangle$}}
\def\macrowitharg#1#2{{\tt\string#1\bra\meta{#2}\ket}}

{\escapechar-1 \xdef\bra{\string\{}\xdef\ket{\string\}}}

\def\intro#1{{#1}{\cal I}}
\def\elim#1{{#1}{\cal E}}

\showboxbreadth 999
\showboxdepth 999
\tracingoutput 1


\let\imp\to
\def\elim#1{{{#1}{\cal E}}}
\def\intro#1{{{#1}{\cal I}}}
\def\lt{<}
\def\eqdef{=}
\def\eps{\mathrel{\epsilon}}
\def\biimplies{\leftrightarrow}
\def\flt#1{\mathrel{{#1}^\flat}}
\def\setof#1{{\left\{{#1}\right\}}}
\let\implies\to
\def\KK{{\mathsf K}}
\let\squashmuskip\relax
\pagestyle{fancy}
\fancyfoot[C]{\footnotesize Page \thepage\ of 6}
\DeclareGraphicsExtensions{.pdf,.png,.jpg}
\title{Elementær Talteori}
\author{Nikolaj Dybdahl Rathcke}
\chead{Nikolaj Dybdahl Rathcke (rfq695)}

\begin{document}
\section*{Logic in Computer Science - Assignment 1}
\subsection*{Exercise 1.1}
\subsubsection*{1e}
Let\\
"Cancer will be cured" be $p$.\\
"A cause is determined" be $q$.\\
"A new drug for cancer is found" be $r$.\\
\\
It can be expressed by the following
$$\neg (q \wedge r) \rightarrow \neg p $$

\subsubsection*{1h}
Let\\
"It will rain" be $p$.\\
"It will shine" be $q$.\\
\\
It can be expressed by the following
$$(\neg p \wedge q) \vee (p \wedge \neg q)$$

\subsection*{Exercise 1.2}
\subsubsection*{1b}
We have $p \wedge q \vdash q \wedge p$
\begin{proofbox}
   \: p\land q 	 \=\mbox{premise}\\
   \: p           \= \land E_1, 1\\
   \: q           \= \land E_2, 1\\
   \: q\land p     \= \land I_1, 3,2\\   
\end{proofbox}

\subsubsection*{1g}
We have $p \vdash q \rightarrow (p \land q)$
\begin{proofbox}
   \lbl{1}\: p 	 \=\mbox{premise}\\
   \[
      \: q		  \=\mbox{assumption}\\
      \: p \land q \= \land I,1,2 \\
   \]
   \: q\to (p\land q) \= \rightarrow I,2-3 \\
\end{proofbox}

\subsubsection*{1j}
We have $q\rightarrow r \vdash (p\rightarrow q) \rightarrow (p\rightarrow r)$
\begin{proofbox}
   \lbl{1}\: q\rightarrow r 	 \=\mbox{premise}\\
   \[
      \: p\rightarrow q		  \=\mbox{assumption}\\
        \[ 
          \: p \=\mbox{assumption}\\
          \: q \= \rightarrow E(3,2) \\
          \: r \= \rightarrow E(4,1)
        \]
      \: p\rightarrow r  \= \rightarrow I,3-5             
   \]
   \: (p\rightarrow q) \rightarrow (p\rightarrow r) \= \rightarrow I,2-6 \\
\end{proofbox}

\subsubsection*{1l}
We have $p\rightarrow q, r\rightarrow s \vdash p\lor r \rightarrow q\lor s$
\begin{proofbox}
   \lbl{1}\: p\rightarrow q 	 \=\mbox{premise}\\
   \lbl{1}\: r\rightarrow s 	 \=\mbox{premise}\\
   \[
      \: p\lor r		  \=\mbox{assumption}\\
        \[ 
          \: p \=\mbox{assumption}\\
          \: q \= \rightarrow E(4,1) \\
          \: q \lor s \= \lor I_1(5) \\
        \]
        \[ 
          \: r \=\mbox{assumption}\\
          \: s \= \rightarrow E(7,2) \\
          \: q \lor s \= \lor I_2(8) \\
        \]
      \: q\lor s  \= \lor E(3,4-6,7-9)
   \]
   \: p\lor r \rightarrow q\lor s \= \rightarrow I,3-10 \\
\end{proofbox}

\subsubsection*{3c}
We have $p\lor q, \neg q\vdash p$
\begin{proofbox}
   \: p\lor q 	 \=\mbox{premise}\\
   \: \neg q 	 \=\mbox{premise}\\
        \[ 
          \: p \=\mbox{assumption}\\
        \]
        \[ 
          \: q \=\mbox{assumption}\\
          \: \bot \= \neg E(2,4) \\
          \: p \= \bot E(5)
        \]
      \: p  \= \rightarrow I(1,3-3,4-6)
\end{proofbox}

\subsubsection*{5a}
We have $((p\rightarrow q)\rightarrow q)\rightarrow ((q\rightarrow p)\rightarrow p)$ \\
\begin{proofbox}
       \[ 
         \: (p\imp q)\imp q \=\mbox{assumption} 
         \[
           \: q\imp p\=\mbox{assumption}
           \[
             \: \neg p \=\mbox{assumption}
             \[
               \: p \=\mbox{assumption} \\
               \: \bot \= \neg E(4,3) \\
               \: q \= \bot E(5)
             \]      
             \: p\imp q \= \imp I(4-6) \\
             \: q \= \imp E(7,1) \\
             \: p \= \imp E(8,2) \\
             \: \bot \= \neg E(9,3)
           \]
           \: \neg \neg p \= \neg I(3-10) \\
           \: p \= \neg\neg E(11)
         \]
         \: (q\imp p)\imp p \= \imp I(2-12)
       \]
       \: ((p\imp q)\imp q) \imp ((q\imp p)\imp p) \= \imp I(1,13)
\end{proofbox}

\subsection*{Exercise 1.4}
\subsubsection*{6a}
The truth table looks as follows\\
\begin{center}
\begin{tabular}{|c|c|c|}
\hline 
p & q & p $\ast$ q \\ 
\hline 
T & T & F \\ 
\hline 
T & F & T  \\ 
\hline 
F & T & T  \\ 
\hline 
F & F & F  \\ 
\hline 
\end{tabular} 
\end{center}

\subsubsection*{6b}
The truth table looks as follows\\
\begin{center}
\begin{tabular}{|c|c|c|}
\hline 
p & q & (p $\ast$ p) $\ast$ (q $\ast$ q) \\ 
\hline 
T & T & F \\ 
\hline 
T & F & F  \\ 
\hline 
F & T & F  \\ 
\hline 
F & F & F  \\ 
\hline 
\end{tabular} 
\end{center}

\subsubsection*{6c}
Yes, it coincides with the table for absurdity, since it will always be false no matter the truth values of $p$ and $q$.

\subsubsection*{6d}
Yes, the operator in circuit design is known as xor (exclusive or).

\subsubsection*{12c}
When the truth values are as follows; $p=T,q=F$ and $r=T$ where the formulas to the left of $\vdash$ is\\
\begin{align*}
T\rightarrow(F\rightarrow T)\\
T\rightarrow T \\
T
\end{align*}
And the the formula to the right of $\vdash$ evaluates to 
\begin{align*}
T\rightarrow(T\rightarrow F)\\
T\rightarrow F \\
F
\end{align*}
Giving two different valuations.

\subsubsection*{13b}
For this task, we will be assuming apples are always green.\\
Giving the atoms the declarative sentences $p=$"It is green" and $q=$"It is an apple" evaluates the premise to
\begin{align*}
\neg T \rightarrow \neg F \\
F \rightarrow T \\
T
\end{align*}
Which is "if it is not green it is not an apple". Which is true as an apple can only be green.\\
But the conclusion becomes
\begin{align*}
\neg F\rightarrow \neg T \\
T \rightarrow F \\
F
\end{align*}
Which is "if it is not an apple, it is not green" which is false as several things can be green.\\
So the premise is true, but the conclusion is false.

\subsection*{Soundness proofs}
The cases below is based on the proof from the assignment text.
\subsubsection*{Case MT}
It has the subproofs $\Phi\vdash\phi\imp\psi$ and $\Phi\vdash\neg\psi$. By induction hypothesis $\Phi\models\phi\imp\psi$ and $\Phi\models\neg\psi$. We will then show that $\Phi\models\neg\phi$ meaning that in all scenarios where $[[\phi\imp\psi]]=[[\neg \psi]]=T$ then $[[\neg\phi]]=T$. If we look at the truthtable
\begin{center}
\begin{tabular}{|c|c||c|c||c|}
\hline 
$\phi$ & $\psi$ & $\phi\imp\psi$ & $\neg\psi$ & $\neg\phi$ \\ 
\hline 
T & T & T & F & F \\ 
\hline 
T & F & F & T & F\\ 
\hline
F & T & T & F & F\\ 
\hline
F & F & T & T & T\\ 
\hline
\end{tabular} 
\end{center}
it is seen to hold.

\subsubsection*{Case $\neg\neg i$}
It has the subproof $\Phi\vdash\phi$. By induction hypothesis $\Phi\models\phi$. We will then show that $\Phi\models\neg\neg\phi$ meaning in all scenarios where $[[\phi]]=T$ then $[[\neg\neg \phi]]=T$. If we look at the truthtable\\
\begin{center}
\begin{tabular}{|c||c|}
\hline 
$\phi$ & $\neg\neg\phi$ \\ 
\hline 
T & T \\ 
\hline 
F & F \\ 
\hline 
\end{tabular} 
\end{center}
it is seen to hold.

\subsubsection*{Case PBC}
It has the subproof $\Phi,\neg\phi\vdash\bot$. By induction hypothesis $\Phi,\neg\phi\models\bot$. We will then show $\Phi\models\phi$. Looking at the truthtable we see
\begin{center}
\begin{tabular}{|c|c||c|}
\hline 
$\neg\phi$ & $\bot$ & $\phi$ \\ 
\hline 
T & F & F \\ 
\hline 
F & F & T \\ 
\hline 
\end{tabular} 
\end{center}
We see there is a problem when $[[\neg\phi]]=F$. But $\Phi,\phi\models\bot$ says that when $\Phi$ are all true then $[[\phi]]=T$. So these lines where $[[\neg\phi]]=F$ and $[[\bot]]=F$ are not lines where $\Phi$ are all true, meaning that $\Phi\models\phi$.

\subsubsection*{Case LEM}
It has no subproof. This means that we need to just show $\models\phi\lor\neg\phi$,  meaning it is true in all scenarios. Looking at the truthtable we see
\begin{center}
\begin{tabular}{|c||c|}
\hline 
$\phi$ & $\phi\lor\neg\phi$ \\ 
\hline 
T & T  \\ 
\hline 
F & T \\ 
\hline 
\end{tabular} 
\end{center}
it is seen to hold true as it is always true.

\end{document}
