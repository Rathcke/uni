\documentclass[12pt]{article}
\usepackage{style}

\begin{document}

\section{Project collaboration}
We have developed a 'game' using 'experience' points and the opportunity to 'level' up. These point are distributed to the group participants by a point generator on: \url{http://www.VixQIT/XP/XP.php}
that generates a random amount of point in a certain range. The range is calculated from tasks completed. Tasks are worth points equal to the hours of work estimated during our scrum and these translates into points that provides the range from your random gain of experience.\\
The experience needed to level up increases each time using the Fibonacci sequence.\\
This game a motivation for us since there are small benefits (bragging rights, beers etc.) and the idea of making it competitive is appealing to all of us. While keeping the simple benefits, it won't ruin the dynamics in the group. Keeping it on a low level, a fun sidekick, we keep the ambitions and motivation high while still maintaining our good teamwork.\\
\\
Our group has met up since the last month and we did some coding on the application. We mainly did the front-end part and we'll hopefully start on the back-end next time we meet up. Up to this point, we had not done much on the application which made the size of the project a little hard to grasp. Since this went very well and gave us a good overview of the project, we expect to do more of these meetings and complete 'chunks' of work.
We tried to assign the group participants a part of the work and worked on it individually. This, however, could be a bad idea since we don't know eachother's code and it might be unfamiliar if we need to use it. So we'll try to explain parts of it in case some of it is too obscure.\\
\\
As for the communication with the client, we have not met physically, but we have an email correspondence established, where we contact each other weekly. We are currently still discussing design matters when it comes to the WebAPI, that is under development by Anders Hyldig. There has not been any progress on this matter. But for the API, we have been informed that some of the functions will be ready in week 18 or 19.

\end{document}