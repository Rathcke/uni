\documentclass[12pt]{article}
\begin{document}
The article describes how we strive to create an ideal rational design process, when developing computer system, and how this systematic approach, in their opinion, always will be unattainable. Instead, we can fake this process by showing the system to others so that it would appear to be designed rationally. They go into depth of explaining why the process is useful, what the process actually consists of, the importance of good documentation and how to fake relational design.\\

Rational design process should encourage the use of legacy software. In the listing reasoning why the rational design process is an idealization the authors argue that the use of legacy software inhibits the ideal software ref(1 II 7), since the legacy software is not written based on the requirements alone. Assuming that the process of faking the ideal design process is adopted as a standard, identifying legacy software that can be implemented according to the requirements would be in most cases, not just economically sound, but the very ideal of rational design. The legacy software would come with the documentation needed, hence minimizing  the amount of new documentation that the developers would have to produce. Furthermore adoption of terminology from the legacy software documentation into new production of documentation, would further standardization both within a organization and with the software development community as a whole. The apparent contradiction between the use of legacy software is also indirectly acknowledged by the authors in their conclusion "... the result is a product can be understood, maintained, and reused".\\

The OOSE book and the article differ greatly in their treatment of requirement documents. The article argues that this should be a formalized document, preferably developed by the client or representatives of the client. The OOSE book separates this document into two stages, a requirement specification produced by the client, users and developers in a natural language, which is then formalized by the developers in a analysis model. The iterative model of the OOSE book is for us, a group of novices in designing and developing software, much more appealing as the task at hand is not as overwhelming. Overall the rational design process requires extreme understanding of the design process, which we have yet to obtain.
\end{document}