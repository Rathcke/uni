\documentclass[12pt]{article}
\begin{document}
In the article, Lewis and Gould describes, what they believe is, the three principles for system design, that is to be followed if you want to make a useful and easy to use computer system. The three principles they recommend are: early focus on users and tasks, empirical measurements and iterative design. Even though these principles may seem intuitive to many, designers tend to avoid them, even if they think they're following them. Lewis and Gould points out many times, that user testing from a very early stage is crucial for the developing process, if a good system is to be made.\\
The article then proceeds to present an example, where their philosophy is being used successfully, in the development of IBM's ADS.\\ \\
The article correctly predicts that usability will be a key aspect of futur software, which we can confirm. Software developed today for the mainstream masses is heavily focused on easy of use, in order to appeal to an audience as wide as possible. An example of this is the success of almost all Apple products.\\ \\ 
Iterative design based on user feedback in comparsing with the SCUM model as described in the OOSE book. The article suggest an design process that extends throughout the development of the software project. This design method as described by the article differs from the books SCUM model in that the final design is fine polished before a new sprint starts, hence leaving the final product a tuning of the initial design. As we understand the article the iterative design is continously redesigning the product based on user input, thereby significantly remodeling the product from the initial design.

\end{document}