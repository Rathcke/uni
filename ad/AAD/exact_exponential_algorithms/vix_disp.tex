\documentclass[a4paper]{article}
\usepackage[utf8]{inputenc}
\usepackage{amsmath}
\usepackage{amssymb}
\usepackage{mathtools}
\usepackage{amsfonts}
\usepackage{lastpage}
\usepackage{tikz}
\usepackage{float}
\usepackage{textcomp}
\usetikzlibrary{patterns}
\usepackage{pdfpages}
\usepackage{gauss}
\usepackage{fancyvrb}
\usepackage[table]{colortbl}
\usepackage{fancyhdr}
\usepackage{graphicx}
\usepackage[margin=2.5 cm]{geometry}

\definecolor{listinggray}{gray}{0.9}
\usepackage{listings}
\lstset{
	language=,
	literate=
		{æ}{{\ae}}1
		{ø}{{\o}}1
		{å}{{\aa}}1
		{Æ}{{\AE}}1
		{Ø}{{\O}}1
		{Å}{{\AA}}1,
	backgroundcolor=\color{listinggray},
	tabsize=3,
	rulecolor=,
	basicstyle=\scriptsize,
	upquote=true,
	aboveskip={0.2\baselineskip},
	columns=fixed,
	showstringspaces=false,
	extendedchars=true,
	breaklines=true,
	prebreak =\raisebox{0ex}[0ex][0ex]{\ensuremath{\hookleftarrow}},
	frame=single,
	showtabs=false,
	showspaces=false,
	showlines=true,
	showstringspaces=false,
	identifierstyle=\ttfamily,
	keywordstyle=\color[rgb]{0,0,1},
	commentstyle=\color[rgb]{0.133,0.545,0.133},
	stringstyle=\color[rgb]{0.627,0.126,0.941},
  moredelim=**[is][\color{blue}]{@}{@},
}

\lstdefinestyle{base}{
  emptylines=1,
  breaklines=true,
  basicstyle=\ttfamily\color{black},
}

\def\checkmark{\tikz\fill[scale=0.4](0,.35) -- (.25,0) -- (1,.7) -- (.25,.15) -- cycle;}
\newcommand*\circled[1]{\tikz[baseline=(char.base)]{
            \node[shape=circle,draw,inner sep=2pt] (char) {#1};}}
\newcommand{\comment}[1]{%
  \text{\phantom{(#1)}} \tag{#1}}
\newcommand*\squared[1]{%
  \tikz[baseline=(R.base)]\node[draw,rectangle,inner sep=0.5pt](R) {#1};\!}
\cfoot{Page \thepage\ of \pageref{LastPage}}
\DeclareGraphicsExtensions{.pdf,.png,.jpg}
\author{Nikolaj Dybdahl Rathcke (rfq695)}

\begin{document}
\begin{center}
\section*{Exact Exponential Algorithms}
\end{center}
Some problems can be hard to solve. We will look for good exact exponential algorithms to solve these problems. One problem is the TSP problem, which by bruteforce takes $\mathcal{O}(n!)$ time. Another is the MIS problem that takes $\mathcal{O}(2^n)$ time by bruteforce. We introduce the notion $O^*(g(n))$ for a function $f(n)=O(g(n)poly(n))$. \\
\\
Dynamic approach to TSP works by calculating the lowest cost path in every subset. The amount of steps required to calculate the optimal route is $O(k^2)$, and we calculate all subsets (without $c_1$), so our running time is
$$\sum_{k=1}^{n-1}O(\binom{n}{k}k^2)=O^*(2^n)$$
The algorithm for MIS works by picking a vertex $v$ with minimum degree and finds the optimal solution (recursively) by removing a vertex from the neighbourhood $N(v)$.\\
The worse case running time of this algorithm becomes the largest number of nodes, $T(n)$. We define $d(v)$ as the degree of the vertex we picked. That means we can express the number of nodes in the branching tree as
\begin{align*}
T(n)&\leq 1+T(n-d(v)-1)+\sum_{i=1}^{d(v)}T(n-d(v_i)-1) \\
\comment{Since v had min degree} &\leq 1+T(n-d(v)-1)+\sum_{i=1}^{d(v)}T(n-d(v)-1) \\
&=1+(d(v)+1)\cdot T(n-d(v)-1)
\end{align*}
Setting $s=(d(v)+1$, we get
\begin{align*}
T(n)&\leq 1+s\cdot T(n-s) \\
&\leq 1+s+s^2+\cdots+s^{n/s} \\
&=\frac{1-s^{n/s+1}}{1-s}=\mathcal{O}^*(s^{n/s})
\end{align*}
This has maximum for $s=3$, which means
$$T(n)=O^*(3^{n/3})$$
We say a problem is fixed parameter tractable if you have an algorithm that runs in $O(f(k)poly(n))$ time for some parameter $k$. If you know that vertex cover needs some size $k$, we can reduce the problem so it has $k^2+k$ vertices and $k^2$ edges. Do that by removing vertices with degree $0$ and $1$ (when $1$, we keep the neighboor). Also include vertices that has a degree strictly larger than $k$.


\end{document}