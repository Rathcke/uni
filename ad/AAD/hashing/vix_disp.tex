\documentclass[a4paper]{article}
\usepackage[utf8]{inputenc}
\usepackage{amsmath}
\usepackage{amssymb}
\usepackage{mathtools}
\usepackage{amsfonts}
\usepackage{lastpage}
\usepackage{tikz}
\usepackage{float}
\usepackage{textcomp}
\usetikzlibrary{patterns}
\usepackage{pdfpages}
\usepackage{gauss}
\usepackage{fancyvrb}
\usepackage[table]{colortbl}
\usepackage{fancyhdr}
\usepackage{graphicx}
\usepackage[margin=2.5 cm]{geometry}

\definecolor{listinggray}{gray}{0.9}
\usepackage{listings}
\lstset{
	language=,
	literate=
		{æ}{{\ae}}1
		{ø}{{\o}}1
		{å}{{\aa}}1
		{Æ}{{\AE}}1
		{Ø}{{\O}}1
		{Å}{{\AA}}1,
	backgroundcolor=\color{listinggray},
	tabsize=3,
	rulecolor=,
	basicstyle=\scriptsize,
	upquote=true,
	aboveskip={0.2\baselineskip},
	columns=fixed,
	showstringspaces=false,
	extendedchars=true,
	breaklines=true,
	prebreak =\raisebox{0ex}[0ex][0ex]{\ensuremath{\hookleftarrow}},
	frame=single,
	showtabs=false,
	showspaces=false,
	showlines=true,
	showstringspaces=false,
	identifierstyle=\ttfamily,
	keywordstyle=\color[rgb]{0,0,1},
	commentstyle=\color[rgb]{0.133,0.545,0.133},
	stringstyle=\color[rgb]{0.627,0.126,0.941},
  moredelim=**[is][\color{blue}]{@}{@},
}

\lstdefinestyle{base}{
  emptylines=1,
  breaklines=true,
  basicstyle=\ttfamily\color{black},
}

\def\checkmark{\tikz\fill[scale=0.4](0,.35) -- (.25,0) -- (1,.7) -- (.25,.15) -- cycle;}
\newcommand*\circled[1]{\tikz[baseline=(char.base)]{
            \node[shape=circle,draw,inner sep=2pt] (char) {#1};}}
\newcommand*\squared[1]{%
  \tikz[baseline=(R.base)]\node[draw,rectangle,inner sep=0.5pt](R) {#1};\!}
\cfoot{Page \thepage\ of \pageref{LastPage}}
\DeclareGraphicsExtensions{.pdf,.png,.jpg}
\author{Nikolaj Dybdahl Rathcke (rfq695)}

\begin{document}
\begin{center}
\section*{Hashing}
\end{center}
\subsection*{Universality}
A hash function $h: U\rightarrow [m]$ maps values from a key universe $U$ into values in $m=[0...m-1]$. \\
Universal hashing is the concept of generating a random \textit{universal} hash function $h$, so when we pick two distinct keys $x,y\in U$, the probability of collision is:
$$Pr_h[h(x)=h(y)]\leq 1/m\ \ or\ \ Pr_h[h(x)=h(y)]\leq c/m$$

\subsection*{Application}
Universal - Hash tables with chaining. When we want lookups in expected constant time, $1+|L(h(x)|$. If we have used a universal hash function, we can expect the buckets to be of size $|S|/m$ . (We have the indicator varialbe $I(y)$. It is the number of collisions with a new key $x\not\in S$).
$$E[L(h(x))]=E[\sum_{y\in S}I(y)]=\sum_{y\in S}E[I(y)]=\sum_{y\in S}E[h(x)=h(y)]=|S|\cdot \frac{1}{m}$$

\subsection*{Strong universality}
A stronger condition known as pairwise independence or strong universality is when given two distinct keys $x,y\in U$ hash to values $r$ and $q$ respectively with probability $1/m^2$. If it is strongly universal, it implies it is also universal, as
$$Pr[h(x)=h(y)]=\sum_{q\in [m]}Pr[h(x)=q\land h(y)=q]=m/m^2=1/m$$
Proof that two keys are hashed individually and each key is hashed uniformly into $[m]$. Uniformly as each pair has exactly $1/m^2$, and there are $m$ values of $r$ for each $q$. \\
Independence (calculate $P[A|B]=\frac{P[A]*P[B]}{P[B]}$).

\subsection*{Application}
Strongly universal - coordinated sampling, important in handling of big data and machine learning. We can define a set $S_{h,t}(A)$ from a set $A$, a strongly universal hash function $h$ and a threshold $t$. The size is $|A|\cdot t/m$ as a strongly universal hash function means that values are uniformly mapped to $[m]$. \\
We can say something about unions and intersections between two sets by multiplying with $m/t$. \\
\\
Chebyshev's inequality?
$$Pr[|X-\mu | \geq q\sigma_x]\leq 1/q^2$$
for $q>0$. Says something about that in any probability distribution, "nearly all" values are close to the mean.

\subsection*{Implementations}
\textbf{Multiply-mod-prime:} \\
Universal (with $c=1$) where $h_{a,b}:[u]\rightarrow [m]$:
$$h_{a,b}(x)=((ax+b) \mbox{ mod } p)\mbox{ mod } m$$
Strongly universal where $h_{a,b}:[p]\rightarrow [p]$:
$$h_{a,b}(x)=(ax+b)\mbox{ mod }p$$
\\
\textbf{Multiply-shift:} \\
Universal (with $c=2$) where $h_a:[2^w]\rightarrow [2^d]$:
$$h_a(x)=\lfloor (ax \mbox{ mod } 2^w)/2^{w-l}\rfloor$$
Strongly universal where $h_{a,b}:[2^w]\rightarrow [2^l]$:
$$h_{a,b}(x)=(ax+b)[w'-l, w']$$
and $w'\geq w+l-1$

\end{document}