\documentclass[a4paper]{article}
\usepackage[utf8]{inputenc}
\usepackage{amsmath}
\usepackage{amssymb}
\usepackage{mathtools}
\usepackage{amsfonts}
\usepackage{lastpage}
\usepackage{tikz}
\usepackage{float}
\usepackage{textcomp}
\usetikzlibrary{patterns}
\usepackage{pdfpages}
\usepackage{gauss}
\usepackage{fancyvrb}
\usepackage[table]{colortbl}
\usepackage{fancyhdr}
\usepackage{graphicx}
\usepackage[margin=2.5 cm]{geometry}

\definecolor{listinggray}{gray}{0.9}
\usepackage{listings}
\lstset{
	language=,
	literate=
		{æ}{{\ae}}1
		{ø}{{\o}}1
		{å}{{\aa}}1
		{Æ}{{\AE}}1
		{Ø}{{\O}}1
		{Å}{{\AA}}1,
	backgroundcolor=\color{listinggray},
	tabsize=3,
	rulecolor=,
	basicstyle=\scriptsize,
	upquote=true,
	aboveskip={0.2\baselineskip},
	columns=fixed,
	showstringspaces=false,
	extendedchars=true,
	breaklines=true,
	prebreak =\raisebox{0ex}[0ex][0ex]{\ensuremath{\hookleftarrow}},
	frame=single,
	showtabs=false,
	showspaces=false,
	showlines=true,
	showstringspaces=false,
	identifierstyle=\ttfamily,
	keywordstyle=\color[rgb]{0,0,1},
	commentstyle=\color[rgb]{0.133,0.545,0.133},
	stringstyle=\color[rgb]{0.627,0.126,0.941},
  moredelim=**[is][\color{blue}]{@}{@},
}

\lstdefinestyle{base}{
  emptylines=1,
  breaklines=true,
  basicstyle=\ttfamily\color{black},
}

\def\checkmark{\tikz\fill[scale=0.4](0,.35) -- (.25,0) -- (1,.7) -- (.25,.15) -- cycle;}
\newcommand*\circled[1]{\tikz[baseline=(char.base)]{
            \node[shape=circle,draw,inner sep=2pt] (char) {#1};}}
\newcommand*\squared[1]{%
  \tikz[baseline=(R.base)]\node[draw,rectangle,inner sep=0.5pt](R) {#1};\!}
\newcommand{\comment}[1]{%
  \text{\phantom{(#1)}} \tag{#1}}
\cfoot{Page \thepage\ of \pageref{LastPage}}
\DeclareGraphicsExtensions{.pdf,.png,.jpg}
\author{Nikolaj Dybdahl Rathcke (rfq695)}

\begin{document}
\begin{center}
  \section*{Approximation Algorithms}
\end{center}
Sometimes we have problems that cannot be solved optimally efficiently - polynomial time. But in practice, it is often good enough to have a near-optimal solution, which is what approximation algorithms do. We say they have an approximation ratio $\rho (n)$ if
$$max\left( \frac{C}{C^*}, \frac{C^*}{C} \right)\leq \rho (n)$$
There are also approximation schemes, which are algorithms that take an input $\varepsilon$, so that the scheme is a $(1+\varepsilon)$-approximation algorithm. We say the scheme is fully polynomial if it is polynomial in both $1/\varepsilon$ and $n$. \\
\\
Consider the optimization problem of the NP-complete optimization problem Vertex cover. The algorithm takes an arbitrary edge and add both endpoints to $C$, and does this until all edges are covered. We show it is a polynomial time $2$-approximation algorithm:\\
\\
It is polynomial since it checks at most all edges once removes edges from the corresponding vertices. It runs in $\mathcal{O}(V + E)$ if we use adjacency lists. When we pick an edge to put in $A$, at least one of the vertices must be in the optimal solution $C^*$. No two edges in $A$ are covered by the same vertex. We have
$$|A|\leq |C^*|$$
Since the number of vertices in the produced solution $C$ is exactly $2|A|$, we have
\begin{align*}
|C|&=2|A| \\
&\leq 2|C^*|
\end{align*}
Another problem is the traveling salesman problem where no efficient solution exist (a hamiltonian cycle with minimum cost). We show there is a $2$-approximation algorithm when it is a complete undirected graph and the cost function $c$ in TSP satisfies the triangle inequality, that is
$$c(u,w)\leq c(u,v)+c(v,w)$$
It works by generating a minimum spanning tree and listing the nodes in order when they are visited in a preorder tree walk, and this hamiltonian circle is returned. It is clearly polynomial running time ($\mathcal{O}(|E|\lg |V|)$ with binary heaps and adjacency list - $\mathcal{O}(|E|+|V|\lg |V|)$ with fib heaps). \\
Proof: The spanning tree $T$ provides a lower bound for the cost of an optimal tour $H^*$, so
$$c(T)\leq c(H^*)$$
Now let us consider the walk $W$. Every vertex is visited twice, so
$$c(W)=2c(T)$$
Which we can simply substitute, so
$$c(W)\leq 2c(H^*)$$
The solution produced is just the walk between each vertex (visit only once), but when the triangle inequality we can deduce $c(H)\leq c(W)$ and therefore $c(H)\leq 2c(H^*)$.

There is no $\rho(n)$ approximation algorithm as it prove $P=NP$ by making a complete undirected graph, the cost function $c(u,v)=1$ when $(u,v)\in E$ and $c(u,v)=\rho(|V|)+1$ if not, and then we could find a hamiltonian cycle.

\end{document}
