\documentclass[a4paper]{article}
\usepackage[utf8]{inputenc}
\usepackage{amsmath}
\usepackage{amssymb}
\usepackage{mathtools}
\usepackage{amsfonts}
\usepackage{lastpage}
\usepackage{tikz}
\usepackage{float}
\usepackage{textcomp}
\usetikzlibrary{patterns}
\usepackage{pdfpages}
\usepackage{gauss}
\usepackage{fancyvrb}
\usepackage[table]{colortbl}
\usepackage{fancyhdr}
\usepackage{graphicx}
\usepackage[margin=2.5 cm]{geometry}

\definecolor{listinggray}{gray}{0.9}
\usepackage{listings}
\lstset{
	language=,
	literate=
		{æ}{{\ae}}1
		{ø}{{\o}}1
		{å}{{\aa}}1
		{Æ}{{\AE}}1
		{Ø}{{\O}}1
		{Å}{{\AA}}1,
	backgroundcolor=\color{listinggray},
	tabsize=3,
	rulecolor=,
	basicstyle=\scriptsize,
	upquote=true,
	aboveskip={0.2\baselineskip},
	columns=fixed,
	showstringspaces=false,
	extendedchars=true,
	breaklines=true,
	prebreak =\raisebox{0ex}[0ex][0ex]{\ensuremath{\hookleftarrow}},
	frame=single,
	showtabs=false,
	showspaces=false,
	showlines=true,
	showstringspaces=false,
	identifierstyle=\ttfamily,
	keywordstyle=\color[rgb]{0,0,1},
	commentstyle=\color[rgb]{0.133,0.545,0.133},
	stringstyle=\color[rgb]{0.627,0.126,0.941},
  moredelim=**[is][\color{blue}]{@}{@},
}

\lstdefinestyle{base}{
  emptylines=1,
  breaklines=true,
  basicstyle=\ttfamily\color{black},
}

\def\checkmark{\tikz\fill[scale=0.4](0,.35) -- (.25,0) -- (1,.7) -- (.25,.15) -- cycle;}
\newcommand*\circled[1]{\tikz[baseline=(char.base)]{
            \node[shape=circle,draw,inner sep=2pt] (char) {#1};}}
\newcommand*\squared[1]{%
  \tikz[baseline=(R.base)]\node[draw,rectangle,inner sep=0.5pt](R) {#1};\!}
\cfoot{Page \thepage\ of \pageref{LastPage}}
\DeclareGraphicsExtensions{.pdf,.png,.jpg}
\author{Nikolaj Dybdahl Rathcke (rfq695)}

\begin{document}
\begin{center}
\section*{Computational Geometry}
\end{center}
If we want to model a terrain in $3d$ where we have measured height in a points $p$ from a set $P$, we can do this by making a \textit{triangulation} of $P$, which is a planar subdivision whos bounded faces are triangles. These are then lifted to their height. \\
Some triangulations seem more natural than other, specifically the triangulations that maximizes the minimum angle. One triangulation that satisfies this is the \textit{Delaunay Triangulation}.
First we introduce Thales's theorem. Thales's theorem states that (with example):
$$\measuredangle arb > \measuredangle apb=\measuredangle aqb > \measuredangle asb$$
Now edge flipping is the act of flipping the edge that is adjacent to two triangles ($4$ vertices) to the opposite vertices. We do this if it is illegal, that is, if flipping it maximizes the minimal angle of the angle vector. \\
\\
A Delaunay graph can be found from Voronoi diagrams, if two faces share an edge in the Voronoi diagram we draw a straight lines between the two vertices that represent these faces. This is always a plane graph, meaning the edges do not intersect eachother. \\
\\
\textbf{Properties of a Delaunay triangulations}:
\begin{itemize}
  \item An edge can only be in the Delaunay graph if there exist a disc, with the endpoints on its boundary so there is no other node (or site) in the disc.
  \item Three points are vertices to the same face if the circumcircle contains no other points in it.
\end{itemize}
We now want to show that a triangulation of $P$ is legal if and only if it is also a Delaunay triangulation of $P$. \\
It follows from the definitions that a Delaunay triangulation is legal. We now want to show that any legal triangulation is also a Delaunay triangulation by contradiction. \\
When it is not a Delaunay triangulation we know that there is a triangle $p_ip_jp_k$ whose circle contains a point $p_l$.
We pick the $p_l$ that maximizes the angle $\measuredangle p_ip_jp_l$. We now consider the triangle $p_ip_jp_m$. Since it is a legal triangulation, that means the edge $e=p_ip_j$ is legal, and that means $p_m$ is not in the disc around $p_ip_jp_k$. We now have two discs where $p_l\in C(p_ip_jp_m)$. \\
We now make a new quadriliteral by making the edge $p_jp_m$. By Thales's theorem, we now have that $\measuredangle p_jp_lp_m > \measuredangle p_ip_jp_l$ which is a contradiction to to the definition of the pair $(p_ip_lp_j,p_l)$. \\
\\
An algorithm to compute the Delaunay triangulation directly is a \textit{randomized incremental algorithm}. This works by creating a large triangle with the "highest" point and two extra points that do not interfere with the discs. We then pick a random point and draw triangles and legalize every edge afterwards. When randomized, it runs in $\mathcal{O}(n\lg n)$ since the expected number of \texttt{legalizeEdges} is $\mathcal{O}(1)$. We also keep a data structure so the expected time it takes to find the location (triangle) of a point $p$ is found in $\mathcal{O}(\lg n)$. 


\end{document}
