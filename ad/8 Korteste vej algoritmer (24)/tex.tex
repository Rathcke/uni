\documentclass[12pt]{article}
\usepackage[utf8]{inputenc}
\usepackage{amsmath}
\usepackage{mathtools}
\usepackage{amsfonts}
\usepackage{lastpage}
\usepackage{tikz}
\usepackage{pdfpages}
\usepackage{gauss}
\usepackage{fancyvrb}
\usepackage{fancyhdr}
\usepackage{graphicx}
\pagestyle{fancy}
\fancyfoot[C]{\footnotesize Page \thepage\ of 15}
\DeclareGraphicsExtensions{.pdf,.png,.jpg}
\title{Database and Web Programming}
\author{Nikolaj Dybdahl Rathcke}
\chead{Nikolaj Dybdahl Rathcke - rfq695}

\begin{document}

Intro - algoritmer til effektivt at finde den korteste vej (dijkstra og bellman-ford)\\
\\
Opskriv vægtet, directed, graf(V,E). Har vægt funktion w som er reele tal. delta(u,v) er vægten for korteste vej hvis findes. Ellers uendeligt.\\
\\
Single-source korteste veje. Fra s til alle v.\\
\\
Opskriv:\\
Trekant ulighed\\
Predeecessor-subgraph property.\\
Lemma 24.2\\
Corollary 24.3\\
No path property\\
\\
bellman-ford (O(VE)): hvad gør den - (kalde til intilaize, relaxtion). Bevis korrekthed\\
\\
Dijkstra: Hvad gør den - min queue som kører lg V - og gør det E gange\\

\end{document}
