\documentclass[11pt,a4paper]{article}
\usepackage[utf8]{inputenc}
\usepackage{amsmath}
\usepackage{amssymb}
\usepackage{mathtools}
\usepackage{amsfonts}
\usepackage{lastpage}
\usepackage{tikz}
\usepackage{pdfpages}
\usepackage{gauss}
\usepackage{fancyvrb}
\usepackage{fancyhdr}
\usepackage{graphicx}
\usepackage{titlesec}
\usepackage[margin=2.5cm]{geometry}
\DeclareGraphicsExtensions{.pdf,.png,.jpg}

\begin{document}

\subsection*{Intro}
Til dette indfører vi en ny differens operator, $\delta_hf(x_0)$, som istedet for at være afhængig af $f(x)$-værdier tæt på $f(x_0)$ kan denne istedet afhænge af en finit mængde punkter ($x_0+a_kh$) som bruges differens operatoren
$$\delta_hf(x_0)=\sum_{j=0}^nc_jf(x_0+a_jh)$$

\subsection*{Taylor series method}
Approximationen kan gives ved følgende 
$$DF(x_0)=\delta_hf(x_0)+O(h^q)$$
Større $q$ er bedre approximation. Mindre $n$ er billigere computation. Når vi skal finde vores vægte $c_j$ som gør vores $q$ højest mulig, laver vi først taylor expansion for $f(z_j)$, hvor $z_j$ er punktet på formen $x_0+a_jh,x_0-a_jh$ eller en mix, så
$$\delta_{+,-,+}f(x_0)=\sum_{j=0}^nc_jf(z_j)=\sum_{j=0}^nc_j[\sum_{k=0}^\infty \frac{1}{k!}f^{(k)}(x_0)(z_j-x_0)^k]$$
Hvor $z_j-x_0$ kan udtrykkes ved $h^k$.\\
Herefter vælges $c_j$ værdier så de gør ordenen af $q$ størst.

\subsection*{Lore}
$n$ er pre selected og alt efter $n$ om det er $\delta{+,-,0}$ kaldes det $n$-punkts forward,backward eller central differens operator.\\
Punkterne er også pre selected.

\end{document}
