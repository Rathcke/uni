\documentclass[12pt]{article}
\usepackage[utf8]{inputenc}
\usepackage{amsmath}
\usepackage{amssymb}
\usepackage{mathtools}
\usepackage{amsfonts}
\usepackage{lastpage}
\usepackage{tikz}
\usepackage{pdfpages}
\usepackage{gauss}
\usepackage{fancyvrb}
\usepackage{fancyhdr}
\usepackage{graphicx}
\pagestyle{fancy}
\fancyfoot[C]{\footnotesize Page \thepage\ of 3}
\DeclareGraphicsExtensions{.pdf,.png,.jpg}
\title{Elementær Talteori}
\author{Nikolaj Dybdahl Rathcke}
\chead{Nikolaj Dybdahl Rathcke (rfq695)}

\begin{document}
Lad sekvenserne være
\begin{align*}
x_1&=[1,0,-2,-6-14,-30,...] \\
x_2&=[1,1,1,1,...] \\
x_3&=[2,4,8,16,...]
\end{align*}
Dette kan bruges til at opstille ligningerne
\begin{align*}
1&=a+2b \\
0&=a+4b
\end{align*}
Hvoraf vi kan udregne de 2 konstanter ved at isolere $b$ i den anden ligning og sætte dette udtryk ind i den første for $b$ for at få $a$
\begin{align*}
0&=a+4b \\
b&=-\frac{a}{4} \\
1&=a+2(-\frac{a}{4}) \\
a&=2
\end{align*}
Og herefter udregne $b$
\begin{align*}
1&=2+2b \\
b&=-\frac{1}{2}
\end{align*}
Af dette kan vi opstille den første sekvens som en lineær kombination af de 2 andre sekvenser.
\begin{align*}
x_1=2x_{2}-\frac{1}{2}x_{3}
\end{align*}


\end{document}
