\documentclass[12pt]{article}
\usepackage[utf8]{inputenc}
\usepackage{amsmath}
\usepackage{amssymb}
\usepackage{mathtools}
\usepackage{amsfonts}
\usepackage{lastpage}
\usepackage{tikz}
\usepackage{pdfpages}
\usepackage{gauss}
\usepackage{fancyvrb}
\usepackage{fancyhdr}
\usepackage{graphicx}
\pagestyle{fancy}
\fancyfoot[C]{\footnotesize Page \thepage\ of 3}
\DeclareGraphicsExtensions{.pdf,.png,.jpg}
\author{Nikolaj Dybdahl Rathcke}
\chead{Nikolaj Dybdahl Rathcke (rfq695)}

\begin{document}

\section*{7.3.3}
Since it is a 4-point Gaussian quadratue rule, we write our $(2n-1)$-th polynomium as
$$f(x)=a_0+a_1x+a_2x^2+a_3x^3+a_4x^4+a_5x^5+a_6x^6+a_7x^7$$
So we want to find the left hand side which can be written as
\begin{align*}
\int_{-1}^1f(x)dx &=[a_0x+a_1(\frac{x^2}{2})+a_2(\frac{x^3}{3})+a_3(\frac{x^4}{4})+a_4(\frac{x^5}{5})\\
&\:\:\:\:+a_5(\frac{x^6}{6})+a_6(\frac{x^7}{7})+a_7(\frac{x^8}{8})]_{-1}^1\\
&=a_0(1-(-1))+a_1(\frac{1^2-(-1)^2}{2})+a_2(\frac{1^3-(-1)^3}{3})+a_3(\frac{1^4-(-1)^4}{4})\\
&\:\:\:\:+a_4(\frac{1^5-(-1)^5}{5})+a_5(\frac{1^6-(-1)^6}{6})+a_6(\frac{1^7-(-1)^7}{7})+a_7(\frac{1^8-(-1)^8}{8})\\
&=a_0(2)+a_1(\frac{0}{2})+a_2(\frac{2}{3})+a_3(\frac{0}{4})+a_4(\frac{2}{5})
+a_5(\frac{0}{6})+a_6(\frac{2}{7})+a_7(\frac{0}{8})\\
&=2a_0+\frac{2}{3}a_2+\frac{2}{5}a_4+\frac{2}{7}a_6 \addtocounter{equation}{1}\tag{\theequation}
\end{align*}
Our formula from the exercise text can be written as
\begin{align*}
\int_{-1}^1f(x)dx &\approx A_0(a_0+a_1x_0+a_2x_0^2+a_3x_0^3+a_4x_0^4+a_5x_0^5+a_6x_0^6+a_7x_0^7)\\
&\:\:\:\:+A_1(a_0+a_1x_1+a_2x_1^2+a_3x_1^3+a_4x_1^4+a_5x_1^5+a_6x_1^6+a_7x_1^7)\\
&\:\:\:\:+A_2(a_0+a_1x_2+a_2x_2^2+a_3x_2^3+a_4x_2^4+a_5x_2^5+a_6x_2^6+a_7x_2^7)\\
&\:\:\:\:+A_3(a_0+a_1x_3+a_2x_3^2+a_3x_3^3+a_4x_3^4+a_5x_3^5+a_6x_3^6+a_7x_3^7)\\
&=a_0(A_0+A_1+A_2+A_3)+a_1(A_0x_0+A_1x_1+A_2x_2+A_3x_3)\\
&\:\:\:\:+a_2(A_0x_0^2+A_1x_1^2+A_2x_2^2+A_3x_3^2)+a_3(A_0x_0^3+A_1x_1^3+A_2x_2^3+A_3x_3^3)\\
&\:\:\:\:+a_4(A_0x_0^4+A_1x_1^4+A_2x_2^4+A_3x_3^4)+a_5(A_0x_0^5+A_1x_1^5+A_2x_2^5+A_3x_3^5)\\
&\:\:\:\:+a_6(A_0x_0^6+A_1x_1^6+A_2x_2^6+A_3x_3^6)+a_7(A_0x_0^7+A_1x_1^7+A_2x_2^7+A_3x_3^7)
\end{align*}
Now we want equation (1) to be the same as what we found above. Obviously we only need to look at $a_0,a_2,a_4$ and $a_6$, meaning we can set up the following equations
\begin{align*}
2&=A_1+ A_2+A_3+A_4 \\
\frac{2}{3}&=A_0x_0^2+A_1x_1^2+A_2x_2^2+A_3x_3^2 \\
\frac{2}{5}&=A_0x_0^4+A_1x_1^4+A_2x_2^4+A_3x_3^4 \\
\frac{2}{7}&=A_0x_0^6+A_1x_1^6+A_2x_2^6+A_3x_3^6
\end{align*}
We know that the $x_i$ nodes are the roots of the $n$-th polynomium, which for $n=4$ is those given in the book. To find the weights, you can solve the linear system with 4 equations and 4 unknowns. Thus you find $A_0,A_1,A_2$ and $A_3$.\\
Since we have now estimated our integral with these found values, we can say
$$\int_{-1}^1f(x)dx\approx A_0f(x_0)+A_1f(x_1)+A_2f(x_2)+A_3f(x_3)$$


\end{document}
