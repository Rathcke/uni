\documentclass[12pt]{article}
\usepackage[utf8]{inputenc}
\usepackage{amsmath}
\usepackage{amssymb}
\usepackage{mathtools}
\usepackage{amsfonts}
\usepackage{lastpage}
\usepackage{tikz}
\usepackage{pdfpages}
\usepackage{gauss}
\usepackage{fancyvrb}
\usepackage{fancyhdr}
\usepackage{graphicx}
\pagestyle{fancy}
\fancyfoot[C]{\footnotesize Page \thepage\ of 3}
\DeclareGraphicsExtensions{.pdf,.png,.jpg}
\title{Elementær Talteori}
\author{Nikolaj Dybdahl Rathcke}
\chead{Nikolaj Dybdahl Rathcke (rfq695)}

\begin{document}

\section*{4.7.13}
We have the following linear system
$$
\begin{bmatrix}
2 & 0 & -1 \\
-2 & -10 & 0 \\
-1 & -1 & 4
\end{bmatrix}
\begin{bmatrix}
x_1 \\
x_2 \\
x_3
\end{bmatrix}
=
\begin{bmatrix}
1 \\
-12 \\
2
\end{bmatrix}
$$
We carry out two iterations using the Jacobi method with our initial guess $x^{(0)}=(0,0,0)$.\\
This means to find $x^{(1)}$ we need to solve
$$
\begin{bmatrix}
2x_1^{(1)} & 0 & 0 \\
0 & -10x_2^{(1)} & 0 \\
0 & 0 & 4x_3^{(1)}
\end{bmatrix}
=
\begin{bmatrix}
1 \\
-12 \\
2
\end{bmatrix}
$$
Where $0$ has replaced the coordinates that are multiplied with coordinates from $x^{(0)}$. We find that $x^{(1)}=(\frac{1}{2},\frac{6}{5},\frac{1}{2})$. With our new approximation of $(x_1,x_2,x_3)$ we solve the following
$$
\begin{bmatrix}
2x_1^{(2)} & 0x_2^{(1)} & -1x_3^{(1)} \\
-2x_1^{(1)} & -10x_2^{(2)} & 0x_3^{(1)} \\
-1x_1^{(1)} & -1x_2^{(1)} & 4x_3^{(2)}
\end{bmatrix}
=
\begin{bmatrix}
1 \\
-12 \\
2
\end{bmatrix}
$$
Rewritten to
$$
\begin{bmatrix}
2x_1^{(2)} & 0 & -\frac{1}{2} \\
-1 & -10x_2^{(2)} & 0 \\
-\frac{1}{2} & -\frac{6}{5} & 4x_3^{(2)}
\end{bmatrix}
=
\begin{bmatrix}
1 \\
-12 \\
2
\end{bmatrix}
$$
Using Maple to calculate these simple equations we find $x^{(2)}=(\frac{3}{4},\frac{11}{10},\frac{37}{40})$.\\
We can see the convergence towards the actual solution $(1,1,1)$
\\
Now we do two iteration of the Gauss Seidel method with the same initial guess. This means we need to solve
$$  
\begin{bmatrix}
2x_1^{(2)} & 0 & 0 \\
-2x_1^{(2)} & -10x_2^{(2)} & 0 \\
-1x_1^{(2)} & -1x_2^{(2)} & 4x_3^{(2)}
\end{bmatrix}
=
\begin{bmatrix}
1 \\
-12 \\
2
\end{bmatrix}
$$
Where $0$ has replaced the coordinates that are multiplied with coordinates from $x^{(0)}$. We find that $x^{(1)}=(\frac{1}{2},\frac{11}{10},\frac{9}{10})$. For the second iteration we need to solve the following
$$
\begin{bmatrix}
2x_1^{(2)} & 0x_2^{(1)} & -1x_3^{(1)} \\
-2x_1^{(2)} & -10x_2^{(2)} & 0x_3^{(1)} \\
-1x_1^{(2)} & -1x_2^{(2)} & 4x_3^{(2)}
\end{bmatrix}
=
\begin{bmatrix}
1 \\
-12 \\
2
\end{bmatrix}
$$
Rewritten to
$$
\begin{bmatrix}
2x_1^{(2)} & 0 & -\frac{9}{10} \\
-2x_1^{(2)} & -10x_2^{(2)} & 0 \\
-x_1^{(2)} & -1x_2^{(2)} & 4x_3^{(2)}
\end{bmatrix}
=
\begin{bmatrix}
1 \\
-12 \\
2
\end{bmatrix}
$$
Using Maple, we find that $x^{(2)}=(\frac{19}{20},\frac{101}{100},\frac{99}{100})$.\\
This is close to the actual solution $(1,1,1)$
\\
Finally, we want to do two iterations with the Conjugate gradient method. Our initial guess is the same. We start off by calculating the residual vector $r_0$ associated with $x_0$.
$$
r_0
=
b-Ax_0
=
\begin{bmatrix}
1 \\
-12 \\
2
\end{bmatrix}
-
\begin{bmatrix}
2 & 0 & -1 \\
-2 & -10 & 0 \\
-1 & -1 & 4
\end{bmatrix}
\begin{bmatrix}
0 \\
0 \\
0
\end{bmatrix}
=
\begin{bmatrix}
1 \\
-12 \\
2
\end{bmatrix}
$$
This is the first iteration, so we set our search direction $p_0=r_0$. Now we compute the scalar $\alpha_0$
$$
\alpha_0
=
\frac{r_0^Tr_0}{p_0^TAp_0}
=
\frac{
\begin{bmatrix}
1 & -12 & 2
\end{bmatrix}
\begin{bmatrix}
1 \\
-12 \\
2
\end{bmatrix}}{
\begin{bmatrix}
1 & -12 & 2
\end{bmatrix}
\begin{bmatrix}
2 & 0 & -1 \\
-2 & -10 & 0 \\
-1 & -1 & 4
\end{bmatrix}
\begin{bmatrix}
1 \\
-12 \\
2
\end{bmatrix}}
=
-\frac{149}{1378}
$$
This means we can compute $x_1$ as
$$
x_1=x_0+\alpha_0p_0
=
\begin{bmatrix}
0 \\
0 \\
0
\end{bmatrix}
-
\frac{149}{1378}
\begin{bmatrix}
1 \\
-12 \\
2
\end{bmatrix}
=
\begin{bmatrix}
-\frac{149}{1378} \\
\frac{894}{689} \\
-\frac{149}{689}
\end{bmatrix}
\approx
\begin{bmatrix}
-0.11 \\
1.30 \\
-0.22
\end{bmatrix}
$$
Now we need to do the second iteration. We start by computing our new residual vector $r_1$ by the following formula
$$
r_1=r_0-\alpha_0Ap_0=
\begin{bmatrix}
1 \\
-12 \\
2
\end{bmatrix}
+\frac{149}{1378}
\begin{bmatrix}
2 & 0 & -1 \\
-2 & -10 & 0 \\
-1 & -1 & 4
\end{bmatrix}
\begin{bmatrix}
1 \\
-12 \\
2
\end{bmatrix}
=
\begin{bmatrix}
1 \\
\frac{523}{689} \\
\frac{5587}{1378}
\end{bmatrix}
\approx
\begin{bmatrix}
1 \\
0.76 \\
4.05
\end{bmatrix}
$$
Now we need to compute the next scalar $\beta_0$ so we can later find $p_1$. This is done by the following formula
$$
\beta_0=\frac{r_1^Tr_1}{r_0^Tr_0}
=
\frac{
\begin{bmatrix}
1 & \frac{523}{689} & \frac{5587}{1378}
\end{bmatrix}
\begin{bmatrix}
1 \\
\frac{523}{689} \\
\frac{5587}{1378}
\end{bmatrix}}{
\begin{bmatrix}
1 & -12 & 2
\end{bmatrix}
\begin{bmatrix}
1 \\
-12 \\
2
\end{bmatrix}}
=
\frac{\frac{34207569}{1898884}}{149}
\approx
0.12
$$
This means we can now compute $p_1$ as
$$
p_1=r_1+\beta_0p_0=
\begin{bmatrix}
1 \\
\frac{523}{689} \\
\frac{5587}{1378}
\end{bmatrix}
+
\frac{\frac{34207569}{1898884}}{149}
\begin{bmatrix}
1 \\
-12 \\
2
\end{bmatrix}
=
\begin{bmatrix}
\frac{2128465}{1898884} \\
-\frac{328396}{474721} \\
\frac{2039512}{474721}
\end{bmatrix}
\approx
\begin{bmatrix}
1.12 \\
-0.69 \\
4.30
\end{bmatrix}
$$
Now that we found $p_1$ we can compute the new scalar $\alpha_1$ and then finally find $x_2$.
\begin{align*}
\alpha_1
&=
\frac{r_1^Tr_1}{p_1^TAp_1}
=
\frac{
\begin{bmatrix}
1 & \frac{523}{689} & \frac{5587}{1378}
\end{bmatrix}
\begin{bmatrix}
1 \\
\frac{523}{689} \\
\frac{5587}{1378}
\end{bmatrix}}{
\begin{bmatrix}
\frac{2128465}{1898884} & -\frac{328396}{474721} & \frac{2039512}{474721}
\end{bmatrix}
\begin{bmatrix}
2 & 0 & -1 \\
-2 & -10 & 0 \\
-1 & -1 & 4
\end{bmatrix}  
\begin{bmatrix}
\frac{2128465}{1898884} \\
-\frac{328396}{474721} \\
\frac{2039512}{474721}
\end{bmatrix}
}\\
&=
\frac{\frac{34207569}{1898884}}{\frac{173875590081}{2616662152}}
\approx
0.27
\end{align*}
And finally, we can find $x_2$ in the samme manner as $x_1$
$$
x_2=x_1+\alpha_1p_1
=
\begin{bmatrix}
-\frac{149}{1378} \\
\frac{894}{689} \\
-\frac{149}{689}
\end{bmatrix}
+
\frac{\frac{34207569}{1898884}}{\frac{173875590081}{2616662152}}
\begin{bmatrix}
\frac{2128465}{1898884} \\
-\frac{328396}{474721} \\
\frac{2039512}{474721}
\end{bmatrix}
=
\begin{bmatrix}
\frac{117367462}{599574001} \\
\frac{665523174}{599574001} \\
\frac{568673243}{599574001}
\end{bmatrix}
\approx
\begin{bmatrix}
0.20 \\
1.11 \\
0.95
\end{bmatrix}
$$
The fractions may be a little heavy to look at, but we need to work with exact arithmetics or the values may vary greatly.\\
We see that $x_2$ converges closer to $1$ in two of the coordinates but not the first one.
\end{document}
