\documentclass[a4paper, fleqn]{article}
\usepackage[utf8]{inputenc}
\usepackage{amsmath}
\usepackage{amssymb}
\usepackage{caption}
\usepackage{mathtools}
\usepackage{amsfonts}
\usepackage{lastpage}
\usepackage{tikz}
\usepackage{float}
\usepackage{textcomp}
\usetikzlibrary{patterns}
\usepackage{pdfpages}
\usepackage{gauss}
\usepackage{fancyvrb}
\usepackage[table]{colortbl}
\usepackage{fancyhdr}
\usepackage{graphicx}
\usepackage[margin=2.5 cm]{geometry}

\setlength\parindent{0pt}
\setlength\mathindent{75pt}

\definecolor{listinggray}{gray}{0.9}
\usepackage{listings}
\lstset{
	language=,
	literate=
		{æ}{{\ae}}1
		{ø}{{\o}}1
		{å}{{\aa}}1
		{Æ}{{\AE}}1
		{Ø}{{\O}}1
		{Å}{{\AA}}1,
	backgroundcolor=\color{listinggray},
	tabsize=3,
	rulecolor=,
	basicstyle=\scriptsize,
	upquote=true,
	aboveskip={0.2\baselineskip},
	columns=fixed,
	showstringspaces=false,
	extendedchars=true,
	breaklines=true,
	prebreak =\raisebox{0ex}[0ex][0ex]{\ensuremath{\hookleftarrow}},
	frame=single,
	showtabs=false,
	showspaces=false,
	showlines=true,
	showstringspaces=false,
	identifierstyle=\ttfamily,
	keywordstyle=\color[rgb]{0,0,1},
	commentstyle=\color[rgb]{0.133,0.545,0.133},
	stringstyle=\color[rgb]{0.627,0.126,0.941},
  moredelim=**[is][\color{blue}]{@}{@},
}

\lstdefinestyle{base}{
  emptylines=1,
  breaklines=true,
  basicstyle=\ttfamily\color{black},
}

\def\checkmark{\tikz\fill[scale=0.4](0,.35) -- (.25,0) -- (1,.7) -- (.25,.15) -- cycle;}
\newcommand*\circled[1]{\tikz[baseline=(char.base)]{
            \node[shape=circle,draw,inner sep=2pt] (char) {#1};}}
\newcommand*\squared[1]{%
  \tikz[baseline=(R.base)]\node[draw,rectangle,inner sep=0.5pt](R) {#1};\!}
\newcommand{\comment}[1]{%
  \text{\phantom{(#1)}} \tag{#1}}
\def\el{[\![}
\def\er{]\!]}
\def\dpip{|\!|}
\def\MeanN{\frac{1}{N}\sum^N_{n=1}}
\cfoot{Page \thepage\ of \pageref{LastPage}}
\DeclareGraphicsExtensions{.pdf,.png,.jpg}

\begin{document}

\section*{Question 1}
Define the relations
\begin{align*}
R &=
\begin{pmatrix}
  0 & 0 \\
  1 & 0
\end{pmatrix} \\
S &=
\begin{pmatrix}
  0 & 1 \\
  0 & 0
\end{pmatrix} 
\end{align*}
Then we have:
\begin{align*}
R\overline{S} &= 
\begin{pmatrix}
  1 & 0 \\
  0 & 0
\end{pmatrix} \\
R^T \cap \overline{RS} &=
\begin{pmatrix}
  0 & 1 \\
  0 & 0
\end{pmatrix} \cap
\begin{pmatrix}
  1 & 1 \\
  1 & 0
\end{pmatrix} \\
&=
\begin{pmatrix}
  0 & 1 \\
  0 & 0
\end{pmatrix} 
\end{align*}
Which means $R\overline{S}\neq R^T \cap \overline{RS}$ for these two relations.

\section*{Question 2}
TODO

\section*{Question 3}
The relation $R$ is:
\begin{itemize}
 \item Reflexive
 \item Symmetric
 \item Transitive
 \item Preorder (it is reflexive and transitive)
 \item Equivalence (it is reflexive, symmetric and transitive)
\end{itemize}
The relation $S$ is:
\begin{itemize}
 \item Reflexive
 \item Symmetric
 \item Transitive
 \item Antisymmetric
 \item Preorder (it is reflexive and transitive)
 \item Order (it is reflexive, transitive and antisymmetric)
 \item Equivalence (it is reflexive, symmetric and transitive)
 \item Partial identity
\end{itemize}

\section*{Question 4}
TODO

\section*{Question 5}
\subsection*{(1)}
Before we enter the while loop, we have:
\begin{align*}
  Q=\begin{pmatrix}
  1 \\
  1 \\
  1 \\
  1 \\
  1
  \end{pmatrix}
  S =\begin{pmatrix}
  1 \\
  0 \\
  1 \\
  1 \\
  1
  \end{pmatrix}
\end{align*}
After the first iteration:
\begin{align*}
  Q=\begin{pmatrix}
  1 \\
  0 \\
  1 \\
  1 \\
  1
  \end{pmatrix}
  S =\begin{pmatrix}
  1 \\
  0 \\
  1 \\
  0 \\
  1
  \end{pmatrix}
\end{align*}\\
After the second iteration:
\begin{align*}
  Q=\begin{pmatrix}
  1 \\
  0 \\
  1 \\
  0 \\
  1
  \end{pmatrix}
  S =\begin{pmatrix}
  1 \\
  0 \\
  0 \\
  0 \\
  0
  \end{pmatrix}
\end{align*}
After the third iteration:
\begin{align*}
  Q=\begin{pmatrix}
  1 \\
  0 \\
  0 \\
  0 \\
  0
  \end{pmatrix}
  S =\begin{pmatrix}
  0 \\
  0 \\
  0 \\
  0 \\
  0
  \end{pmatrix}
\end{align*}
After the fourth iteration:
\begin{align*}
  Q=\begin{pmatrix}
  0 \\
  0 \\
  0 \\
  0 \\
  0
  \end{pmatrix}
  S =\begin{pmatrix}
  0 \\
  0 \\
  0 \\
  0 \\
  0
  \end{pmatrix}
\end{align*}
Since they are equal, we break out of the while loops and the program returns \texttt{TRUE}.

\subsection*{(2)}
The relation $R$ with entries in $(1,2),(2,3),(3,4),(4,5)$ or defined as:
\begin{align*}
  R = \{(x,y)\ |\ (x,y)\in A \times A \land y=x+1\}
\end{align*}
for $A=\{1,2,3,4,5\}$ will make $5$ complete iterations. This is because they are not equal to begin with (as row $5$ is empty) and it will only make a row in $S$ empty if the $y$'th row is also empty. Therefore it will take remove the entry in row $4$, then row $3$ until it is empty. This takes $4$ iterations, but since $Q$ is an iteration behind, we will need a fifth to make them equal.

\section*{Question 6}


\end{document}
