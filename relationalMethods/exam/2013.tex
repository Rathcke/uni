\documentclass[a4paper, fleqn]{article}
\usepackage[utf8]{inputenc}
\usepackage{amsmath}
\usepackage{amssymb}
\usepackage{caption}
\usepackage{mathtools}
\usepackage{amsfonts}
\usepackage{lastpage}
\usepackage{tikz}
\usepackage{float}
\usepackage{textcomp}
\usetikzlibrary{patterns}
\usepackage{pdfpages}
\usepackage{gauss}
\usepackage{fancyvrb}
\usepackage[table]{colortbl}
\usepackage{fancyhdr}
\usepackage{graphicx}
\usepackage[margin=2.5 cm]{geometry}

\setlength\parindent{0pt}
\setlength\mathindent{75pt}

\definecolor{listinggray}{gray}{0.9}
\usepackage{listings}
\lstset{
	language=,
	literate=
		{æ}{{\ae}}1
		{ø}{{\o}}1
		{å}{{\aa}}1
		{Æ}{{\AE}}1
		{Ø}{{\O}}1
		{Å}{{\AA}}1,
	backgroundcolor=\color{listinggray},
	tabsize=3,
	rulecolor=,
	basicstyle=\scriptsize,
	upquote=true,
	aboveskip={0.2\baselineskip},
	columns=fixed,
	showstringspaces=false,
	extendedchars=true,
	breaklines=true,
	prebreak =\raisebox{0ex}[0ex][0ex]{\ensuremath{\hookleftarrow}},
	frame=single,
	showtabs=false,
	showspaces=false,
	showlines=true,
	showstringspaces=false,
	identifierstyle=\ttfamily,
	keywordstyle=\color[rgb]{0,0,1},
	commentstyle=\color[rgb]{0.133,0.545,0.133},
	stringstyle=\color[rgb]{0.627,0.126,0.941},
  moredelim=**[is][\color{blue}]{@}{@},
}

\lstdefinestyle{base}{
  emptylines=1,
  breaklines=true,
  basicstyle=\ttfamily\color{black},
}

\def\checkmark{\tikz\fill[scale=0.4](0,.35) -- (.25,0) -- (1,.7) -- (.25,.15) -- cycle;}
\newcommand*\circled[1]{\tikz[baseline=(char.base)]{
            \node[shape=circle,draw,inner sep=2pt] (char) {#1};}}
\newcommand*\squared[1]{%
  \tikz[baseline=(R.base)]\node[draw,rectangle,inner sep=0.5pt](R) {#1};\!}
\newcommand{\comment}[1]{%
  \text{\phantom{(#1)}} \tag{#1}}
\def\el{[\![}
\def\er{]\!]}
\def\dpip{|\!|}
\def\MeanN{\frac{1}{N}\sum^N_{n=1}}
\cfoot{Page \thepage\ of \pageref{LastPage}}
\DeclareGraphicsExtensions{.pdf,.png,.jpg}

\begin{document}


\section*{Question 3}
The relation $P$ is:
\begin{itemize}
 \item Reflexive
 \item Transitive
 \item Antisymmetric
 \item Preorder (it is reflexive and transitive)
 \item Order (it is reflexive, transitive and antisymmetric)
\end{itemize}

\section*{Question 5}
\subsection*{(1)}
Before we enter the while loop, we have:
\begin{align*}
  Q=\begin{pmatrix}
  0 \\
  0 \\
  1 \\
  0 \\
  0
  \end{pmatrix}
  P =\begin{pmatrix}
  0 \\
  0 \\
  0 \\
  1 \\
  0
  \end{pmatrix}
\end{align*}
After the first iteration:
\begin{align*}
  Q=\begin{pmatrix}
  0 \\
  0 \\
  1 \\
  1 \\
  0
  \end{pmatrix}
  P =\begin{pmatrix}
  0 \\
  1 \\
  0 \\
  0 \\
  1
  \end{pmatrix}
\end{align*}
After the second iteration:
\begin{align*}
  Q=\begin{pmatrix}
  0 \\
  1 \\
  1 \\
  1 \\
  1
  \end{pmatrix}
  P =\begin{pmatrix}
  0 \\
  0 \\
  0 \\
  0 \\
  0
  \end{pmatrix}
\end{align*}
Now, $P$ is empty, so we break out of the loop and return $Q$. We can reach nodes
$2,3,4$ and $5$ from node $3$.

\subsection*{(2)}
We can transform $S$ into a vector by calculating \texttt{SL}. That way, the program runs
as usual, but since we need to return a partial identity, we return the intersection of
\texttt{Q} with the identity matrix instead. This yields the following program:
\begin{verbatim}
Reachable(R,S)
  DECL P,Q
  BEG
    Q = SL
    P = -Q & R^ * Q
    WHILE -empty(P) DO
      Q = Q | P
      P = -Q & R^ * P
    OD
    RETURN Q & I
  END.
\end{verbatim}

\end{document}
