\documentclass[a4paper, fleqn]{article}
\usepackage[utf8]{inputenc}
\usepackage{amsmath}
\usepackage{amssymb}
\usepackage{caption}
\usepackage{mathtools}
\usepackage{amsfonts}
\usepackage{lastpage}
\usepackage{tikz}
\usepackage{float}
\usepackage{textcomp}
\usetikzlibrary{patterns}
\usepackage{pdfpages}
\usepackage{gauss}
\usepackage{fancyvrb}
\usepackage[table]{colortbl}
\usepackage{fancyhdr}
\usepackage{graphicx}
\usepackage[margin=2.5 cm]{geometry}

\setlength\parindent{0pt}
\setlength\mathindent{75pt}

\definecolor{listinggray}{gray}{0.9}
\usepackage{listings}
\lstset{
	language=,
	literate=
		{æ}{{\ae}}1
		{ø}{{\o}}1
		{å}{{\aa}}1
		{Æ}{{\AE}}1
		{Ø}{{\O}}1
		{Å}{{\AA}}1,
	backgroundcolor=\color{listinggray},
	tabsize=3,
	rulecolor=,
	basicstyle=\scriptsize,
	upquote=true,
	aboveskip={0.2\baselineskip},
	columns=fixed,
	showstringspaces=false,
	extendedchars=true,
	breaklines=true,
	prebreak =\raisebox{0ex}[0ex][0ex]{\ensuremath{\hookleftarrow}},
	frame=single,
	showtabs=false,
	showspaces=false,
	showlines=true,
	showstringspaces=false,
	identifierstyle=\ttfamily,
	keywordstyle=\color[rgb]{0,0,1},
	commentstyle=\color[rgb]{0.133,0.545,0.133},
	stringstyle=\color[rgb]{0.627,0.126,0.941},
  moredelim=**[is][\color{blue}]{@}{@},
}

\lstdefinestyle{base}{
  emptylines=1,
  breaklines=true,
  basicstyle=\ttfamily\color{black},
}

\pagestyle{fancy}
\def\checkmark{\tikz\fill[scale=0.4](0,.35) -- (.25,0) -- (1,.7) -- (.25,.15) -- cycle;}
\newcommand*\circled[1]{\tikz[baseline=(char.base)]{
            \node[shape=circle,draw,inner sep=2pt] (char) {#1};}}
\newcommand*\squared[1]{%
  \tikz[baseline=(R.base)]\node[draw,rectangle,inner sep=0.5pt](R) {#1};\!}
\newcommand{\comment}[1]{%
  \text{\phantom{(#1)}} \tag{#1}}
\def\el{[\![}
\def\er{]\!]}
\def\dpip{|\!|}
\def\MeanN{\frac{1}{N}\sum^N_{n=1}}
\cfoot{Page \thepage\ of \pageref{LastPage}}
\DeclareGraphicsExtensions{.pdf,.png,.jpg}

\author{Nikolaj Dybdahl Rathcke (Student ID: 74763954)}
\title{Relational Methods \\ Assignment 2}
\lhead{Relational Methods}
\rhead{Assignment 2}

\begin{document}
\maketitle

\section*{Question 3}
\subsection*{(a)}
Then we can express the relation $R$ as:
\begin{align*}
  R&=\left((P\cap D)^T\cap \overline{\left(P\cap D\right)^T\overline{I}}\right)L
\end{align*}
We probably need some explanation for that. Everything in the parentheses is just the
function $\mbox{up}(S)$, where $S=(P\cap D)^T$. The relation $S$ is then a relation that
has an entry (set to $1$) in $(x,x)$ if $x$ is prime ($p^1$) and no entries in $(x,y)$
for $x\neq y$. \\
For any $x$ which is not prime, $S$ has entries in $(x,y)$ for prime factors $y_i$. Now, if a number is a prime power, it has only one prime factor, so using the
\texttt{up} function will keep an entry in that row if it is a prime power and remove all
entries if it not. Finally, taking the composition with $L$ will fill out the rows with an
entry, so it is represented as a set. \\
Note that the special cases for $x=0$ and $x=1$ are taken care of as these have no entries in
$P\cap D$.

\subsection*{(b)}
For all prime powers $x$, we simply have an entry in $(x,x)$ by:
\begin{align*}
  F_1&=I\cap R
\end{align*}
For all non prime powers, we can get its divisors that are prime powers by:
\begin{align*}
  F_2=R\cap (D\cap \overline{I})
\end{align*}
Now we need to remove the entries $(x,z)\in F_2$ for all pairs of entries $(x,z),(y,z)\in
F_2$ that satisfy $1=(x,y)\in (D\cap \overline{I})$. Say a function $F_3$ does that, then
the relation is given by:
\begin{align*}
  F &= F_1\cup F_3(F_2)
\end{align*}
While I think you need to use direct products to construct $F_3$, I could not immediately
make it work.

\section*{Question 4}
\subsection*{(c)}
We can express $W$ by the simple relational definition:
\begin{align*}
  W = U\cap V
\end{align*}
This is easily seen as a kernel needs to be both absorbent and stable. So a set needs to
satisfy that $S\subseteq P$ and $S\subseteq Q$. The sets that satisfy this is exactly
those in $U\cap V$.

\end{document}
