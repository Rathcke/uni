\documentclass[a4paper, fleqn]{article}
\usepackage[utf8]{inputenc}
\usepackage{amsmath}
\usepackage{amssymb}
\usepackage{caption}
\usepackage{mathtools}
\usepackage{amsfonts}
\usepackage{lastpage}
\usepackage{tikz}
\usepackage{float}
\usepackage{textcomp}
\usetikzlibrary{patterns}
\usepackage{pdfpages}
\usepackage{kbordermatrix}
\usepackage{gauss}
\usepackage{fancyvrb}
\usepackage[table]{colortbl}
\usepackage{fancyhdr}
\usepackage{graphicx}
\usepackage[margin=2.5 cm]{geometry}

\setlength\parindent{0pt}
\setlength\mathindent{75pt}

\definecolor{listinggray}{gray}{0.9}
\usepackage{listings}
\lstset{
	language=,
	literate=
		{æ}{{\ae}}1
		{ø}{{\o}}1
		{å}{{\aa}}1
		{Æ}{{\AE}}1
		{Ø}{{\O}}1
		{Å}{{\AA}}1,
	backgroundcolor=\color{listinggray},
	tabsize=3,
	rulecolor=,
	basicstyle=\scriptsize,
	upquote=true,
	aboveskip={0.2\baselineskip},
	columns=fixed,
	showstringspaces=false,
	extendedchars=true,
	breaklines=true,
	prebreak =\raisebox{0ex}[0ex][0ex]{\ensuremath{\hookleftarrow}},
	frame=single,
	showtabs=false,
	showspaces=false,
	showlines=true,
	showstringspaces=false,
	identifierstyle=\ttfamily,
	keywordstyle=\color[rgb]{0,0,1},
	commentstyle=\color[rgb]{0.133,0.545,0.133},
	stringstyle=\color[rgb]{0.627,0.126,0.941},
  moredelim=**[is][\color{blue}]{@}{@},
}

\lstdefinestyle{base}{
  emptylines=1,
  breaklines=true,
  basicstyle=\ttfamily\color{black},
}

\pagestyle{fancy}
\def\checkmark{\tikz\fill[scale=0.4](0,.35) -- (.25,0) -- (1,.7) -- (.25,.15) -- cycle;}
\newcommand*\circled[1]{\tikz[baseline=(char.base)]{
            \node[shape=circle,draw,inner sep=2pt] (char) {#1};}}
\newcommand*\squared[1]{%
  \tikz[baseline=(R.base)]\node[draw,rectangle,inner sep=0.5pt](R) {#1};\!}
\newcommand{\comment}[1]{%
  \text{\phantom{(#1)}} \tag{#1}}
\def\el{[\![}
\def\er{]\!]}
\def\dpip{|\!|}
\def\MeanN{\frac{1}{N}\sum^N_{n=1}}
\cfoot{Page \thepage\ of \pageref{LastPage}}
\DeclareGraphicsExtensions{.pdf,.png,.jpg}

\author{Nikolaj Dybdahl Rathcke (Student ID: 74763954)}
\title{Combinatorics (MATH429) \\ Assignment 4}
\lhead{Combinatorics (MATH429)}
\rhead{Assignment 4}

\begin{document}
\maketitle

\section*{Question 1}
We know that $\{1,5,6\}$ is a basis $B$ and represented as a vector matroid:
\begin{align*}
  B =
  \kbordermatrix{
    & 1 & 5 & 6 \\
    & 1 & 0 & 0 \\
    & 0 & 1 & 0 \\
    & 0 & 0 & 1
  }
\end{align*}
Now we let non-Pappus matroid be represented by the following matrix $A$:
\begin{align*}
  A=\kbordermatrix{
    & 2 & 3 & 4 & 7 & 8 & 9 \\
    1 & 1 & 1 & 0 & 1 & 1 & 1 \\
    5 & 1 & x_1 & 1 & x_2 & 0 & x_3 \\
    6 & 1 & x_4 & x_5 & 0 & x_6 & x_7
  }
\end{align*}
Since $\{2,6,9\}$ is a basis, we know its determinant is $0$, so we can work out what
$x_3$ should be in the position by calculating the $3$ by $3$ determinant and equaling it to zero:
\begin{align*}
  1\cdot(0\cdot x_7-1\cdot x_3)-0\cdot(1\cdot x_7-1\cdot x_3)+1\cdot(1\cdot 1-1\cdot 0)=0
  \\
  \Rightarrow -x_3+1=0
\end{align*}
So $x_3=1$. \\
Next, we look at $x_1$ and $x_4$. Since $\{1,2,3\}$ is not a basis. In
the same way, we can isolate $x_3$ and $x_4$ by looking at the determinant:
\begin{align*}
  x_4-x_1=0
\end{align*}
So $x_1=x_4$. \\
We can now look at $x_7$. We know $\{3,5,9\}$ is not a basis, which gives us:
\begin{align*}
  x_7-x_4=0
\end{align*}
So $x_1=x_4=x_7$. \\
Now $\{3,4,8\}$ is not a basis, so we get:
\begin{align*}
  x_6+(x_4-x_1x_5)&=0 \ \ \ \ \Leftrightarrow \\
  x_6&=x_4-x_1x_5 \\
     &=x_1(1-x_5)
\end{align*}
Since $x_1=x_4$. Lastly, if we look at the circuit $\{2,4,7\}$:
\begin{align*}
  -(x_5x_2)+(x_5-1)&=0 \ \ \ \ \ \ \Leftrightarrow \\
  x_5-1&=x_5x_2 \ \ \Rightarrow \\
  x_5&\neq 0 \ \ \ \ \ \ \ \Rightarrow \\
  x_2&=\frac{x_5-1}{x_5} \comment{Divide by $x_5$}
\end{align*}
We can now conclude that $\{7,8,9\}$ is not a basis as:
\begin{align*}
  -(x_6x_3)-(x_2x_7)+(x_2x_6) &= x_2x_6-x_6-x_2x_7 \\
                              &= \frac{x_5-1}{x_5}x_6-x_6-\frac{x_5-1}{x_5}x_7
  \comment{Substitute $x_2$} \\
  &= \frac{x_5-1}{x_5}x_1(1-x_5)-x_1(1-x_5)-\frac{x_5-1}{x_5}x_7
  \comment{Substitute $x_6$} \\
  &=\frac{x_5(x_1-x_1x_5)-x_1(1-x_5)}{x_5}-x_1(1-x_5)-\frac{x_5-1}{x_5}x_7 \\
  &=\frac{-x_1(1-x_5)-(x_5x_7-x_7)}{x_5} \\
  &=\frac{-x_1+x_1x_5-x_5x_7+x_7}{x_5} \\
  &=x_1-x_7-\frac{x_7-x_1}{x_5} \\
  &= 0 \comment{Since $x_1=x_7$}
\end{align*}
as the determinant is $0$. So if the non-Pappus was representable over a field, then
$\{7,8,9\}$ would not be a basis, but as we have just shown that it is not.

\section*{Question 2}
The set $\{1,2,4\}$ is a basis, so:
\begin{align*}
  B =
  \kbordermatrix{
    & 1 & 2 & 4 \\
    & 1 & 0 & 0 \\
    & 0 & 1 & 0 \\
    & 0 & 0 & 1
  }
\end{align*}
We can represent $Q_6$ by the following matrix:
\begin{align*}
  A=\kbordermatrix{
    &   3 & 5 & 6 \\
    1 & 1 & 1 & 1 \\
    2 & x_1 & 1 & x_2 \\
    4 & 0 & 1 & x_3
  }
\end{align*}
We know $\{3,4,5\}$ is not a basis, so as before, we can isolate $x_1$:
\begin{align*}
  -1+x_1&=0 \Leftrightarrow \\
  x_1&=1
\end{align*}
Any set with of size $3$ that include the element $\{6\}$ must be a basis, so the
determinant is non-zero. \\
For the determinant of $\{1,2,6\}$, we only get the term $x_3$, so $x_3\neq 0$. \\
Another lucky choice of set that gives us some information is $\{2,5,6\}$:
\begin{align*}
  -(x_3)+1&\neq 0 \Leftrightarrow \\
  x_3 &\neq 1
\end{align*}
The determinant of $\{1,4,6\}$ will give us:
\begin{align*}
  -x_2 \neq 0
\end{align*}
So $x_2\neq 0$. For the determinant of $\{3,4,6\}$, we get:
\begin{align*}
  -x_2+1&\neq 0 \Leftrightarrow \\
  x_2&\neq 1
\end{align*}
If we look at the set $\{1,5,6\}$, we have:
\begin{align*}
  x_3-x_2 \neq 0 \Leftrightarrow \\
  x_2 \neq x_3
\end{align*}
Now since $x_1,x_2\neq \{0,1\}$ and they are not equal each other, we need a field that
has $4$ or more elements to represent the matroid $Q_6$.

\section*{Question 3}
A hyperplane is a flat with rank $r(M)-1$, that is, we cannot add any element to the
hyperplane that increases its rank. So if $X$ is a hyperplane, we know it has rank
$r(M)-1$. By Preposition 7.5(i) in the lecture notes, a set $X$ is spanning if
$r(X)=r(M)$, meaning the hyperplane must be a maximal non-spanning set. \\
If $X$ is a maximal non-spanning set, meaning if we add an element to the set, we must
increases the rank of the set (as otherwise it's not maximal) and it will become a
spanning set. Therefore the maximal non-spanning set must have rank $r(M)-1$ where we
cannot add any element that increases its rank - which is exactly what a hyperplane is.

\section*{Question 4}
\subsection*{(a)}

\subsection*{(b)}
For a spanning set to be a unique, we need it to be a unique basis as well. We also need
the basis to have rank equal to the size of the ground set. So a spanning set is unique
in a matroid with ground set $E$, if the matroid has a unique basis $B$ so that $|B|=|E|$

\section*{Question 7}

\section*{Question 8}



\end{document}
