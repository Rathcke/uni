\documentclass[a4paper, fleqn]{article}
\usepackage[utf8]{inputenc}
\usepackage{amsmath}
\usepackage{amssymb}
\usepackage{caption}
\usepackage{mathtools}
\usepackage{amsfonts}
\usepackage{lastpage}
\usepackage{tikz}
\usepackage{float}
\usepackage{textcomp}
\usetikzlibrary{patterns}
\usepackage{pdfpages}
\usepackage{kbordermatrix}
\usepackage{gauss}
\usepackage{fancyvrb}
\usepackage[table]{colortbl}
\usepackage{fancyhdr}
\usepackage{graphicx}
\usepackage[margin=2.5 cm]{geometry}

\setlength\parindent{0pt}
\setlength\mathindent{75pt}

\definecolor{listinggray}{gray}{0.9}
\usepackage{listings}
\lstset{
	language=,
	literate=
		{æ}{{\ae}}1
		{ø}{{\o}}1
		{å}{{\aa}}1
		{Æ}{{\AE}}1
		{Ø}{{\O}}1
		{Å}{{\AA}}1,
	backgroundcolor=\color{listinggray},
	tabsize=3,
	rulecolor=,
	basicstyle=\scriptsize,
	upquote=true,
	aboveskip={0.2\baselineskip},
	columns=fixed,
	showstringspaces=false,
	extendedchars=true,
	breaklines=true,
	prebreak =\raisebox{0ex}[0ex][0ex]{\ensuremath{\hookleftarrow}},
	frame=single,
	showtabs=false,
	showspaces=false,
	showlines=true,
	showstringspaces=false,
	identifierstyle=\ttfamily,
	keywordstyle=\color[rgb]{0,0,1},
	commentstyle=\color[rgb]{0.133,0.545,0.133},
	stringstyle=\color[rgb]{0.627,0.126,0.941},
  moredelim=**[is][\color{blue}]{@}{@},
}

\lstdefinestyle{base}{
  emptylines=1,
  breaklines=true,
  basicstyle=\ttfamily\color{black},
}

\pagestyle{fancy}
\def\checkmark{\tikz\fill[scale=0.4](0,.35) -- (.25,0) -- (1,.7) -- (.25,.15) -- cycle;}
\newcommand*\circled[1]{\tikz[baseline=(char.base)]{
            \node[shape=circle,draw,inner sep=2pt] (char) {#1};}}
\newcommand*\squared[1]{%
  \tikz[baseline=(R.base)]\node[draw,rectangle,inner sep=0.5pt](R) {#1};\!}
\newcommand{\comment}[1]{%
  \text{\phantom{(#1)}} \tag{#1}}
\def\el{[\![}
\def\er{]\!]}
\def\dpip{|\!|}
\def\MeanN{\frac{1}{N}\sum^N_{n=1}}
\cfoot{Page \thepage\ of \pageref{LastPage}}
\DeclareGraphicsExtensions{.pdf,.png,.jpg}

\author{Nikolaj Dybdahl Rathcke (Student ID: 74763954)}
\title{Combinatorics (MATH429) \\ Assignment 5}
\lhead{Combinatorics (MATH429)}
\rhead{Assignment 5}

\begin{document}
\maketitle

\end{document}
\section*{Question 1}
For contradiction, assume we have two disjoint circuits in $M^*$ when for all pairs of
hyperplanes $H_1$ and $H_2$ in $M$, then $H_1 \cup H_2 \neq
E$. This means there is an element $e\in E$, which is in the cocircuits $C_1^*$ and $C_2^*$,
who are the complements of $H_1$ and $H_2$ respectively. This is a contradiction, so to
have two disjoint circuits in $M^*$, we must have two hyperplanes $H_1$ and $H_2$ in $M$
where $H_1\cup H_2=E$. \\
For the other way, assume we have two hyperplanes $H_1$ and $H_2$ so that $H_1\cup H_2 =
E$, then for any element $e\in E$, either $e\in H_1$ or $e\in H_2$ or $e\in H_1,H_2$.
If an element $f$ is in two cocircuits, it means that $f\neq\in H_1, H_2$, but no such
element exists, therefore we must have two cocircuits are disjoint.

\section*{Question 2}
If $X$ is independent and not a basis in $M$, then there is a set of elements $A\in E$, so that
$X\cup A\subseteq \mathcal{B}(M)$. This implies that the set $E(M)-X-e$ cannot be
independent in $M^*$ as it not disjoint from the basis $X\cup A$ in $M$. \\
Likewise, if $E(M)-X$ is independent but not a cobasis, then there is a set of elements
$A$, so that $E(M)-X+A$ is independent, so $E(M)-X+A\subseteq \mathcal{B}(M^*)$, so $X$ cannot be independent. \\
Therefore $X$ has to be a basis in $M$ and $E(M)-X$ has to be a cobasis.

\section*{Question 4}
\subsection*{(a)}
It is both necessary and sufficient. Obviously it is necessary since if the bases are not
in both $M$ and $M^*$, then $M\neq M^*$. It is sufficient, since if all the bases are
equal to the cobases, then all independent sets are the same as they are all the proper
subset of the bases (and cobases).

\subsection*{(b)}
It is necessary, but not sufficient. It is trivial to see that it is necessary if the
matroid is identically self-dual. However, to see it is not sufficient, consider the
following simple counter example.\\
Let $M$ be the matroid on $E=\{1,2\}$ with $\mathcal{I}(M)=\{\emptyset,\{1\}\}$. This
means the dual matroid has $\mathcal{I}(M^*)=\{\emptyset,\{2\}\}$. Now the two matroids
have the same set of flats, namely the empty set, but they are not the same.

\subsection*{(c)}
It is both necessary and sufficient. If the circuits and cocircuits are the same, then
the bases and cobases must be the same. The necessity and sufficiency then follows from the
same argumentation as in (a).

\section*{Question 6}
For a matroid $M$ represented by matrix $A=[I_r|D]$, then the dual matroid $M^*$ is
represented by the matrix $B=[-D^T|I_{n-r}]$. Since $M[A]$ contains no coloop, that means
$-D^T$ does not contain any zero columns. When $D^T$ is transposed, we then have that
there are no non-zero rows in $(D^T)^T=D$. Obviously, the identity matrix $I_r$ contains
no zero rows either, meaning $A=[I_r|D]$ must contain at least two non-zero entries in
each row.

\section*{Question 7}
If a matroid $M$ is $\mathbb{F}$-representable, then Theorem 8.4 tells us that the dual
matroid $M^*$ is also $\mathbb{F}$-representable. \\
The dual matroid of $U_{n,n+2}$ is the
matroid $U_{2,n+2}$. This is exactly the $n+2$ line, which is only representable over
fields $\mathbb{F}\geq n+1$.

\end{document}
