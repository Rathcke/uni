\documentclass[12pt]{article}
\usepackage[utf8]{inputenc}
\usepackage{amsmath}
\usepackage{amssymb}
\usepackage{mathtools}
\usepackage{amsfonts}
\usepackage{lastpage}
\usepackage{tikz}
\usepackage{pdfpages}
\usepackage{gauss}
\usepackage{fancyvrb}
\usepackage{fancyhdr}
\usepackage{graphicx}
\pagestyle{fancy}
\fancyfoot[C]{\footnotesize Page \thepage\ of 4}
\DeclareGraphicsExtensions{.pdf,.png,.jpg}
\title{Anden individuelle aflevering}
\author{Nikolaj Dybdahl Rathcke - 180692 \\ rfq695 \\\\ Antal ord: 971}
\lhead{Nikolaj Dybdahl Rathcke (rfq695)}
\rhead{Datalogiens Videnskabsteori}

\begin{document}
\maketitle

\section*{Introduktion}
Med fremgangen i teknologien er det nu muligt at effektivisere en masse processer og spare en masse resurser. Dette er selvfølgelig godt for velfærden, men samtidig tager den digitale velfærd også en række ulemper samt spørgsmål med sig.\\
Vi kigger på hvilke overvejelser der skal gøres for at den digitale velfærd kan blive taget imod på en social acceptabel måde.

\section*{1: "Digital Velfærd 2013" - Beskrivelse}
Debatten handler om hvorvidt danskere er klar til at imødekomme den teknologiske fremgang og bruge den til at modernisere vores velfærdssamfund - skabe "digital velfærd". \\
Den digitale velfærd lægger stærkt op til at den enkelte borger skal sørge for sig selv. Ment i den forstand, at mange flere opgaver, sundheds-, social- og undervisningsmæssige, er i hænderne på danskeren selv. Der er nemlig meget at spare hvis for eksempel antallet af lægebesøg kan mindskes ved at danskeren står for at undersøge sig selv.\\
Men på trods af at digitalisering af mange ting er godt og at mange velfærdsydelser allerede er digitaliseret, er der også konsekvenser og spørgsmål. Debatten snakker om at digtale løsninger ikke altid ville kunne betale sig, at der er en grænse på hvor mange digitale muligheder der er.\\
En anden ulempe der diskuteres er, at private informationer ligger frit tilgængelige. For eksempel kan et samspil mellem de offentlige instanser sørge for at borgeren ikke skulle give sine oplysninger flere gange - dette sparer tid og penge. Men det er ikke alle der kan lide at deres information ligger digitalt og tilgængeligt. Som nævnt i debatten på side 10: "Hvor går grænsen mellem hensynet til den enkelte og hensynet til det fælles gode?"\\
Til sidst nævnes også hvordan den digitale velfærd kan bruges på tværs af landegrænser, som på den måde kan lette arbejdet for blandt andet læger i andre lande. \\
Ansvaret hos borgeren selv er et gennemgående tema i hele debatten, men for at det kan lade sig gøre kræves det også at borgeren har resurserne til det. Altså at man blandt andet skal have computer med internetadgang.

\section*{2: "Digital Velfærd 2013" - Analyse og Diskussion}
Hvis vi vender tilbage det gennemgående tema, om at danskeren skal sørge for sig selv, må vi huske på at der en masse arbejdspladser som beskæftiger sig med netop hvad digitaliseringen er igang med at erstatte. Vi skal sørge for, at vi ikke bare fjerner arbejdspladser, fordi en lang række processer kan blive \textit{automatiseret}, men at vi derimod bruger teknologien som et værktøj for at \textit{effektivisere} processerne istedet. \\
Som der nævnes i [2] i afsnit 2, skal vi foretage nogle omstillinger og omskolinger for at kunne gennemføre ændringerne på en tryg måde. Dette var selvfølgelig meget mere aktuelt i 1980, hvor vi i dag allerede har omstilelt os. Men det er stadig vigtigt at bruge de menneskelige resurser - vi skal bare bruge dem til andre opgaver.\\
Selve debatten virker desuden også meget afsluttet. Forstået på den måde, at der er mange gode pointer, men de fleste af overvejelserne er allerede taget for en og så kan det være svært at have en helt anden opfattelse. På den måde er debatten også en modsrid med det demokratiske aspekt Koch-Jakobsen har givet udtryk for i [2]. Nemlig, at den enkelte borger har muligheden for at være med i debatten. Debatten kan derfor styrkes ved at stræbe efter at skabe mere disussion istedet for at være mere målrettet mod en afrunding. Dette er trods alt hvad meningen med et debatoplæg er.

\section*{3: The Computing Profession - Diskussion}
I artiklen, [3], snakkes der meget om de forskellige "professionelle roller" der er i samfundet. Herunder ingeniør rollen, som datalogi studerende hovedsageligt er en del af, som er den "skabende" person.\\
Artiklen prøver at overbevise om, at modsat det traditionelle syn med "Mekanistiske syn mod den romantiske syn"\footnote{[3], side 81}, hvor rollerne var opdelt i hvem der skabte redskaberne og hvem der sørgede for de blev brugt på socialt acceptable måder, skal datalogen altså være begge dele.\\
Man skal som datalog have en forståelse for hvad det indebærer når der udvikles ny teknologi - være en del af alle de profesionnelle roller\footnote{[3], side 85} med et fokus på det store billede.\\
Jeg kan som datalogi studerende godt se mig selv være indblandet i nogle af de andre roller. Men jeg kan ikke se mig selv som en beslutningstager om hvordan teknologien skal bruges, men derimod kan jeg bruges som en formidler af et emne. Jeg kan beskrive et emne på en måde borgerne kan forstå og derved lægge et fundament for debatten om hvordan det kan bruges til at lave digital velfærd.\\
Jeg kan herefter springe tilbage i ingeniør rollen og hjælpe med at udvikle den teknologi med henblik på hvad borgerne er blevet enige om.\\
Dette er hvordan jeg ser datalogen kunne bruges på en "romantisk" måde, men jeg mener dog at der ikke skal være for meget indblanding fra en datalogs side og at der hovedsageligt stadig skal fokuseres på at være "mekanisten".

\section*{Konklusion}
Det kan være svært at forudsige alle konsekvenser der kommer af digitaliseringen. Selvfølgelig skal vi passe på vi for eksempel ikke kommer til at tvinge borgere ud af arbejdsmarkedet, men samtidig skal vi omfavne alle de fordele der kommer af det. Der er nemlig skabt nye muligheder for en effektivisering af stort set alle områder og en masse processer kan endda blive automatiseret. Vi skal dog være i stand til at overveje hvilke konsekvenser der kan være af det. Dette er ikke nødvendigvis kun ingeniøren der skal lave disse overvejelse, men alle borgere skal kunne bidrage så der kan findes løsninger som samfundet, som helhed, får det bedste ud af.

\section*{Referencer}
[1] Digital Velfærd, fælles offentlig debatoplæg, 2013 \\
$ $
[2] Dansk IT-politik, Svend Jakobsen, 1980 \\
$ $
[3] The Computing Profession, Dahlbom og Mathiassen, 1997


\end{document}
