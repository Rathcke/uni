\documentclass[12pt]{article}
\usepackage[utf8]{inputenc}
\usepackage{amsmath}
\usepackage{amssymb}
\usepackage{mathtools}
\usepackage{amsfonts}
\usepackage{lastpage}
\usepackage{tikz}
\usepackage{pdfpages}
\usepackage{gauss}
\usepackage{fancyvrb}
\usepackage{fancyhdr}
\usepackage{graphicx}
\pagestyle{fancy}
\fancyfoot[C]{\footnotesize Page \thepage\ of 4}
\DeclareGraphicsExtensions{.pdf,.png,.jpg}
\title{Tredje individuelle aflevering}
\author{Nikolaj Dybdahl Rathcke - 180692 \\ rfq695 \\\\ Antal ord: 1069}
\lhead{Nikolaj Dybdahl Rathcke (rfq695)}
\rhead{Datalogiens Videnskabsteori}

\begin{document}
\maketitle

\section*{Spørgsmål 1}
Ophavsretsloven eksisterer for at beskytte værker, så spredning af værket ved såkaldt eksemplarfremstilling ikke er lovligt uden ophavsmanden samtykke, hvor samtykket ofte opnås imod betaling.\\
Rettighedshaveren har altså eneret og andre skal derfor have tilladelse af denne til at bruge værket i forskellige sammenhænge. Skulle nogen bruge et værk uden tilladelse kan rettighedshaveren forlange erstatninger eller også kan der være tale om andre straffe.\\
Der er dog begrænsninger på hvorvidt man skal have en tilladelse. For eksempel forsvinder rettighederne for et værk 70 år efter ophavsmanden er død. Der er også andre scenarier hvor et værk kan bruges uden samtykke fra ophavsmanden (for at undgå absurde tilfælde), og det er disse der er gennemgået i ophavsretsloven.\\
For en datalog kan dette bruges til at beskytte ens programmer som er udarbejdet selvstændigt. Dette er vigtigt for at en datalog kan tjene penge på sit arbejde. Desuden beskytter det også ideerne bag et program.\\
Det betyder til gengæld også at man skal være påpasselig når man "låner" dele af andres kildekode for ens eget program. Desuden skal man også være opmærksom på at det ikke altid er en selv der har rettighederne hvis man er under en arbejdsgiver medmindre at man indgår særlige aftaler.

\section*{Spørgsmål 2}
På trods af at det reelt set er en eksemplarfremstilling og selvom sønnen skulle være overdraget rettighederne, har han ikke ret i sin påstand - og derved heller ikke erstatningskrav. \\
Dette skyldes, at der ifølge [1, s. 124-125] kan bruges § $24$ af ophavsretsloven. Stykke $2$ af denne paragraf siger
\begin{quote}
\textit{"Kunstværker må afbildes, når de er varigt anbragt på eller ved en for almenheden tilgængelig plads eller vej. Bestemmelsen i 1. pkt. finder ikke anvendelse, såfremt kunstværket er hovedmotivet og gengivelsen udnyttes erhvervsmæssigt."}
\end{quote}
Dette betyder at man gerne må offentliggøre fotografier af gavlmaleriet da det befinder sig et offentligt sted og kunstværket ikke er hovedmotivet på gengivelsen der er lagt på hjemmesiden. Om det bruges erhvervsmæssigt kan selvfølgelig diskuteres, men det er underordnet i denne sammenhæng.


\section*{Spørgsmål 3}
Idet Mathias har købt licensen, kan der ifølge [3, s. 15] bruges konsumptionsreglen under § $19$,.
\begin{quote}
\textit{"Når et eksemplar af et værk med ophavsmandens samtykke er solgt eller på anden måde overdraget til andre inden for Det Europæiske Økonomiske Samarbejdsområde, må eksemplaret spredes videre ..."}
\end{quote}
Altså må programmet gerne spredes og derved videresælges. \\
Der må dog ikke laves udlejninger eller udlån af programmet. Desuden betyder det athvis  han sælger licensen videre kan han ikke selv gøre brug af programmet da der i så fald ville mangle en licens.

\section*{Spørgsmål 4}
I [1, s. 161] beskrives \textit{open source} software som computerprogrammer, hvor, når en person har købt programmet, det er muligt at ændre i kildekoden. Altså justere eller forbedre programmet uden at der sker nogen ophavsretskrænkelser. Det skal ikke blandes sammen med \textit{Freeware} som er programmer der er gratis at bruge men hvor ophavsmanden stadig har sine rettigheder.\\
På baggrund af dette må Mathias altså gerne ændre og bruge bestillingsmodulet på denne måde. Det skal dog pointeres at han skal have erhvervet sig modulet på en legal måde.

\section*{Spørgsmål 5}
I denne situation har Sune ret til vederlag for koden. Dette skyldes, at på trods af Sune før har arbejde sammen med Data-snedkeren, står der ikke at modulet som Mathias har brugt er lavet i et af disse samarbejder. Og selv i et samarbejde kunne der være lavet en skriftlig aftale om ophavsrettighederne. Dette hører altså ind under § $1$, hvor værket er beskyttet af Sune.\\
Hvis derimod Sune havde lavet det under Data-snedkeren og ingen særlige aftaler var indgået hører det ifølge [1, s. 164-165] under § $59$.
\begin{quote}
\textit{"Ophavsretten til et edb-program, der er frembragt af en arbejdstager under udførelsen af dennes arbejde eller efter arbejdsgiverens anvisninger, overgår til arbejdsgiveren."}
\end{quote}
Og så ville Mathias stå med rettighederne og ville i så fald kunne justere i koden og lave nye eksemplarer mm.\\
Men uden disse relationer eller uden andre forhold (såsom at det var gratis open source software), må Mathias ikke bruge det og licensen er derfor ikke lovlig.

\section*{Spørgsmål 6}
Ligesom nævnt i (5) har Mathias ikke ophavsrettighederne (fra § $59$ i [2]) medmindre der er indgået en særlig skriftlig aftale. Dog har han stadig adgang til systemerne, så under antagelse af at dette er berettiget skal der bruges § $36$ og § $37$ som også er anført i [2]. \\
Ifølge § $36$ pkt 1, har Mathias ret til at lave ændringer når han har adgang til programmet. Men § $37$ stk. 2, pkt. 3 siger:
\begin{quote}
\textit{"De oplysninger, der er indhentet i forbindelse med anvendelsen af stk. 1, må ikke benyttes til udvikling, fremstilling eller markedsføring af et edb-program, der i sin udtryksform i vid udstrækning svarer til det oprindelige, eller til nogen anden handling, som krænker ophavsretten."}
\end{quote}
Altså må programmer der er hentet på denne måde ikke bruges kommercielt. Dette er tilfældet for Mathias, og det er derfor ham der laver en forseelse i at bruge programmet.

\section*{Spørgsmål 7}
Ja, salget er lovligt. Dette skyldes igen § $19$ som også er nævnt i (3).\\
Disse eksemplarer er i første omgang blevet solgt til sønnen til at eje, og han har derfor ret til at sprede det videre.\\
Henriette, moren, har desuden fået tilladelse til at sælge værkerne af sønnen, som har rettighederne.

\section*{Spørgsmål 8}
Dette kan give anledninger til ophavsretslige problemer. Dette skyldes igen konsumptionsreglen § $19$ som kan ses i (3). Denne siger at spillene gerne må doneres, men de må ikke udlejes eller udlånes. \\
Desuden er der andre ophavsretslige problemer, da nogle af spillene jo er kopieringer. Dette er en eksemplarfremstilling og ifølge § $37$ stk. 2, pkt. 3 - der også kan ses i (6) - må disse fremstillinger ikke laves. Disse kopier af spil kan dog diskuteres om de hører under § $36$, pkt. 2. Fra [2]:
\begin{quote}
"Den, der har ret til at benytte et edb-program, må fremstille et sikkerhedseksemplar af programmet for så vidt det er nødvendigt for benyttelsen af det"
\end{quote}
I dette tilfælde skal det også kun være ejeren af spillet alene som gør brug af det. Altså ved at logge på nettet med eget brugernavn og kodeord. Men med både udlån og kopieringer er der væsentlige ophavsretslige problemer.

\section*{Referencer}
[1] Rosenmeier, M. (2014): Ophavsret for begyndere, En bog til ikke-jurister.\\
$ $
[2] Uddrag af ophavsretsloven \\
$ $
[3] Schultz, L. (2015): Retten til edb-programmer og andre digitale værker


\end{document}
