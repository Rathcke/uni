\documentclass[12pt]{article}
\usepackage[utf8]{inputenc}
\usepackage{amsmath}
\usepackage{amssymb}
\usepackage{mathtools}
\usepackage{amsfonts}
\usepackage{lastpage}
\usepackage{tikz}
\usepackage{pdfpages}
\usepackage{gauss}
\usepackage{fancyvrb}
\usepackage{fancyhdr}
\usepackage{graphicx}
\pagestyle{fancy}
\fancyfoot[C]{\footnotesize Page \thepage\ of 3}
\DeclareGraphicsExtensions{.pdf,.png,.jpg}
\title{Elementær Talteori}
\author{Nikolaj Dybdahl Rathcke}
\chead{Nikolaj Dybdahl Rathcke (rfq695)}

\begin{document}
\section*{Elementær Talteori - 3. aflevering}

\subsection*{Opgave 1}
Vi starter med at omskrive ligningen (alle udregninger er mod $263$.
\begin{align*}
x^2 + 20x + 211 &\equiv 0\\
(x+10)^2+111&\equiv 0\\
(x+10)^2&\equiv -111 \\
(x+10)^2&\equiv 152
\end{align*}
Vi finder så om der er en løsning ved brug af legendre symbolet. Vi ser, at 152 har primtals faktoriseringen $152= 2^3*19$.
\begin{align}
(\frac{152}{263})=(\frac{2}{263})(\frac{2}{263})(\frac{2}{263})(\frac{19}{263})=1*1*1*(-1)=-1
\end{align}
Her ser vi, at $263\equiv -1$ (mod $8$), og derved får vi fra theorem 4.1.7 \textbf{[St]} at $(\frac{2}{263})=1$. Yderligere fra samme theorem, får vi at
\begin{align}
(\frac{19}{263})=(-1)^{\frac{19-1}{2}\frac{263-1}{2}}(\frac{263}{19})=(-1)^{1179}(\frac{263}{19})=(-1)(\frac{263}{19})=-(\frac{16}{19})
\end{align}
Hvor $16=2^4$ og $(\frac{2}{19})=-1$ da $19\equiv 3$ (mod $8$) igen fra theorem 4.1.7. Altså hvis vi regner videre på (2) får vi
$$-(\frac{16}{19})=-(\frac{2}{19})(\frac{2}{19})(\frac{2}{19})(\frac{2}{19})=-(-1)^4=-1$$
Og dette konkluderer det næstsidste lighedstegn i (1). Da dette giver $-1$ har ligningen altså ingen heltals løsninger.

\subsection*{Opgave 2}
\textbf{(a)}\\
At den har orden 5, betyder at elementet $c$ skal opløftes i femte potens for at give identitetselementet, $1$, for den multiplikative gruppe.\\
Eftersom gruppen $Z_p^*$ har orden $p-1$ og vi ved at $5|p-1$ siger Cauchy's theorem, at når gruppen er endelig, og $5$ er et primtal der dividerer gruppens orden, at der må være et element, $c$, i gruppen med orden $5$.\\
\\
\textbf{(b)}\\
Hvis vi skriver ligningen ud og laver nogle omskrivninger, får vi
\begin{align*}
(2c+2c^{-2}+1)^2&\equiv 5\:mod\:p \\
4(c^2+c^{-2})+4(c+c^{-1})+4&\equiv 0\:mod\:p \\
4(c^2+c^{-2}+c+c^{-1}+1)&\equiv 0\:mod\:p
\end{align*}
som er ækvivalent med 
\begin{align}
c^2+c^{-2}+c+c^{-1}+1\equiv 0 \:mod\:p \\
c^2+c^{3}+c+c^{4}+1\equiv 0 \:mod\:p \\
c^4+c^{3}+c^2+c^{1}+1\equiv 0 \:mod\:p
\end{align}
Eftersom $c$ har orden 5, betyder det at $c^5-1\equiv 0$ mod $p$. \\
Dette kan også skrives som $(c-1)(c^4+c^{3}+c^2+c^{1}+1)\equiv 0$ mod $p$. Da $c\neq 1$, må ligningen $c^4+c^{3}+c^2+c^{1}+1\equiv 0 \:mod\:p$ altså være sand og derved gælder det at $g^2\equiv 5$ mod $p$.\\
\\
\textbf{c}\\
Da vi kan se, at der altid vil være et element af orden $5$ samt at der er en løsning til $g^2\equiv 5$ mod $p$, må legendre symbolet $(\frac{5}{p})$ altså være $1$.\\
Alternativt kan vi bruge theorem 4.1.7 [\textbf{St}], som kan bruges idet begge tal er primtal, og få
$$(\frac{5}{p})=-1^{\frac{5-1}{2}\frac{p-1}{2}}(\frac{p}{5})=(\frac{p}{5})$$
Sidste lighedstegn er da $-1$ altid vil være opløftet i et lige tal. Det vides desuden, at $p\equiv 1$ mod $5$, altså kan vi reducere det til
$$(\frac{1}{5})=1$$
Og derved er det vist for primtal $p$ på denne form ($5q+1$).

\subsection*{Opgave 3}
Ikke lavet.

\subsection*{Opgave 4}
Vi kan bruge Dirichlet foldning af $|\mu|*\mu$, da begge er aritmetiske funktioner, til at skrive den nye aritmetiske funktion givet ved
$$(|\mu|*\mu)(n)=\sum_{d/n}|\mu|(d)\mu(\frac{n}{d})$$
ved at sætte $\frac{n}{d}=e$ kan vi omskrive til
$$(|\mu|*\mu)(n)=\sum_{de=n}|\mu|(d)\mu(e)$$
Ikke løst.


\subsection*{Opgave 5}
I theorem 5.8 [\textbf{JJ}] er det netop bevist, at 
$$\sum_{d|n}\phi(d)=n$$
Andel del af opgaven, hvor vi vil vise
$$\frac{\phi(n)}{n}=\sum_{d|n}\frac{\mu(d)}{d}$$
trækker vi på korollar 8.7 [\textbf{JJ}] som siger, at
$$\phi(n)=\sum_{d|n}\frac{\mu(d)n}{d}$$
Her kan vi trække $n$ ud af summen da denne kun antager en værdi.
$$\phi(n)=n\sum_{d|n}\frac{\mu(d)}{d}$$
Og herefter dividere på begge sider med $n$
$$\frac{\phi(n)}{n}=\sum_{d|n}\frac{\mu(d)}{d}$$
Og derved er det vist ved simple omskrivninger fra dette korollar.



\end{document}
