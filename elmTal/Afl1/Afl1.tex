\documentclass[12pt]{article}
\usepackage[utf8]{inputenc}
\usepackage{amsmath}
\usepackage{amssymb}
\usepackage{mathtools}
\usepackage{amsfonts}
\usepackage{lastpage}
\usepackage{tikz}
\usepackage{pdfpages}
\usepackage{gauss}
\usepackage{fancyvrb}
\usepackage{fancyhdr}
\usepackage{graphicx}
\pagestyle{fancy}
\fancyfoot[C]{\footnotesize Page \thepage\ of 4}
\DeclareGraphicsExtensions{.pdf,.png,.jpg}
\title{Elementær Talteori}
\author{Nikolaj Dybdahl Rathcke}
\chead{Nikolaj Dybdahl Rathcke (rfq695)}

\begin{document}
\section*{Elementær Talteori - Første aflevering}

\subsection*{\{\textbf{St}\} Opgave 1.3}
Prove that there are infinitely many primes of the form $6x-1$.\\
\\
Vi ved der er uendelige mange primtal.\\
Vi ved at alle tal har en unik primtalsfaktorisering.\\
Desuden ved vi alle tal kan skrives på formen $6x-r, r\in \{0,1,2,3,4,5\}$.\\
\\
Vi ser hurtigt, at hvis $r=\{0,2,3,4\}$ er $6x-r$ dividerbart med enten $2$ eller $3$, og kan derfor ikke være et primtal.\\
Dette udelader primtallene $2,3$ eller primtal på formen $6x-5$ og $6x-1$.\\
For at bevise at der uendelige mange på formen $6x-1$ benytter vi modstrid. Antag at der findes en endelige mængde primtal, $p_1,p_2, .., p_n$, på formen $6x-1$. Vi konstruerer herefter et tal $k=6(p_1p_2..p_n)-1$, som altså selv har formen $6x-1$.\\
Vi observerer at enhvert $p_i \nmid k$ da $p_i\mid (p_1p_2..p_n)\mid 6(p_1p_2..p_n) $ som medfører $p_i\mid 6(p_1p_2..p_n)$, hvorved $p_i\nmid 6(p_1p_2..p_n)-1$ da denne difference skal være mindst $2$ (det mindste primtal). \\
Dette betyder at $k$ kan primfaktoriseres som $q_1q_2..q_m$, hvor alle led er på formen $6x-5$. Dog ses, at hvis alle led er på formen $6x-5$ så er produktet også på formen $6x-5$. Altså må der eksistere et primtal $p\notin (p_1,p_2,..,p_n)$ som er en primfaktor i $k$ hvilket er en modstrid og der må altså eksistere uendelige primtal på formen $6x-1$.

\subsection*{\{\textbf{St}\} Opgave 2.3}
Use Algorithm 2.3.7 to find $x, y\in \mathbb{Z}$ such that $2261x+1275y=17$.\\
\\
Vi bruger fremgangsmåden som er beskrevet i algoritme 2.3.7 \{\textbf{St}\}
\begin{equation*}
\begin{aligned}
2261&=1*1275+986 \\
1275&=1*986+289 \\
986&=3*289+119 \\
289&=2*119+51 \\
119&=2*51+17 \\
51&=3*17+0
\end{aligned}
\begin{aligned}[c]
\;\;\;\;\;\;\;\;\;\;\;\;\;\;\;
\end{aligned}
\begin{aligned}
986&=(1,-1) \\
289&=(0,1)-(1,-1)=(-1,2) \\
119&=(1,-1)-3(-1,2)=(4,-7) \\
51&=(-1,2)-2(4,-7)=(-9,16) \\
17&=(4,-7)-2(-9,16)=(22,-39) \\
0&=(-9,16)-3(22,-39)=(-75,133)
\end{aligned}
\end{equation*}
Derved bliver resten $0$ og algoritmen terminerer. Her er $gcd(a,b)=d=17$, som findes ved at tage $a$ når algoritmen er termineret. En løsning findes ved at tage Bézout koefficienterne fra næstsidste række, nemlig $(x,y)=(22,39)$, eller den generelle løsning
\begin{align*}
x&=x_0+\frac{bn}{d}=22+\frac{1275n}{17} \\
y&=y_0-\frac{an}{d}=-39-\frac{2261n}{17}
\end{align*}
Som er alle løsninger $x,y\in \mathbb{Z}$ til ligningen $2261x+1275y=17$.

\subsection*{\{\textbf{St}\} Opgave 2.10}
Find an integer $x$ such that $37x\equiv 1$ (mod $101$)\\
\\
Vi starter med at bruge Euklids algoritme til at finde $gcd(37,101)$.
\begin{align*}
101&=2*37+27 \\
37&=1*27+10 \\
27&=2*10+7 \\
10&=1*7+3 \\
7&=2*3+1 \\
3&=3*1+0
\end{align*}
Vi ser, at $gcd(37,101) = 1$ og vi kan derfor substituere tilbage, for at finde et $x$ til ligningen  $37x-101y=1$.\\
\begin{align*}
1&=7-2*3=7-2*(10-7)=3*7-2*10 \\
&=3*(27-3*10)-2*10 = 3*27-8*10 \\
&=3*27-8*(37-27)=11*27-8*37 \\
&=11*(101-2*37)-8*37=11*101-30*37 \\
&=-30*37+11*101  
\end{align*}
Vi ser, at $x=-30$ er en løsning, da $-30*37 \equiv 1$ (mod 101), hvilket betyder $-30$ er en invers til $37$ mod $101$.
  
\subsection*{\{\textbf{St}\} Opgave 2.13}
Find an $x\in \mathbb{Z}$ such that $x\equiv -4$ (mod $17$) and $x\equiv 3$ (mod $23$).\\
\\
Vi vil bruge chinese remainder theorem til dette da ligningerne opfylder vi løser ligningssystemet
\begin{align*}
x&\equiv a\: (mod\:m) \\
x&\equiv b\: (mod\:n)
\end{align*}
Hvor $a=3,m=23,b=-4,n=17$.\\
Vi bruger algoritme 2.2.3 \{\textbf{St}\} til at finde et $x$. Et krav for at kunne bruge denne er at $m$ og $n$ er relativt primiske. Da både $17$ og $23$ er primtal, betyder det at dette er opfyldt. Vi kan så finde integers $c,d$ så $cm+dn=1$.
\begin{equation*}
\begin{aligned}
23&=1*17+6 \\
17&=2*6+5 \\
6&=1*5+1 \\
5&=5*1+0
\end{aligned}
\begin{aligned}[c]
\;\;\;\;\;\;\;\;\;\;\;\;\;\;\;
\end{aligned}
\begin{aligned}
6&=(1,-1) \\
5&=(0,1)-2(1,-1)=(-2,3) \\
1&=(1,-1)-(-2,3)=(3,-4) \\
0&=(-2,3)-5(3,-4)=(-17,23)
\end{aligned}
\end{equation*}
Her finder vi $c$ til $3$ og $d$ til $-4$. Anden del af algoritmen giver derved at 
$$x=a+(b-a)cm=3+(-4-3)*3*23=-480$$
Som er det $x$ der løser ligningsystemet.\\
\\
\textit{Note}: Ved dette $x$ returnerer modolu operation altså $-4$ istedet for $13$.

\subsection*{\{\textbf{JJ}\} Opgave 2.6}
Prove that every prime $p\neq 3$ has the form $3q+1$ or $3q+2$ for some integer $q$; prove that there are infinitely many primes of the form $3q+2$\\
\\
Vi ved at alle tal kan skrives som $3q+r,r\in \{0,1,2\}$.\\
Vi ved der kun findes primtal $p$, hvor $p>1$ hvilket betyder $q> 0$.\\
\\
Derved har vi at alle tal med $r=0$ er på formen $3q$ og altså må
\begin{itemize}
\item $3$ gå op i tallet, og de kan derfor ikke være primtal.
\item Tallet være $3$.
\item Tallet være 0.
\end{itemize}
Da $0$ ikke er et primtal og $3$ ikke er inkluderet i beviset, betyder det at alle primtal må være på formen $3q+1$ eller $3q+2$.\\
\\
Vi vil vise, at der uendelige mange primtal på formen $3q+2$ som er ækvivalent med $3q-1$ da de rammer de samme tal. Dette gøres ved modstrid.\\
Vi antager at der findes en endelig mængde primtal på formen $3q-1$ noteret som $p_1,p_2,..,p_n$. Vi konstruerer nu et tal $k=3(p_1p_2..p_n)-1$ som også har formen $3q-1$.\\
Af samme argumentation som i \{\textbf{St, Opgave 1.3}\} ser vi igen at enhvert $p_i \nmid k$ da $p_i\mid (p_1p_2..p_n)\mid 3(p_1p_2..p_n)$. Det medfører $p_i\mid 3(p_1p_2..p_n)$ og derved $p_i\nmid 3(p_1p_2..p_n)-1$ da forskellen skal være mindst $2$ da det er det mindste primtal.\\
Det betyder $k$ kan primfaktoriseres som $q_1q_2..q_m$, hvor alle led er på formen $3q+1$. Igen ses at hvis alle led er på formen $3q+1$ så er produktet også på formen $3q+1$. Der må derfor eksistere et primtal $p\notin (p_1,p_2,..,p_n)$ som er en primfaktor i $k$ hvilket er en modstrid og der må altså eksistere uendelige primtal på formen $3q-1$ og derved også $3q+2$.\\
\\
\textit{Note}: Ideen med at omskrive $3q+2$ til $3q+1$ er for at undgå at $k$ er på formen $3q+2$, da en primtalsfaktor kunne være $2$ som også har formen $3q+2$.

\end{document}
