\documentclass[12pt]{article}
\usepackage[utf8]{inputenc}
\usepackage{amsmath}
\usepackage{amssymb}
\usepackage{mathtools}
\usepackage{amsfonts}
\usepackage{lastpage}
\usepackage{tikz}
\usepackage{pdfpages}
\usepackage{gauss}
\usepackage{fancyvrb}
\usepackage{fancyhdr}
\usepackage{graphicx}
\pagestyle{fancy}
\fancyfoot[C]{\footnotesize Page \thepage\ of 4}
\DeclareGraphicsExtensions{.pdf,.png,.jpg}
\title{Elementær Talteori}
\author{Nikolaj Dybdahl Rathcke}
\chead{Nikolaj Dybdahl Rathcke (rfq695)}

\begin{document}
\section*{Elementær Talteori - 2. aflevering}

\subsection*{Opgave 1}
Antag at $a$ er en primitiv rod modulo $p$, hvor $p$ er et primtal. Vis at $a^k$ er en primitiv rod modulo $p$ hvis og kun hvis $gcd(k, p-1) = 1$. Brug dette til at bevise \{\textbf{St}\} Prop 2.5.12 når $n=p$.\\
\\
Vi starter med at vise den første implikation, at hvis $gcd(k,p-1)=1$ så er $a^k$ en primitiv rod mod $p$.\\
Dette vises ved at vise der findes et $n$ således at $a^{k^n}\equiv a$ (mod $p$), da de derfor må generere samme elementer i sættet $\mathbb{Z}/p\mathbb{Z}$ og derved må $a^k$ også være en primitiv rod.\\
At $gcd(k,p-1)=1$ betyder at vi kan opstille en ligning $kx+(p-1)y=1$ da der findes  $x$ og $y$ der løser lignignen. Da begge sider er $1$ kan vi opløfte begge sider som potens af $a$, så vi får
\begin{align*}
a^{(kx+(p-1)y)}& \equiv a^1 \:(mod\:p)\\
a^{kx}a^{(p-1)y}& \equiv a \:(mod\:p) \\
(a^k)^x(a^{(p-1)})^y& \equiv a \:(mod\:p) \\
\end{align*}
Da $p-1=\varphi(p)$ får vi fra Eulers theorem at $a^{\varphi (p)}\equiv 1$ (mod $p$).
\begin{align*}
(a^k)^x(a^{\varphi})^y&\equiv a \:(mod\:p)  \\
(a^k)^x(1^y)&\equiv a \:(mod\:p)  \\
(a^k)^x&\equiv a \:(mod\:p) 
\end{align*}
Og vi finder et $x$ som er det $n$ vi ønskede at finde.\\
\\
Vi skal desuden vise den modsatte implikation, at hvis $a^k$ er en primitiv rod af $p$, så er $gcd(k,p-1)=1$.\\
Vi ved at der må findes et $n$ så $a^{nk}\equiv a$ (mod $p$) eller at $a^{nk-1}\equiv 1$ (mod $p$).\\
Vi kan skrive $nk-1=m*\varphi(p)+r$ (mod $p$).\\
Vi vil vise at $r=0$ da dette betyder vi får $nk-1=m*\varphi(p)$ eller $nk+m*\varphi(p)=1$ som så betyder at $gcd(k,\varphi(p))=gcd(k,p-1)=1$.\\
Ved at sætte begge sider af $nk-1=m*\varphi(p)+r$ i potens af $a$ får vi
\begin{align*}
a^{nk-1}&\equiv a^{m*\varphi(p)+r} \:(mod\:p) \\
a^{nk-1}&\equiv (a^{\varphi(p)})^ma^r  \:(mod\:p)
\end{align*}
Igen bruger vi Eulers theorem og får så
\begin{align*}
a^{nk-1}&\equiv (a^{\varphi(p)})^ma^r \:(mod\:p) \\
a^{nk-1}&\equiv 1^ma^r \:(mod\:p) \\
a^{nk-1}&\equiv a^r \:(mod\:p)
\end{align*}
Vi ved desuden, at $a^{nk-1}\equiv 1$ (mod $p$). Og får så
\begin{align*}
1&\equiv a^r \:(mod\:p)
\end{align*}
Dette medfører at $r=0$ og derfor gælder denne implikation også.\\
\\
Da alle primitive rødder vil være på formen $a^k$ med et $k$ der er indbyrdes primiske med $p-1$ betyder det at der vil være netop så mange primitive rødder som der er $k$ indbyrdes primiske med $p-1$, hvilket jo netop er $\varphi(p-1)$. Desuden er $\varphi(p)=p-1$ for et primtal $p$. Altså må \{\textbf{St}\} Prop 2.5.12 gælde.

\subsection*{Opgave 2}
Find alle primitive rødder modulo $19$.\\
\\
Vi bruger algoritme 2.5.16 i \{\textbf{St}\} til at finde de primitive rødder. Vi starter med at finde primtals faktoriseringen af $\varphi(p)=p-1=18$. \\
Denne er $2*3^2$, altså er faktorene $p_i=\{2,3\}$ og der er $\varphi(18)=18(1-\frac{1}{2})(1-\frac{1}{3})=6$ primitive rødder.\\
Der skal altså gælde, at et tal $a$ er en primitiv rod, hvis der gælder $a^{(p-1)/p_i} \not\equiv 1$ (mod $p$) for alle $p_i$, hvor vi kigger på $2\leq a\leq 18$. Følgende udregninger er modulo $19$.\\
\\
\begin{equation*}
\begin{aligned}
2^{18/2}&=18 \\
3^{18/2}&=18 \\
4^{18/2}&=1 \\
5^{18/2}&=1 \\
6^{18/2}&=1 \\
7^{18/2}&=1 \\
8^{18/2}&=18 \\
9^{18/2}&=1 \\
10^{18/2}&=18 \\
11^{18/2}&=1 \\
12^{18/2}&=18 \\
13^{18/2}&=18 \\
14^{18/2}&=18 \\
15^{18/2}&=18 \\
16^{18/2}&=1 \\
17^{18/2}&=1 \\
18^{18/2}&=18
\end{aligned}
\begin{aligned}[c]
\;\;\;\;\;\;\;\;\;\;\;\;\;\;\;
\end{aligned}
\begin{aligned}
2^{18/3}&=7 \\ 
3^{18/3}&=7 \\
\\
\\
\\
\\
8^{18/3}&=1 \\
\\
10^{18/3}&=11 \\
\\
12^{18/3}&=1 \\
13^{18/3}&=11 \\
14^{18/3}&=7 \\
15^{18/3}&=11 \\
\\
\\
18^{18/3}&=1
\end{aligned}
\end{equation*}
Hvoraf de $6$ tal der ikke giver $1$ (mod $19$) er nogle af tilfældende er tallene 2,3,10,13,14 og 15, som altså er de primiske rødder modulo $19$.

\subsection*{Opgave 3}
Antag at $a=-1$ eller $a$ er en perfekt kvadrat. Vis at $a$ ikke er en primitiv rod modulo $p$ for noget primtal $p >3$. Bemærk at dette forklarer antagelsen i Artins formodning (\{\textbf{St}\} Conjecture 2.5.14). Afgør desuden hvorvidt $a$ er en primitiv rod modulo $3$.\\
\\
Hvis $a=-1$ set det let, at $a$ ikke kan generere hele den multiplikative gruppe da den kun danner elementerne $1$ og $p-1$. Så hvis $p>3$ kan den ikke være en primitiv rod. Hvis $p=3$ vil $a=-1$ kunne være en primitiv rod, idet den genererer både $1$ og $-1=p-1=2$.\\
Hvis $a$ er et perfekt kvadrat. For at $a$ er en primitiv rod, betyder det altså, at $a,a^2,a^3,..,a^{(p-1)}$ genererer alle tal $1,2,3,..,p-1$. Der gælder desuden at $a^{(p-1)} \equiv 1$ (mod $p$). \\
For at vise at et perfekt kvadrat ikke er en primitiv rod, skal vi blot finde et modeksempel. Hvis $2$ potenser af $a$ giver samme rest mod $p$ betyder det at hele den multiplikative gruppe $\mathbb{Z}/p\mathbb{Z}$ ikke bliver genereret.\\
Vi ser at $(a^2)^{(p-1)/2}=a^{p-1}\equiv 1$ (mod $p$). Men vi ved allerede at $(a^2)^{(p-1)}\equiv 1$ (mod $p$). Altså giver disse to potenser ($p-1$ og $\frac{p-1}{2}$) det samme resultat og hele den multiplikative gruppe bliver ikke genereret og dermed kan et perfekt kvadrat ikke være en primitiv rod.\\
Hvis $p=3$ holder dette stadig ikke af samme argumentering da $(b^2)^{(p-1)/2}=(b^2)^{(p-1)}$ hvor vi ved disse to potenser er forskellige da $p-1=2$.

\subsection*{Opgave 4}
Vis den modsatte implikation af Miller-Rabin sætningen. Altså at hvis der for alle
$a \not\equiv 0$ mod $p$ gælder sætning (1) så er $p$ et primtal.\\
\\
Vi starter med at vise den første betingelse holder.
\begin{align*}
a^m &\equiv 1\:(mod\:p)
\end{align*}
Ikke lavet. \\
\textit{Ide:} At vise det ved hjælp af Fermats theorem, hvor $a^{p-1}\equiv 1$ (mod $p$) og derved konkludere at i tilfælde hvor $m$ kan skrives som sådan er $p$ et primtal.\\
\\
Og herefter, at den anden betingelse holder
\begin{align*}
a^{2^rm}\equiv -1\:(mod\:p), 0\leq r < k
\end{align*}
Ikke lavet. \\
\textit{Ide:} Vis at venstre side vil kunne skrives som $p-1$ i tilfælde hvor første betingelse ikke viste noget.

\end{document}
